\section{Motivation and First Examples}

We begin by asking ourselves what kinds of topologies we can endow sets of functions with. We will be particularly interested in metrisable topologies and even more so in completely metrisable topologies. We have already seen one very natural topology in cases where $X$ is an arbitrary set and $Y$ has a topology. We will investigate this further and study completeness properties in the process.

\subsection{The Topology of Pointwise Convergence}

Let $Y$ be a topological space and $X$ an arbitrary set.

\begin{boxdefinition}[The Topology of Pointwise Convergence]\label{Ch4:Def:Top_Pointwise_Conv}
    The \textbf{topology of pointwise convergence} on $Y^X$ is defined to be the product topology on $Y^X$.
\end{boxdefinition}

We can give a characterisation of this topology in terms of nets.

\begin{boxexercise}
    Let $\parenth{f_a}_{a \in \D}$ be a net in $Y^X$. For all $f \in Y^X$, TFAE:
    \begin{enumerate}[label = (\arabic*)]
        \item $f_a \to f$
        \item For all $x \in X$, $f_a(x) \to f(x)$
    \end{enumerate}
    Note that $(2)$ is a sensible statement to make because for all $x \in X$, $\parenth{f_a(x)}_{a \in \D}$ is a net in $Y$.
\end{boxexercise}

The big advantage of this topology is that it is very general: it does not require $X$ to have any topological structure whatsoever.

But pointwise convergence is not the strongest notion of convergence out there. We can define uniform convergence of nets, and then define a topology of uniform convergence.

\subsection{The Topology of Uniform Convergence}

Consider the following setup.

Let $X$ be a set and let $\parenth{Y, d}$ be a metric space. Define a new metric $\bar{d}$ on $Y$ by
\begin{align*}
    \bar{d}\of{y_1, y_2} :=
    \begin{cases}
        d\of{y_1, y_2} & \text{ if } d\of{y_1, y_2} < 1 \\
        1 & \text{ if } d\of{y_1, y_2} \geq 1
    \end{cases}
    = \min\of{d\of{y_1, y_2}, 1}
\end{align*}
One way to view this new metric space is as the original metric space `squashed' to a ball of radius $1$. It is a `trivial exercise' to show that this is a metric.

Note that $\parenth{Y, d}$ and $\parenth{Y, \bar{d}}$ have the same open sets. As a result, one can show they also have the same Cauchy sequences and the same convergent sequences.

Along these lines, we can define a `uniform metric' and `uniform topology' on $Y^X$.

\begin{boxdefinition}[The Topology of Uniform Convergence]\label{Ch4:Def:Top_Unif_Conv}
    Let $X$ be a set and $\parenth{Y, d}$ be a metric space. Denote by $\bar{d}$ the metric described above. The \textbf{uniform metric} is defined for all $f_1, f_2 \in Y^X$ by
    \begin{align*}
        \dunif\of{f_1, f_2} := \sup \setst{\bar{d}\of{f_1(x), f_2(x)}}{x \in X}
    \end{align*}
    We call the induced topology the \textbf{uniform topology} or the \textbf{topology of uniform convergence}.
\end{boxdefinition}

The reason for this terminology is that a sequence of functions converges with respect to the uniform metric obtained from $d$ if and only if it converges uniformly with respect to $d$.

It turns out that the uniform metric construction preserves completeness.

\begin{boxproposition}
    If $\parenth{Y, d}$ is a complete metric space and $X$ is an arbitrary set, then $\parenth{Y^X, \dunif}$.
\end{boxproposition}
\begin{proof}
    Let $\parenth{f_n}_{n \in \N}$ be Cauchy in $\parenth{Y^X, d\unif}$. It is easy to see that for all $x \in X$, the sequence $\parenth{f_n(x)}_{n \in \N}$ is convergent in $\parenth{Y, d}$. Since this metric space is complete, $\parenth{f_n(x)}_{n \in \N}$ converges to some limit. We can therefore define
    \begin{align*}
        f : X \to Y : x \mapsto \lim_{n \to \infty} f_n(x)
    \end{align*}
    ie, we take $f$ to be the pointwise limit of the $\parenth{f_n}_{n \in \N}$. We will show that in $\parenth{Y, d\unif}$, the (Cauchy) sequence $\parenth{f_n}_{n \in \N}$ converges to $f$.

    Fix $\eps > 0$ and assume that $\eps < 1$\footnote{We don't need to do this right away, but it makes the rest of the argument significantly simpler because $d\unif$ is defined in terms of $\bar{d}$, and $\bar{d}$ always takes values $\leq 1$.}. Since $\parenth{f_n}_{n \in \N}$ is Cauchy, we can find some $N \in \N$ such that for all $n_1, n_2 \geq N$, $d\unif\of{f_{n_1}, f_{n_2}} < \infty$. Then, by definition of the uniform metric, for all $x \in X$, $d\of{f_{n_1}(x) ,  f_{n_2}(x)} < \eps$.
    
    Send $n_2 \to \infty$. Then, $f_{n_2}(x) \to f(x)$. Since $d$ is continuous, we can see that $d\of{f_{n_1}(x), f(x)} \leq \eps$. Since this is true for all $x \in X$, it follows that $d\unif\of{f_{n_1}, f} \leq \eps$ for all $n_1 \geq N$. This is enough, because the real numbers ``have lots of room'' - so even if we ``replace the OG epsilon by a smaller one'' we are fine! Yay :)
\end{proof}

We now move to a slightly different context.

\begin{boxnotation}
    For all topological (and metric) spaces $X$ and $Y$, denote by $C(X, Y)$ the set of all continuous functions from $X$ to $Y$.
\end{boxnotation}

For the remainder of this subsection, let $X$ be a topological space and let $(Y, d)$ be a metric space. Since $C(X, Y)$ is a subset of $Y^X$ and $d\unif$ is a metric on $Y^X$, we can restrict $d\unif$ to $C(X, Y)$ and thus view $\parenth{C(X, Y), d\unif}$ as a metric space in its own right.

The following ``is like a really important fact''.

\begin{boxtheorem}\label{Ch4:Thm:Unif_lim_cont_cont}
    Let $\parenth{f_n}_{n \in \N}$ be a sequence in $C(X, Y)$, and assume that there is some $f \in Y^X$ such that $f_n \to f$ with respect to $d\unif$. Then, $f$ is continuous, that is, we can view $f$ as an element of $C(X, Y)$.
\end{boxtheorem}
\begin{proof}
    To show that $f$ is continuous at all $x\in X$, we let $\eps>0, \eps<1$, and $n$ such that $d_{unif}(f_n, f)<\ve/3$ - note the $\ve/3$ providing evidence that we are thinking ahead! We now know that $f_n$ is continuous at $x$, so we find some open set $U$ containing $x$ for which we have $d(f_n(x), f_n(x'))<\ve/3$, and also $d(f_n(x), f_n(x'))<\ve/3$ for all $x,x'\in U$. Noting now that $d(f(x), f(x'))\leq d(f(x), f_n(x))+d(f_n(x), f_n(x'))+d(f(x'), f_n(x'))\leq 3\cdot \eps/3=\eps$. 
\end{proof}

The above is best summarised by saying that \textbf{a uniform limit of continuous functions is continuous}.

\begin{boxcorollary}
    Let $X$ be an arbitrary topological space and let $\parenth{Y, d}$ be a complete metric space. The metric space $\parenth{C(X, Y), d\unif}$ is complete.
\end{boxcorollary}
\begin{proof}
    Let $\parenth{f_n}_{n \in \N}$ be a Cauchy sequence in $C(X, Y)$. As before, if we define
    \begin{align*}
        f : X \to Y : x \mapsto \lim_{n \to \infty} f(x)
    \end{align*}
    then we know that $f_n \to f$ in $\parenth{Y^X, d\unif}$. \Cref{Ch4:Thm:Unif_lim_cont_cont} then tells us that $f$ is continuous, so $f_n \to f$ in $\parenth{C(X, Y), d\unif}$.
\end{proof}

% Not quite sure where to put the following

\subsection{An Extension Property for Normal Spaces}

The uniform metric has many applications. In this subsection, we will investigate one of them: an extension property for normal spaces.

\begin{boxtheorem}[Tietze Extension Theorem]\label{SP:Tietze}
    Let $X$ be a normal space and let $A \ssq X$ be closed. Let $f : A \to \R$ be continuous. There exists some $g : X \to \R$ continuous such that $g \restriction A = f$.
\end{boxtheorem}
\begin{quote}
    \textit{``The zen of the proof is that I'm going to reduce the complexity of the situation just a little bit, and then build, pretty much by hand, a Cauchy sequence of continuous functions that converge with respect to the uniform metric. I'm then going to take a limit and get the function that I want.''}
\end{quote}
\begin{proof}
    Let $h : \R \to \parenth{-1, 1}$ be a homeomorphism. Replacing $f$ by $h \circ f$, we may assume that $f$ is bounded. For any real-valued function $g$, define the following notation:
    \begin{align*}
        \norm{g}_A &:= \sup_{a \in A} \abs{g(a)} \\
        \norm{g}_X &:= \sup_{x \in X} \abs{g(x)} 
    \end{align*}
    We construct a sequence of functions $\parenth{f_n}_{n \in \N}$, the limit of which will be our candidate.

    Define $f_0 : X \to \R$ by $f_0 = 0$. Find $c_0 > 0$ such that $\norm{f - f_0}_{A} = \norm{f}_A \leq c_0$. This is something we can do because $f$ is bounded. We will proceed by recursion.

    Suppose we have found a continuous function $f_n : X \to \R$ such that $\norm{f - f_n}_A \leq c_n$. Define the following disjoint subsets of $A$:
    \begin{align*}
        A_n^- &:= \setst{a \in A}{f(a) - f_n(a) \in \brac{-c_n, -\frac{c_n}{3}}} \\
        A_n^+ &:= \setst{a \in A}{f(a) - f_n(a) \in \brac{\frac{c_n}{3}}, c_n}
    \end{align*}
    Since $A$ is closed and $f - f_n$ is continuous, $A_n^-$ and $A_n^+$ are both disjoint, closed subsets of $X$. Since $X$ is normal, we can apply Urysohn's Lemma (\Cref{SP:Urysohn}) to find some continuous $\phi_n : X \to \brac{0, 1}$ with $\phi_n \restriction A_n^- = 0$ and $\phi_n \restriction A_n^+ = 1$.

    Define\footnote{This is the sort of thing that would \textit{not} feature in Anand/Tim's database of motivated proofs...}
    \begin{align*}
        g_n := \frac{2c_n}{3} \phi_n - \frac{c_n}{3}
    \end{align*}
    Observe that on $A_n^-$, $g_n$ is constant with value $-\frac{c_n}{3}$, and on $A_n^+$, $g_n$ is constant with value $\frac{c_n}{3}$. Indeed, $g_n(x) \in \brac{-\frac{c_n}{3}, \frac{c_n}{3}}$ for all $x \in X$. That is, $\norm{g_n} \leq \frac{c_n}{3}$.

    Let $f_{n + 1} := f_n + g_n$. Let's examine how $f - f_{n + 1}$ behaves on $A$. Indeed, $f - f_{n + 1} = \parenth{f - f_{n}} - g_n$. For all $a \in A$, we can see that $\abs{f(a) - f_{n + 1}(a)} \leq \frac{2 c_n}{3}$, so indeed $\norm{f - f_{n + 1}}_A \leq \frac{2 c_n}{3}$.

    We can then perform the recursive construction choosing $c_{n + 1} = \frac{2 c_n}{3}$, so we can see that for all $n \in \N$, $c_n = \parenth{\frac{2}{3}}^n c_0$.

    By an `easy calculation' one can show that $\parenth{f_n}_{n \in \N}$ is Cauchy with respect to the uniform metric. \Cref{Ch4:Thm:Unif_lim_cont_cont} then tells us that this limit, denoted $g : X \to \R$, is indeed continuous. Since $\norm{f - f_n}_A \leq c_n$ for all $n \in \N$ and $c_n \to 0$ as $n \to \infty$, we can conclude that $g \restriction A = f$.
\end{proof}

Note that it is \textbf{vital} that $A$ be closed in the above theorem. If this is not true, the conclusion is no longer true.

\begin{boxcexample}
    Let $X = S^1$, the circle. Consider a point $p \in X$ and define $A := X \setminus \set{p}$.
    % \begin{figure}[H]
    %     \centering
    %     \begin{tikzpicture}
    %         \draw (0, 0) circle 1;
    %     \end{tikzpicture}
    % \end{figure}
\end{boxcexample}

% We have most of this already - the minute he starts saying stuff we don't have, I'll start (-sounds good, and hope you have been well this past week - glad to hear it (and indeed, thanks for asking (and back to topology :))))(just to be safe))))))

% I've been great, thanks! Haven't seen you around - hope you've been well too :)

% :D

\section{Families of Functions}

Professor Cummings, in his own words, is about to bore us with a long list of topologies. But I highly doubt they'll be boring... something tells me they'll be quite the opposite!

\subsection{Equicontinuity}

Throughout this subsection, fix a topological space $X$ and a metric space $\parenth{Y, d}$.

\begin{boxdefinition}[Equicontinuity]
    Fix a family $\F \ssq Y^X$.
    \begin{enumerate}
        \item For some point $x_0 \in X$, the family $\F$ is \textbf{equicontinuous at $x_0$} iff for all $\eps > 0$, there is an open neighbourhood $U \ni x_0$ such that for all $f \in \F$ and $x \in U$, $d\of{f\of{x_0}, f(x)} < \eps$.
        \item $\F$ is \textbf{equicontinuous} if $\F$ is equicontinuous at all points in $X$.
    \end{enumerate}
\end{boxdefinition}

Note that equicontinuity is \textbf{not the same notion as uniform continuity}: we are describing a family of functions \textit{collectively} when we talk about equicontinuity. Moreover, they're continuous ``in the same way'' (ie, for the same $\eps$, the same conclusions hold \textit{simultaneously} for \textit{all of them}).

Here is a motivating theorem.

\begin{boxtheorem}\label{Ch4:Thm:Equicontinuous_of_Tot_Unif_Bdd}
    Fix a family $\F \ssq C\of{X, Y}$ of continuous functions from $X$ to $Y$. If $\F$, viewed as a sub-metric space of the space $\parenth{Y^X, d\unif}$, is totally bounded, then $\F$ is equicontinuous.
\end{boxtheorem}

One may recall we had ``a rather arduous'' discussion of compactness in metric spaces, which involved discussion of ``total-boundedness'' (which stated that for any fixed $\ve>0$, we could cover our space with finitely many $\ve$-balls - this condition can be seen to capture the idea of finite volume). Now we can view $F\ssq C(X,Y)$ with $C(X,Y)$ considered as a metric space with the uniform metric, and then if $F$ is totally bounded in the uniform metric, our $F$ must be equicontinuous.

\begin{proof}[Proof of \Cref{Ch4:Thm:Equicontinuous_of_Tot_Unif_Bdd}.]
    ``This is one of those proofs that writes itself as long as you're a bit careful and do things in the right order.''

    Fix $0 < \eps < 1$ (we are working in the uniform metric, so we can take $\eps < 1$). Fix an arbitrary point $x_0 \in X$. We will show that $\F$ is equicontinuous at $x_0$.
    
    Let $\delta = \frac{\eps}{3}$. Note that $0 < \delta  < 1$ also. Since $\F$ is totally bounded, $\F$ is expressible as a finite union of $\delta$-balls: there exist $f_1, \ldots, f_n \in \F$ such that
    \begin{align*}
        \F \ssq \bigcup_{i = 1}^{n} B_{d\unif}\of{f_i, \delta}
    \end{align*}
    The key to uniform boundedness is that we get a finite family that covers $\F$. We can now do the following.
    
    For each $i$, $f_i$ is continuous at $x_0$, so we can find open neighbourhoods $U_i \ni x_0$ such that for all $x \in U_i$, $d\of{f_i\of{x_0}, f_i\of{x}} < \delta$, with $d$ being the metric on $Y$.
    
    Let $U = \bigcap_{i=1}^{n} U_i$. We know that $U$ is open and that $x_0 \in U$ because $x_0 \in U_i$ for every $i \in \set{1, \ldots, n}$. We're now ready to show that $\F$ is equicontinuous at $x_0$.

    We now consider an arbitrary function $f\in \F$ and an arbitrary $x\in U$, and find $1 \leq i\leq n$ such that $d\unif \of{f, f_i}<\delta$. For our fixed $x\in U$, we know that
    \begin{align*}
        d\of{f_i(x), f_i\of{x_0}} &< \delta \\
        d\of{f(x), f_i\of{x}} &< \delta \\
        d\of{f\of{x_0}, f_i\of{x_0}} &< \delta
    \end{align*}
    Applying the triangle inequality, we can see that
    \begin{align*}
        d\of{f(x), f\of{x_0}}
        \leq d\of{f_i(x), f_i\of{x_0}} + d\of{f(x), f_i\of{x}} + d\of{f\of{x_0}, f_i\of{x_0}}
        < 3\delta = \eps
    \end{align*}
    Since $\eps < 1$, $d\of{f(x), f\of{x_0}} = \bar{d}\of{f(x), f\of{x_0}}$, and we're done. % Maybe fine-tune a bit
\end{proof}

Recall that the \textit{product topology} on $Y^X$ (where $Y$ is a space and $X$ is a set) is called the \textbf{``topology of pointwise convergence''} (cf. \Cref{Ch4:Def:Top_Pointwise_Conv}). We also recall that the \textit{metric topology} on $Y^X$ (where $Y$ a metric space and $X$ a set) which is induced by the uniform metric is called the \textbf{``topology of uniform convergence''} (cf. \Cref{Ch4:Def:Top_Pointwise_Conv}).

%idk how you want to format these ``recalls'' - sounds good - this is true, idk if it would be nice to have them all listed at some point (i.e. itemized), but you are the creative mind here

% I think just as plain text is fine - with maybe a cross-reference to the original one... might be overkill to have formatting for this too (we don't want to be in a situation where there's no normal text, only boxed text :D)

\subsection{A Study in Compactness}

It's time for another topology! After all, there can never be enough topologies...

As usual, let $X$ be a topological space and let $\parenth{Y, d}$ be a metric space. A word of warning: the following definition is one that'll ``leave a bit of work to you.''

\begin{boxdefinition}[The Topology of Compact Convergence]\label{Ch4:Def:Top_Compact_Conv}
    The \textbf{topology of compact convergence} on $Y^X$ is the topology generated by the basic open sets
    \begin{align*}
        B_C\of{f, \eps} &:=
        \setst{g \in Y^X}{\sup_{x \in C}\of{d\of{f(x), g(x)}} < \eps}
    \end{align*}
    indexed by
    \begin{itemize}
        \item Compact sets $C \ssq X$
        \item Functions $f \in Y^X$
        \item $\eps > 0$
    \end{itemize}
\end{boxdefinition}

The reason this definition requires some work is that we need to show that the above `basic open sets' do, indeed, form a basis for some topology on $Y^X$, by showing the appropriate covering property and the appropriate intersection property. This is an ``amusing little exercise.''

\begin{boxexercise}[An Amusing Little Exercise using Properties of Compact Sets]
    Show that the sets $B_C\of{f, \eps}$ defined in \Cref{Ch4:Def:Top_Compact_Conv} do, indeed, form the basis of a topology.
\end{boxexercise}

This is useful because in complex analysis, we know that if sequences of holomorphic functions converge with respect to this topology of compact convergence, their limit is holomorphic too.

\begin{boxdefinition}[Compactly Generated Space]
    We say that the space $X$ is ``compactly generated'' to mean that for all $A\ssq X$, $A$ is open in $X$ exactly when, for all compact $C\ssq X$ we have $A\cap C$ relatively open in $C$.
\end{boxdefinition}
One can think of a compactly generated space as a space in which the topological properties of our space are captured by its compact subspaces.

\begin{boxproposition}
    If $X$ is locally compact or first-countable, then $X$ is compactly generated.
\end{boxproposition}
\begin{proof}
    We show each case separately.
    \begin{enumerate}[label = \underline{Case \arabic*:}]
        \item \underline{$X$ is locally compact.}

        Let $A \ssq X$ be such that $A \cap C$ is relatively open for all compact $C \ssq X$. Fix $x \in A$. Since $X$ is locally compact, there is some compact set $C \ssq X$ and some open set $U \ssq C$ such that $x \in U$. $A \cap C$ is relatively open in $C$, so $A \cap U$ is open. Thus, $A$ is open.

        \item \underline{$X$ is first-countable.}

        Let $B \ssq X$ be such that $B \cap C$ is relatively closed for all compact $C \ssq X$. It will now suffice to show that $B$ is closed. To do so, we let $x\in \cl(B)$, and let $(U_n)_{n\in \N}$ be a countable neighbourhood basis of open sets - by a little topological joke, we observe that $V_n=\cap_{i\leq n}U_i$ will also form a (nested) countable neighbourhood basis for $x$. We now let $x_n\in B\cap V_n$ for each $n\in \N$, and note that the sequence of $(x_n)$ converges to $x$ - finish the proof as an exercise? \sorry %(!) % It's fine - we'll do it another time :D
    \end{enumerate}
\end{proof}

% Sorry I just vanished - things a-happening on GitHub, PRs waiting to get merged... anxious contributors... you get the idea. Thank you!!

%not at all

``by an ancient and seemingly pointless homework exercise''
``I'm getting close to the danger zone here [running out of time] - EASY!''