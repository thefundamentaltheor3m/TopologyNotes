\section{Homotopy}

Next, we take our first step on the path to defining the fundamental group(oid): defining paths!

Throughout this section, let $X$ be a topological space. A path is exactly what one would think.

\begin{boxdefinition}[Path]
    Fix points $x_0, x_1 \in X$. A \textbf{path in $X$ from $x_0$ to $x_1$} is a continuous function $\gamma : [0, 1] \to X$ such that $\gamma(0) = x_0$ and $\gamma(1) = x_1$.
\end{boxdefinition}

A loop is also exactly what one would think.

\begin{boxdefinition}[Loop]
    Fix a point $x \in X$. A \textbf{loop in $X$ centred at $x$} is a path from $x$ to $x$.
\end{boxdefinition}

\subsection{A Word on Paths}

It turns out that paths can be composed, provided they are `compatible'.

\begin{proposition}[Composing Paths]
    Fix points $x_0, x_1, x_2 \in X$. If $\gamma$ is a path from $x_0$ to $x_1$ and $\delta$ is a path from $x_1$ to $x_2$, then the function
    \begin{align*}
        \eps : [0, 1] \to X : t \mapsto
        \begin{cases}
            \gamma\of{2t} & \text{ if } 0 \leq t \leq \frac{1}{2} \\
            \delta\of{2t - 1} & \text{ if } \frac{1}{2} \leq t \leq 1
        \end{cases}
    \end{align*}
    is a path between $x_0$ and $x_2$.
\end{proposition}
\begin{proof}
    It is clear that $\eps$ is a well-defined function: at the point $t = \frac{1}{2}$,
    \begin{align*}
        \gamma\of{2t} = \gamma\of{1} = x_1 = \delta\of{0} = \delta\of{2t - 1}
    \end{align*}
    So indeed the function is well-defined.

    All we need to show is that $\eps$ is continuous. But this is true because of a homework exercise\todo{Add sth here} in which we showed that if we have a space that is a union of two closed sets then we can continuous `glue' together continuous functions that agree on the overlap of these closed sets. So $\eps$ is the `gluing' of $\gamma$ of $\delta$, and is hence continuous.
\end{proof}

\begin{boxdefinition}[Composition of Paths]
    Fix points $x_0, x_1, x_2 \in X$. If $\gamma$ is a path from $x_0$ to $x_1$ and $\delta$ is a path from $x_1$ to $x_2$, then the function $\eps : [0, 1] \to X$ defined as above is called the \textbf{composition} of $\gamma$ and $\delta$. We denote it $\delta \star \gamma$ or simply $\delta \gamma$.
\end{boxdefinition}

The problem with this notion of composition, though, is that it is not associative: given paths $\gamma, \delta, \eps : [0, 1] \to X$ that are compatible, the compositions $\eps\of{\delta \gamma}$ and $\parenth{\eps \delta} \gamma$ have the same range, but they go at different speeds! So they're not quite the same function. Nevertheless, they are \textbf{homotopic}: a notion we will explore in the next subsection.

\subsection{Homotopies between Functions}

For the remainder of this chapter, we use the following notation for a `rather nice' topological space with which we are already quite familiar.

\begin{boxnotation}
    Denote by $I$ the unit interval $[0, 1]$ endowed with the Euclidean topology.
\end{boxnotation}

for the remainder of this subsection, denote by $X$ and $Y$ two topological spaces. We now define a homotopy of two functions between topological space.

\begin{boxdefinition}[Homotopy between Functions]
    Let $f, g : X \to Y$ be continuous. A \textbf{homotopy from $f$ to $g$} is a continuous function $F : X \times I \to Y$ such that $F(x, 0) = f(x)$ and $F(x, 1) = g(x)$ for all $x \in X$.
\end{boxdefinition}

The picture we want to have in mind is the following.

\begin{figure}[H]
    \centering
    \begin{tikzpicture}[scale=5]
        \draw[->] (0, 0) -- (0, 1);
        \node[left] at (0, 0.5) {$I$};
        % \draw[->] (0, 0) -- (1, 0);
        % \node[below] at (0.5, 0) {$X$};
        \draw[dashed, blue] (0, 0) -- (1, 0);
        \node[above, blue] at (0.5, 0) {$f$};
        \draw[dashed, blue] (0, 1) -- (1, 1);
        \node[below, blue] at (0.5, 1) {$g$};
    \end{tikzpicture}
    \caption{Homotopy of Functions}
\end{figure}

Recall that we previously sketched that a ``path from $x$ to $x'$ in $X$ is some continuous function $\gamma:I\rightarrow X$ satisfying $y(0)=x, y(1)=x'$. But now it's time for another definition:
\begin{boxdefinition}[Path Homotopy]
    Given $\gamma, \delta$ paths from $x$ to $x'$ points in $X$, we say that a ``path homotopy from $\gamma$ to $\delta$'' is a continuous $F:I\times I\rightarrow X$ satisfying $F(s,0)=y(s), F(s,1)=\delta(s)$ for all $s\in X$, and $F(0,t)=x, F(1,t)=x'$ for all $t\in I$. 
\end{boxdefinition}
Note that ``it is annoying'' both indices of the homotopy $F(s,t)$ take value in $I$, but remember what homotopy is.

% \begin{figure}[H]
%     \centering
%     \begin{tikzpicture}[scale=5]
%         \draw[thick,  domain=-3:3] plot ({\x}, {0.3 \x^2 - 1});
%         \draw[thick,  domain=-3:3] plot ({\x}, {-0.3 \x^2 + 1});
%     \end{tikzpicture}
%     \caption{Homotopy of Paths}
% \end{figure}
Recall the apparition last class which presented to us that for $\gamma$ a path from $x$ to $y$, $\delta$ a path from $y$ to $z$, then we could \textit{compose the paths} to get $\gamma\circ \delta(t)=\begin{cases}
    &\gamma(2t)\text{ for }t\in [0,1/2]\\
    &\delta(2t-1)\text{ for }t\in [1/2,1]
\end{cases}$

The following is an easy exercise. (``there will be a lot of these...'')

\begin{boxexercise}
    If $\gamma$ is a path homotopic to $\gamma'$, then for all `compatible' paths $\delta$, $\gamma \star \delta$ is homotopic to $\gamma' \star \delta$.
\end{boxexercise}

The homotopy in question is particularly nice, because the bit of the path corresponding to $\delta$ (which corresponds to ``time indices'' ranging from $0$ to $1$) is held constant.

We also now go back to an example we saw earlier, where we said that $\gamma \star \parenth{\delta \star \eps}$ and $\parenth{\gamma \star \delta} \star \eps$ are not necessarily equal (where $\gamma, \delta, \eps$ are composable paths). However, these paths are, as it turns out, homotopic.

\begin{boxexercise}\label{Ch5:Exo:Path_comp_assoc_up_to_homotopy}
    Let $X$ be a space and let $\gamma, \delta, \eps : [0, 1] \to X$ be composable paths. The paths
    \begin{align*}
        \gamma \star \parenth{\delta \star \eps}
        \quad \text{and} \quad
        \parenth{\gamma \star \delta} \star \eps
    \end{align*}
    are homotopic.
\end{boxexercise}

It is also possible to show that homotopy is an equivalence relation on the space of paths from one point to another.

\begin{boxproposition}\label{Ch5:Prop:Homotopy_equiv_rel}
    Given points $x, y \in X$, the relation $\sim$ on the set of all paths between $x$ and $y$ expressed by path homotopy is an equivalence relation.
\end{boxproposition}
\begin{proof}[Proof sketch]
    We don't give a full proof because ``life is too short''...
    \begin{enumerate}
        \item \underline{Reflexivity.}
        The identity is a homotopy.

        \item \underline{Symmetry.}
        Run the clock backwards.

        \item \underline{Transitivity.}
        You can do some trickery...
    \end{enumerate}
\end{proof}