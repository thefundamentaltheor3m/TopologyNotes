\section{The Fundamental Group}

How sneakily he defined the ``homotopy groupoid.''

\subsection{The Homotopy Groupoid and the Fundamental Group}

Something that \Cref{Ch5:Prop:Homotopy_equiv_rel} allows us to conclude is that the operation $\star$ of composition of paths is really an operation on the space of equivalence classes of paths, up to homotopy.

Suppose now that we have paths $\gamma, \delta, \epsilon$ with $\gamma$ from $x_0$ to $x_1$ and $\epsilon$ from $x_2$ to $x_3$, and $\delta$ from $x_1$ to $x_2$, it is also ``easy'' to show (``I keep saying that 'this is easy', but that's because it is") that $\gamma \star (\delta \star \epsilon)$ is homotopic to $(\gamma \star \delta)\star \epsilon$.

\begin{boxdefinition}[The Fundamental Group]
    Let $(X, x)$ be a pointed space. We define the \textbf{fundamental group of $X$ based at $x$} to be
    \begin{align*}
        \quotient{\set{\text{Loops in $X$ centred at $x$}}}{\text{homotopy}}
    \end{align*}
    That is, $\pi_1(X, x)$ is the set of path homotopy classes of loops at $x$.
\end{boxdefinition}

We can, in fact, think of $\star$ as a binary operation on the fundamental group.

\begin{boxproposition}
    The fundamental group of a pointed space $(X, x)$ is indeed a group under $\star$.
\end{boxproposition}
\begin{proof}
    The usual litany of properties...
    \begin{enumerate}
        \item \underline{Associativity.}
        \Cref{Ch5:Exo:Path_comp_assoc_up_to_homotopy}

        \item \underline{Identity.} Just pick the ``stupidest possible loop'': the homotoy class of the loop $\id_x : I \to X : t \mapsto x$. This is indeed a two-sided identity of this binary operation (because you're thinking about things up to homotopy).

        \item \underline{Inverses.} Let $\gamma$ be a loop at $x$. Define $\gamma\inv : I \to X : t \mapsto \gamma(1 - t)$. Let $[\gamma]$ be the path homotopy class of $\gamma$. It is not hard to show that $[\gamma] \star \brac{\gamma\inv} = \brac{\id_x}$. Composition the other way works identically.
    \end{enumerate}
\end{proof}

Moreover, we have machinery that allows us to be basepoint-independent in path-connected spaces.

\begin{boxexercise}
    Fix points $x, y \in X$. If there is a path from $x$ to $y$, then $\pi_1(X, x) \cong \pi_1(X, y)$.
\end{boxexercise}

\begin{boxlemma}
    Let $\gamma : I \to X$ be a path from $x$ to $x'$ be a path from $x$ to $x'$ in $X$, and let $f : X \to Y$ be continuous. Then,
    \begin{enumerate}
        \item $f \circ \gamma : I \to Y$ is a path from $f(x)$ to $f\of{x'}$ in $Y$.
        \item If $\gamma$ is path-homotopic to another path $\delta$ from $x$ to $x'$, then $f \circ \gamma$ and $f \circ \delta$ are path-homotopic in $Y$.
    \end{enumerate}
\end{boxlemma}
``Honestly, it's a work of a moment to...just do it! It makes perfect sense if you think about the types.

Now here's the fun really getting started.''

\subsection{Functoriality of $\pi_1$}

In this subsection, we show that $\pi_1$ is, in fact, a functor from the category of pointed topological spaces to the category of groups.

First, we recall that a morphism in the category of pointed topological spaces is a continuous function that preserves basepoints. If we can show that such functions induce group homomorphisms in some `natural' sense, then we can use that as a candidate for how $\pi_1$ should behave on morphisms.

\begin{boxproposition}[Behaviour of $\pi_1$ on Morphisms]
    Let $x, y$ be distinguished points in $X$ and $Y$. Consider a morphism $f : (X, x) \to (Y, y)$ in the category of pointed topological spaces. $f$ induces a group homomorphism
    \begin{align*}
        \pi_1(f) : \pi_1(X, x) \to \pi_1(Y, y) : \brac{\gamma} \mapsto \brac{f \circ \gamma}
    \end{align*}
\end{boxproposition}
\begin{proof}
    \sorry    
\end{proof}

We are now ready to show that $\pi_1$ respects the composition structure of $\Top_*$.

\begin{boxtheorem}[Functoriality of $\pi_1$] \hfill
    \begin{enumerate}
        \item For all pointed topological spaces $(X, x) \in \Top_*$,
        \begin{align*}
            \pi_1\of{\id_{(X, x)}} = \id_{\pi_1(X, x)}
        \end{align*}

        \item For all $(X, x), (Y, y), (Z, z) \in \Top_*$ and morphisms $f : (X, x) \to (Y, y)$ and $g : (Y, y) \to (Z, z)$,
        \begin{align*}
            \pi_1\of{g \circ f} = \pi_1(g) \circ \pi_1(f)
        \end{align*}
    \end{enumerate}
\end{boxtheorem}
\begin{proof}
    This is not very difficult or interesting... \sorry (or not).
\end{proof}

\subsection{The Fundamental Group and Path-Connectedness}

Throughout this subsection, fix a topological space $X$. Recall that $X$ is path-connected iff for all $x, y \in X$, there exists a path from $x$ to $y$.

We can apply the functorial properties of the fundamental group to show the following interesting fact.

\begin{boxproposition}
    If $x$ and $x'$ are points in $X$ and $\gamma$ is a path from $x$ to $x'$, then $\gamma$ induces a group isomorphism from $\pi_1\of{X, x}$ to $\pi_1\of{X, x'}$.
\end{boxproposition}
\begin{proof}[Proof sketch]
    This proof will leave a ``certain amount of work to [the reader].''

    Let $\gamma$ be a path from $x$ to $x'$. Map any $[\delta] \in \pi_1\of{X, x}$ to $\brac{\gamma\inv \star \delta \star \gamma}$. It is possible to verify that this is indeed an isomorphism.
\end{proof}

Note that when talking about the fundamental group, we only care about paths \textit{up to homotopy} (and so in this statement, we aren't really making claims about individual/specific paths).

Another insight we can glean from the above result is that in path-connected spaces, the choice of basepoint is immaterial. We will therefore adopt the following notation.

\begin{boxnotation}
    If $X$ is a path-connected space, then whenever we wish to describe the isomorphism type of its fundamental group, we will simply write $\pi_1(X)$. In particular, we will omit any mention of a basepoint because the choice of basepoint in a path-connected space does not have any bearing on the isomorphism type of the fundamental group.
\end{boxnotation}

\subsection{Application: A Homotopy Theoretic Proof of Brouwer's Fixed Point Theorem}

In this subsection, we give a rather pleasant proof of Brouwer's Fixed Point Theorem using what we know about the fundamental group.

\begin{boxlnotation}
    Throughout this subsection (and only this subsection), denote by
    \begin{enumerate}
        \item $D$ the \textbf{unit disc} $\setst{(x, y) \in \R^2}{x^2 + y^2 \leq 1}$
        \item $S$ the \textbf{unit circle} $\setst{(x, y) \in \R^2}{x^2 + y^2 = 1}$
    \end{enumerate}
\end{boxlnotation}

We begin by computing the fundamental groups of these objects.

\begin{boxexample}[Fundamental Group of the Disc]
    Since $D$ is contractible, its fundamental group is the trivial group.
\end{boxexample}

\begin{boxexample}[Fundamental Group of the Circle]
    $\pi_1\of{S} \cong \Z$. The choice of basepoint is immaterial because $S$ is path-connected.
\end{boxexample}


\begin{boxtheorem}[Brouwer's Fixed Point Theorem]
    For all $f : D \to D$ is continuous, there is some $p \in D$ such that $f(p) = p$.
\end{boxtheorem}
\begin{proof}
    Fix some continuous $f : D \to D$. Suppose that for all $p \in D$, $f(p) \neq p$. Define $g : D \to S$ as follows: for all $p \in D$,
    \begin{enumerate}
        \item Draw a straight line from $f(p)$ to $p$ (which is something we can do unambiguously because $f(p) \neq p$).
        \item Let $g(p)$ be the point where this line meets $S$.
    \end{enumerate}
    The picture we want to have in mind is the following.
    \begin{figure}[H]
        \centering
        \begin{tikzpicture}[scale = 3]
            \draw (0,0) circle (1);
            \labelledpoint{0.3}{0.4}{0}{-0.1}{f(p)};
            \labelledpoint{-1}{0}{-0.6}{-0.6}{g(p)};
            % \node[anchor=above]{$p$} at (-0.2, 0.2); % change to labelledpoint
            % \node[anchor=above]{$f(p)$} at (0.5, 0.2); % change to labelledpoint
        \end{tikzpicture}
        \caption{Caption}
    \end{figure}
    There are two key points we will show (\sorry):
    \begin{enumerate}
        \item For all $p \in S$, $g(p) = p$.
        \item $g$ is continuous.
    \end{enumerate}
    Given these points, the proof becomes quite simple: if $i : S \to D$ denotes the (continuous) inclusion map from $S$ to $D$, then for any $p \in S$, the following diagram commutes in $\Top_*$:
    \begin{cd*}
    	{(S, p)} & {(D, p)} \\
    	& {(S, p)}
    	\arrow["i", from=1-1, to=1-2]
    	\arrow["\id_S"', from=1-1, to=2-2]
    	\arrow["g", from=1-2, to=2-2]
    \end{cd*}
    Since $\pi_1$ is a functor, we can apply it to the above commutative triangle to get the following commutative triangle in $\Grp$:
    \begin{cd*}
    	{\Z} & {0} \\
    	& {\Z}
    	\arrow["{\pi_1(i)}", from=1-1, to=1-2]
    	\arrow["{\id_{\Z}}"', from=1-1, to=2-2]
    	\arrow["{\pi_1(g)}", from=1-2, to=2-2]
    \end{cd*}
    But it is impossible for this diagram to commute, because $\id_{\Z}$ is obviously not the trivial homomorphism. So, we cannot define such a function $g$ at all points, and the only reason we cannot do that is that there is some point that is fixed by $f$.
\end{proof}