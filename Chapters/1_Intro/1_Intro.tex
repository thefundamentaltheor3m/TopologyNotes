\chapter{An Introduction to Topology} \label{Ch1:CH}
\thispagestyle{empty}

We begin by making a few slightly non-standard notational choices.

\begin{boxnotation}
    Given a function $f : X \to Y$ and a subset $A \subseteq X$, we write
    \begin{align*}
        f[A] := \setst{f(x) \in Y}{x \in A}
    \end{align*}
    In similar fashion, given $B \subseteq Y$, we write
    \begin{align*}
        f\inv[B] := \setst{x \in X}{f(x) \in B}
    \end{align*}
\end{boxnotation}

The reason it is important to distinguish between the notations $f(A)$ and $f[A]$ is that there might be a situation in which $A \in X$ \textit{and} $A \subseteq X$. % This is as clear an indication as any that the lecturer is a set theorist.
We will not make any more notational choices at this stage that differ from standard mathematical conventions. As and when we do, we will introduce them.

We now recall basic facts about metric spaces.

\section{Metric Spaces}

Recall the definition of a metric space.

\begin{boxdefinition}[Metric Spaces]
    A \textbf{metric space} is a pair $\parenth{X, d}$ consisting of a set $X$ and a function $d : X \times X \to \R$ such that
    \begin{enumerate}
        \item for all $x, y \in X$, $d(x, y) \geq 0$ for all $x, y \in X$
        \item for all $x, y \in X$, $d(x, y) = 0$ if and only if $x = y$
        \item for all $x, y \in X$, $d(x, y) = d(y, x)$
        \item for all $x, y, z \in X$,
        \begin{align*}
            d\of{x, z} \leq d\of{x, y} + d\of{y, z}
        \end{align*}
    \end{enumerate}
    We call the function $d$ a \textbf{metric on $X$}.
\end{boxdefinition}

We give several familiar examples.

\begin{boxexample}[Some Familiar Metric Spaces] \hfill
    \begin{enumerate}
        \item $\R^n$ under the Euclidean metric
        \begin{align*}
            d\of{x, y} = \sqrt{\sum_{i=1}^{n} \parenth{x_i - y_i}^2}
        \end{align*}
        for all $x = \parenth{x_1, \ldots, x_n}, y = \parenth{y_1, \ldots, y_n} \in \R^n$

        \item Any set $X$ under the equality metric
        \begin{align*}
            d(x, y) =
            \begin{cases}
                0 & \text{ if } x = y \\
                1 & \text{ if } x \neq y 
            \end{cases}
        \end{align*}
        for all $x, y \in X$

        \item Any subset $Y \subseteq X$ of a metric space $\parenth{X, d}$ under the restriction of $d$ to $Y \times Y \subseteq X \times X$.

        \item Given two metric spaces $\parenth{X_1, d_1}$ and $\parenth{X_2, d_2}$, there are numerous viable metrics we can define on $X_1 \times X_2$. One of them would be taking the \textit{maximum} of $d_1$ and $d_2$; another would be the \textit{sum}; a third would be
        \begin{align*}
            d\of{\parenth{x_1, y_1} , \parenth{x_2, y_2}} := \sqrt{d\of{x_1, y_1}^2 + d\of{x_2, y_2}^2}
        \end{align*}
        for all $\parenth{x_1, y_1} \in X_1$ and $\parenth{x_2, y_2} \in Y_2$. We define this third metric space to be the \textbf{product metric}, and it is easily seen that the product of Euclidean spaces (under the Euclidean metric) is indeed a Euclidean space (under the Euclidean metric).

        \item The set $C^0\of{\brac{0, 1}}$ of continuous functions from $[0, 1]$ to $\R$ under the supremum metric
        \begin{align*}
            d(f, g) = \norm{f - g}_{\infty} = \sup_{x \in \brac{0, 1}} \abs{f(x) - g(x)}
        \end{align*}
        for all $f, g \in C^{0}\of{\brac{0, 1}}$. More generally, any compact set works (not just $\brac{0, 1}$).

        \item The set $C^0\of{\brac{0, 1}}$ under the metric
        \begin{align*}
            d(f, g) = \sqrt{\int_{0}^{1} \parenth{f(x) - g(x)}^2 \, \dx}
        \end{align*}
        for all $f, g \in C^0\of{\brac{0, 1}}$, which we know is positive-definite because continuous functions that are zero almost everywhere are zero (and nonnegative functions whose integral is zero are zero almost everywhere).

        \item Consider the set
        \begin{align*}
            l^2\of{\R} = \setst{\parenth{x_n}_{n \in \N}}{x_n \in \R \text{ and } \sum_{n=0}^{\infty} x_n^2 < \infty}
        \end{align*}
        We can define the $l^2$ metric on this set by
        \begin{align*}
            d(x, y) := \sqrt{\sum_{n=0}^{\infty} \parenth{x_i - y_i}^2}
        \end{align*}
        for all $x, y \in l^2\of{\R}$. More than showing that this satisfies the properties of a metric, what is tricky here is showing that this metric is well-defined. But this is doable, and we will end the discussion of this example on that note.
    \end{enumerate}
\end{boxexample}

After this barrage of examples of metric spaces, we are finally ready to move onto more interesting definitions.

\subsection{Continuity of Functions}

We begin by discussing the notion of continuity of functions between metric spaces.

\begin{boxdefinition}[Continuity]\label{Ch1:Def:Continuity_Metric}
    Let $\parenth{X, d}$ and $\parenth{X', d'}$ be metric spaces. We say that a function $f : X \to X'$ is \textbf{continuous at a point $x_0 \in X$} if for all $\eps > 0$, there exists a $\delta > 0$ such that for all $x \in X$, if $d\of{x, x_0} < \delta$, then $d'\of{f\of{x}, f\of{x_0}} < \eps$. We say that $f$ is \textbf{continuous} if $f$ is continuous at every point $x_0 \in X$.
\end{boxdefinition}

We mention two interesting facts that we do not bother to prove.

\begin{boxexercise}[Argument-Wise Continuity of Metrics]
    If $\parenth{X, d}$ is a metric space, for all $a \in X$, the function
    \begin{align*}
        x \mapsto d\of{a, x} : X \to \R
    \end{align*}
    is continuous.
\end{boxexercise}

\begin{boxexercise}[Composition of Continuous Functions]
    A composition of continuous functions is continuous.
\end{boxexercise}

% Lecture 2

\subsection{Sequences, Convergence and Uniqueness of Limits}

% Notational conventions: added to start of chapter.

We recall what it means for a sequence to converge in a metric space.

\begin{boxdefinition}[Convergence of a Sequence in a Metric Space]\label{Ch1:Def:Convergence_Metric}
    Let $\parenth{X, d}$ be a metric space, and let $\parenth{x_n}_{n \in \N} \subseteq X$ be a sequence in $X$. Given some $x \in X$, we say that \textbf{$x_n$ converges to $x$}, denoted $x_n \to x$, if $\forall \eps > 0$, $\exists N \in \N$ such that $\forall n \geq N$, $d\of{x_n, x} < \eps$. We say \textbf{$x$ is the limit of $x_n$ as $n \to \infty$}.
\end{boxdefinition}

We can show that limits in a metric space are unique.

\begin{boxproposition}\label{Ch1:Prop:Uniqueness_of_Limits_Metric}
    Let $\parenth{X, d}$ be a metric space and let $\parenth{x_n}_{n \in \N} \subseteq X$ be a sequence in $X$. There is at most one $x$ such that $x_n  \to x$.
\end{boxproposition}
\begin{proof}
    Suppose, for contradiction, that there exist distinct points $x, x' \in X$ such that $x_n \to x$ and $x_n \to x'$. Pick $\eps := d\of{x, x'} / 2$. For this $\eps$, we know there is some $N \in \N$ such that for $n \geq N$, $d\of{x_n, x} < \eps$. Similarly, we know there is some $N' \in \N$ such that for $n \geq N'$, $d\of{x_n, x'} < \eps$. Pick $M := \max\of{N, N'} + 1$. Then, applying the fact that $d\of{x_M, x} = d\of{x, x_M}$,
    \begin{align*}
        d\of{x, x'} \leq d\of{x, x_M} + d\of{x_M, x'} < \eps + \eps = d\of{x, x'}
    \end{align*}
    which clearly is a contradiction. So, $x$ and $x'$ cannot be distinct.
\end{proof}

\begin{boxwarning}
    In this course, we will encounter spaces where a sequence can have more than one limit. Proceed with caution!
\end{boxwarning}

Finally, we discuss topological properties of subsets of topological spaces.

\subsection{Open and Closed Sets}

We recall the definition of an open ball.

\begin{boxdefinition}[Open Ball]
    Let $\parenth{X, d}$ be a metric space. Fix $a \in X$ and $\eps > 0$. We define the \textbf{open ball of radius $\eps$ centred at $a$} to be
    \begin{align*}
        B_{\eps}\of{a} := \setst{b \in X}{d\of{a, b} < \eps}
    \end{align*}
\end{boxdefinition}

The reason for the term `open' in the above definition is the following definition.

\begin{boxdefinition}[Open Sets]\label{Ch1:Def:Open_Sets_Metric_Spaces}
    Let $\parenth{X, d}$ be a metric space. We say that $U \subseteq X$ is \textbf{open} if for all $a \in U$, there exists some $\eps > 0$ such that $B_{\eps}\of{a} \subseteq U$.
\end{boxdefinition}

It is easy to see that an open ball is indeed an open set.

We can also define a dual notion.

\begin{boxdefinition}[Closed Sets]
    Let $\parenth{X, d}$ be a metric space. We say that $U \subseteq X$ is \textbf{closed} if its complement $X \setminus U$ is open.
\end{boxdefinition}

We recall basic properties of open and closed sets.

\begin{boxproposition}\label{Ch1:Prop:Properties_of_Open_Sets}
    Let $\parenth{X, d}$ be a metric space.
    \begin{enumerate}
        \item Both $\emptyset$ and $X$ are both open and closed.
        \item An arbitrary union of open sets is open.
        \item A finite intersection of open sets is open.
    \end{enumerate}
\end{boxproposition}

We omit the proof, because it is easy and basic.

We are now ready to venture into more general waters.

\section{Introduction to Topological Spaces}

A topological space, broadly speaking, is one in which we wish to be able to discuss the notion of convergence without depending on the notion of distance. The definition of convergence in metric spaces really only relies on the openness of open balls. We can generalise it merely by generalising the definition of open sets.

We begin by defining the notion of a \textbf{topological space}.

\begin{boxdefinition}[Topological Space]
    Let $X$ be a set. A \textbf{topology} on $X$ is a family $\tau$ of subsets of $X$ such that
    \begin{enumerate}
        \item $\emptyset, X \in \tau$
        \item $\tau$ is closed under arbitrary unions
        \item $\tau$ is closed under finite intersections
    \end{enumerate}
    We say a subset of $X$ is \textbf{open} with respect to a topology $\tau$ if it lies in $\tau$ and \textbf{closed} if its complement lies in $\tau$. We call the pair $\parenth{X, \tau}$ a \textbf{topological space}.
\end{boxdefinition}

The terminology for open sets is visibly consistent with the terminology used in metric spaces. In fact, this is exactly what \Cref{Ch1:Prop:Properties_of_Open_Sets} demonstrates: that the open sets of a metric space, defined as in \Cref{Ch1:Def:Open_Sets_Metric_Spaces}, do indeed define a topology on it. Every metric space is thus also a topological space. Note that sets can be both open and closed, such as $\emptyset$ and the entire set, just as with metric spaces. We sometimes refer to such sets as being ``clopen''.

\subsection{Sequences, Convergence and Limits}

The reason why we introduced topological spaces to begin with is because we wanted a more general setting in which to talk about convergence. The question is, what is the best way of talking about convergence in arbitrary topological spaces?

Since every metric space is a topological space, and seeing as we already have a notion of convergence in metric spaces, we would want to define the notion of convergence in a topological space to be a \textit{generalisation}. To that end, we restate the definition of convergence in metric spaces using only the language of open sets.

\begin{boxproposition}\label{Ch1:Prop:Convergence_Metric_Topological}
    Let $\parenth{X, d}$ be a metric space. For every sequence $\parenth{x_n}_{n \in \N} \subseteq X$, the following are equivalent:
    \begin{enumerate}[label = \normalfont \arabic*.]
        \item $x_n$ converges to $x$ as in \Cref{Ch1:Def:Convergence_Metric}
        \item For every open set $U \subseteq X$ such that $x \in U$, there is an $N \in \N$ such that for all $n \geq N$, $x_n \in U$.
    \end{enumerate}
\end{boxproposition}

We do not prove this proposition here, but note that it is not difficult to prove.

Since the second statement in \Cref{Ch1:Prop:Convergence_Metric_Topological} does not actually mention any \textit{metric space} properties of $X$, it is a viable definition for convergence in topological spaces.

\begin{boxdefinition}[Convergence of a Sequence in a Topological Space]\label{Ch1:Def:Convergence_Topological}
    Let $\parenth{X, \tau}$ be a topological space, and let $\parenth{x_n}_{n \in \N} \subseteq X$ be a sequence in $X$. Given some $x \in X$, we say that \textbf{$x_n$ converges to $x$}, denoted $x_n \to x$, if for every open set $U \subseteq X$ such that $x \in U$, there is an $N \in \N$ such that for all $n \geq N$, $x_n \in U$.
\end{boxdefinition}

\Cref{Ch1:Prop:Convergence_Metric_Topological} essentially tells us that if $\parenth{X, d}$ is a metric space, then \Cref{Ch1:Def:Convergence_Metric,Ch1:Def:Convergence_Topological} are equivalent, where we apply \Cref{Ch1:Def:Convergence_Topological} to the topological space structure induced by the metric space structure.

\begin{boxexample}[Convergence in the Indiscrete Topology]\label{Ch1:Eg:Indiscrete_Top_not_Hausdorff}
    Consider a set $X$ along with the indiscrete topology (ie, view it as the topological space $\parenth{X, \set{\emptyset, X}}$). Then, \textit{every sequence converges to every point}.
\end{boxexample}

\Cref{Ch1:Eg:Indiscrete_Top_not_Hausdorff} demonstrates that the uniqueness of limits seen in \Cref{Ch1:Prop:Uniqueness_of_Limits_Metric} does not always hold in topological spaces. See \Cref{Ch1:Def:Hausdorff} for more.

Note that going forward, we often omit the topology when talking about topological spaces, unless it has specific properties that are essential for our purposes.

Recall that limits in a metric space are unique (\Cref{Ch1:Prop:Uniqueness_of_Limits_Metric}). We noted that this is not always true, but we are interested enough in distinguishing spaces where this is true from spaces where this is not true that we have a name for such spaces.

\begin{boxdefinition}[Hausdorff/$T_2$ Spaces]\label{Ch1:Def:Hausdorff}
    Let $\parenth{X, \tau}$ be a topological space. We say $\parenth{X, \tau}$ is \textbf{Hausdorff}, or $T_2$, if for all $a, b \in X$ with $a \neq b$, there exist $U, V \in \tau$ such that $a \in U$, $b \in V$ and $U \cap V = \emptyset$.
\end{boxdefinition}

The Hausdorff/$T_2$ property is an instance of a \textit{separation property}. There are many more such properties, including $T_1$, $T_3$, $T_{3 \frac{1}{2}}$, and $T_4$, and it is highly non-trivial to prove facts like $T_4 \implies T_{3 \frac{1}{2}}$. We will see such things as the course progresses.

Note that \Cref{Ch1:Prop:Uniqueness_of_Limits_Metric} tells us precisely that Metric Spaces are Hausdorff/$T_2$. Indeed, this tells us that Hausdorffitude is a \textit{necessary} condition for a topology on a space to be \textbf{metrisable}---that is, for it to be induced by a metric. In particular, \Cref{Ch1:Eg:Indiscrete_Top_not_Hausdorff} tells us that the indiscrete topology is non-metrisable. There is the more interesting question of what conditions might be \textit{sufficient} for a topology to be metrisable. Metrisability is, indeed, highly non-trivial property, and we will see some metrisation theorems in this course.

\subsection{Continuity}

Just as we generalised the notion of convergence from metric spaces to topological spaces, so too can we generalise the notion of continuity.

\begin{boxdefinition}[Continuity]\label{Ch1:Def:Continuity_Topological}
    Let $X$ and $Y$ be topological spaces, and let $f : X \to Y$ be a function. We say $f$ is \textbf{continuous} if for every $U \subseteq Y$ open in $Y$, the pre-image $f\inv\!\brac{U}$ is open in $X$.
\end{boxdefinition}

It is easy to show that \Cref{Ch1:Def:Continuity_Topological} is equivalent to \Cref{Ch1:Def:Continuity_Metric}. We do not do this here.

There are many examples of continuous functions, some familiar and some slightly pathological.

\begin{boxexample}[Weird Topologies lead to Weird Notions of Continuity]
    Let $X$ and $Y$ be topological spaces.
    \begin{enumerate}
        \item If $X$ and $Y$ both have the \textit{discrete} topology on them, then \textit{any} function $f : X \to Y$ is continuous, because all subsets of $X$ and $Y$ are open.

        \item If $X$ and $Y$ both have the \textit{indiscrete} topology on them, then \sorry
    \end{enumerate}
\end{boxexample}

\subsection{Interiors and Closures}

We briefly discuss how to ``make subsets of a topological space open or closed''. We do this by defining interiors and closures.

\begin{boxdefinition}[The Interior of a Set]
    Let $X$ be a topological space and let $A \subseteq X$. We define the \textbf{interior} of $A$, denoted $\interior{A}$ or $A^{\circ}$, to be the union of all open subsets of $A$.
\end{boxdefinition}

Note that $\interior{A}$ is always open. Moreover, it is the largest (with respect to inclusion) open subset of $A$. Finally, note that the sets that are equal to their interiors are precisely the open sets.

We have a `dual' notion of interiors that capture closure properties.

% Closure is the intersection of all closed subsets.

\begin{boxdefinition}[Closure]
    Let $X$ be a topological space and let $A \subseteq X$. We define the \textbf{closure} of $A$, denoted $\closure{A}$ or $\bar{A}$, to be the intersection of all closed subsets of $X$ containing $A$.
\end{boxdefinition}

Dually to the interior, $\closure{A}$ is always closed. Moreover, it is the smallest (with respect to inclusion) closed subset of $X$ containing $A$. Finally, note that the sets that are equal to their closures are precisely the closed sets.

\subsection{Bases and Sub-Bases}

It is interesting to talk about whether a topology can be \textit{generated} by a family of subsets by taking unions. In the case of metric spaces, it is not difficult to show that this is true.

\begin{boxproposition}[Metric Space Topologies are Generated by Open Balls]\label{Ch1:Prop:Met_Top_gen_by_open_balls}
    Let $\parenth{X, d}$ be a metric space. Any $U \subseteq X$ is open if and only if $U$ is an open of open balls.
\end{boxproposition}
\begin{proof}
    \Cref{Ch1:Prop:Properties_of_Open_Sets} tells us that if $U$ is a union of open sets, then it is open. Conversely, for all $x \in U$, there is some $\eps_x$ such that $B_{\eps_x}(x) \subseteq U$. Then, it is easily seen that
    \begin{align*}
        U = \bigcup_{x \in U} B_{\eps_x}(x)
    \end{align*}
    showing that $U$ is a union of open sets.
\end{proof}

We have a special term for families of open sets that `generate' a topology.

\begin{boxdefinition}[Basis of a Topology]
    Let $\parenth{X, \tau}$ be a topological space. A \textbf{basis for $\tau$} is a subset $B \subseteq \tau$ such that every $U \in \tau$ is a union of sets in $B$.
\end{boxdefinition}

\Cref{Ch1:Prop:Met_Top_gen_by_open_balls} tells us precisely that the open balls in a metric space form a \textit{basis} for the topology induced by the metric.

We have a nice criterion to check if a subset of a topology is a basis.

\begin{boxproposition}
    Let $\parenth{X, \tau}$ be a topological space. Let $B \subseteq \tau$. TFAE:
    \begin{enumerate}
        \item $B$ is a basis for $\tau$.
        \item $X$ is a union of elements of $B$. Moreover, for all $U, V \in B$, $U \cap V$ is a union of elements of $B$.
    \end{enumerate}
\end{boxproposition}

It is immediate that the first statement implies the second. The converse is an exercise in set algebra, which we do not do here.

It is also possible to define bases using `sub-bases'. To that end, we have a preliminary fact.

\begin{boxlemma}
    Let $X$ be a set and let $\Pi$ be a non-empty set of topologies on $X$. Then,
    \begin{align*}
        \bigcap \Pi = \setst{A \subseteq X}{\forall \tau \in \Pi, \, A \in \tau}
    \end{align*}
\end{boxlemma}

We do not prove this here. We merely mention that the proof has the same flavour as such statements in algebra as ``an arbitrary intersection of subgroups/ideals/subfields is a subgroup/ideal/subfield''.

A particular consequence of the above is that we can make the following definition.

\begin{boxdefinition}[Topology Generated by a Set]
    Let $\parenth{X}$ be a set and let $C$ be any collection of subsets of $X$. We can define a topology on $X$ by
    \begin{align*}
        \Pi(C) := \bigcap \setst{\tau \text{ a topology on } X}{C \subseteq \tau}
    \end{align*}
    We call this the \textbf{topology on $X$ generated by $C$}.
\end{boxdefinition}

Note that it is \textit{always} possible to define a topology generated by a subset, since $\Pi(C)$ is always non-empty: it always contains the discrete topology.

We can now define a sub-basis of a topology.

\begin{boxdefinition}[Sub-Basis of a Topology]
    Let $X$ be a set and let $C \subseteq X$. If $\tau$ is the least topology of $X$ that contains $C$, then we say that \textbf{$C$ is a sub-basis of $\tau$}.
\end{boxdefinition}

Indeed, any subset is a sub-basis of \textit{some} topology, namely, the topology it generates.

\subsection{New Topologies from Old Ones}

There are numerous techniques to define new topologies from old ones. We begin with the most obvious definition imaginable.

\begin{boxdefinition}[The Subspace Topology]
    Let $X$ be a topological space. Any subset $Y \subseteq X$ inherits a topology from $X$, known as the \textbf{subspace topology}, the open sets of which are precisely those sets of the form $U \cap Y$ for open sets $U \subseteq X$.
\end{boxdefinition}

With that out of the way, we move onto more interesting examples.

We begin with a topology that is defined using continuity of functions.

\begin{boxdefinition}[Initial Topology]\label{Ch1:Def:Init_Top}
    Let $X$ be a set. Given an index set $\I$, topological spaces $X_i$ and functions $f_i : X \to X_i$ for $i \in \I$, we define the \textbf{initial topology on $X$ with respect to $X_i$ and $f_i$} to be the least topology on $X$ (with respect to inclusion) such that for all $i \in \I$, $f_i$ is continuous.
\end{boxdefinition}

Note that it is always possible to do this, because at worst, we the discrete topology on $X$ renders continuous every function from $X$ to any topological space.

Given the definition of continuity (\Cref{Ch1:Def:Continuity_Topological}), it is possible to be explicit about the initial topology.

\begin{boxlemma}
    Let $X, \I, X_i, f_i$ be as in \Cref{Ch1:Def:Init_Top}. The corresponding initial topology $\tau$ is given by
    \begin{align*}
        \tau = \setst{f_i\inv\!\brac{U} \subseteq X}{i \in \I \text{ and } U \subseteq X_i \text{ is open}}
    \end{align*}
\end{boxlemma}

We give an important example illustrating the notion of an initial topology.

\begin{boxexample}[Products]\label{Ch1:Eg:Prod_Top_as_Init_Top}
    Let $\I$ be an index set and let $X_i$ be topologies. Consider the Cartesian product $X$ of these $X_i$, defined as follows:
    \begin{align*}
        X = \prod_{i \in \I} X_i = \setst{\parenth{x_i}_{i \in \I}}{\forall i \in \I\, x_i \in X_i}
    \end{align*}
    We know (either by the category theoretic definition of a product, of which the Cartesian product is the instance in the category of sets, or by sheer common sense) that there are projections $\pi_i : X \to X_i$ that map any $\parenth{x_j}_{j \in \I}$ to $x_i$, indexed by $i \in \I$. We can compute the initial topologies of these projections to obtain a topology on $X$.
\end{boxexample}

We can give a special name to the topology defined above, which is a key way of constructing new topologies from old ones.

\begin{boxdefinition}[The Product Topology]\label{Ch1:Def:Prod_Top}
    Let $\I, X_i, X, \pi_i$ be as in \Cref{Ch1:Eg:Prod_Top_as_Init_Top}. We call the initial topology of the $\pi_i$ the \textbf{product topology} on $X$.
\end{boxdefinition}

We can give a more explicit description of the product topology.

\begin{boxproposition}
    Let $\I, X_i, X, \pi_i$ be as in \Cref{Ch1:Eg:Prod_Top_as_Init_Top} and \Cref{Ch1:Def:Prod_Top}. Then,
    \begin{enumerate}
        \item A sub-basis of the product topology is given by all sets of the form
        \begin{align*}
            \setst{\parenth{x_j}_{j \in \I}}{x_i \in U}
        \end{align*}
        for $i \in \I$ and $U \subseteq X_i$ open.

        \item A basis of the product topology is given by all sets of the form
        \begin{align*}
            \setst{\parenth{x_j}_{j \in \I}}{x_{i_1} \in U_1, \ldots, x_{i_n} \in U_n}
        \end{align*}
        for $n \in \N$, $i_1, \ldots, i_n \in \I$, and $U_k \subseteq X_{i_k}$ open for $1 \leq k \leq n$.
    \end{enumerate}
\end{boxproposition}

We do not prove this here.

Dually to how we defined a topology on the Cartesian product of spaces as the initial topology of the projections, we can define a topology on any subset of a space as the initial topology of the inclusion.\todo{Write this out either as an example followed by a definition or as just a definition.}

Finally\footnote{Pun intended}, we define the \textit{final} topology.

\begin{boxdefinition}[Final Topology]\label{Ch1:Def:Final_Top}
    Let $X$ be a set. Given an index set $\I$, topological spaces $X_i$ and functions $f_i : X_i \to X$ for $i \in \I$, we define the \textbf{final topology on $X$ with respect to $X_i$ and $f_i$} to be the largest topology on $X$ (with respect to inclusion) such that for all $i \in \I$, $f_i$ is continuous.
\end{boxdefinition}
\begin{remark}
    To state it explicitly, we observe that if $X$ has more open sets (i.e. the topology on $X$ is ``larger''), then it is ``harder'' for a function $f_i:X_i\to X$ to be continuous (as $f_i^{-1}(U)$ must be open in $X_i$ for all open $U\subseteq X$, which forces the topology on $X_i$ to be larger as well).
\end{remark} % Go to shortcuts - you'll find a surprise there

Note that it is always possible to construct a final topology, because at worst, we the indiscrete topology on $X$ renders continuous every function into $X$ from any topological space.

Given the definition of continuity (\Cref{Ch1:Def:Continuity_Topological}), it is possible to be explicit about the initial topology.

\begin{boxlemma}
    Let $X, \I, X_i, f_i$ be as in \Cref{Ch1:Def:Final_Top}. The corresponding initial topology $\tau$ is given by
    \begin{align*}
        \tau = \setst{U \subseteq X}{i \in \I \text{ and } f\inv(U) \subseteq X_i \text{ is open}}
    \end{align*}
\end{boxlemma}

There are many good examples of final topologies, some of which are likely to be familiar to the reader.

\begin{boxexample}[Topologies on Quotients]\label{Ch1:Eg:Quot_Top}
    Let $X$ be a topological space and let $\sim$ be an equivalence relation on $X$. Let $q : X \surj \quotient{X}{\sim}$ be the canonical surjection from $X$ to the quotient set. The final topology of $q$ is a topology on $\quotient{X}{\sim}$.
\end{boxexample}

We give a special name to such a topology on a quotient.

\begin{boxdefinition}[Quotient Topology]
    Let $X$, $\sim$, and $q$ be as in \eqref{Ch1:Eg:Quot_Top}. We call the final topology on $\quotient{X}{\sim}$ with respect to $X$ and $q$ the \textbf{quotient topology}.
\end{boxdefinition}

Note that the above definition really applies to \textit{all} situations where $X$ is a topological space and $Y$ is a set such that there is a surjection $q : X \surj Y$: in this case, $Y$ is essentially the quotient of $X$ by the relation $\sim$ where $x \sim y$ iff $f(x) = f(y)$.

We can express the quotient topology in a more familiar way.

\begin{boxproposition}
    \sorry % State (no need to prove) that the usual def of the quotient topology is equivalent to the final topology of the quotient map
\end{boxproposition}

\subsection{Separation Properties}

Throughout this subsection, let $X$ denote a topological space.

We recall the definition of $T_2$ spaces from \Cref{Ch1:Def:Hausdorff}.
% no no no uncomment it you're fine :D
\Cref{Ch1:Def:Hausdorff} already exists btw - just link to it or smth if you want
\begin{boxdefinition}[$T_2$ (i.e. the Hausdorff condition)]
    We say that a topological space is \textbf{$T_2$}, or \textbf{Hausdorff}, if $\forall x\neq y$ there exist open sets $U, V$ such that $x\in U, y \in V$ that we have $U\cap V=\emptyset$
\end{boxdefinition}

There is also a (weaker) notion of separation called the $T_1$ property.

\begin{boxdefinition}[$T_1$ property]
    We say that $X$ is a \textbf{$T_1$ space} if for all \textit{distinct} $x, y \in X$, there is an open set $U \subseteq X$ with $x \in U$ and $y \notin U$.
\end{boxdefinition} 

There is a notion that is weaker still.

\begin{boxdefinition}[$T_0$ property]
    We say $X$ is a \textbf{$T_0$ space} if for all distinct $x, y \in X$, we have
    \begin{align*}
        \setst{U \subseteq X}{U \text{ is open and } x \in U}
        \neq
        \setst{V \subseteq X}{V \text{ is open and } y \in U}
    \end{align*}
    That is, either there is an open set $U$ with $x \in U$ and $y \notin U$ or there is an open set $V$ with $y \in V$ and $x \notin V$.
\end{boxdefinition}
\begin{remark}
    We emphasise that here, we are not able to choose which of $x$ and $y$ is contained in the open set $U$ witnessing the $T_0$ property.
\end{remark}

Before we give a classic example of a $T_0$ topology, we recall the definition of a partially ordered set (also abbreviated `poset').

\begin{boxdefinition}[Partial Order]
    Let $P$ be a set. We say that a binary relation $\leq$ on $P$ is a \textbf{partial order} if it is reflexive, antisymmetric and transitive. We call the pair $\parenth{P, \leq}$ a \textbf{partially ordered set}, often abbreviated \textbf{poset}.
\end{boxdefinition}

There are many familiar examples of posets in mathematics.

\begin{boxexample}[A Collection of Sets as a Poset]
    Given any collection of sets, the collection can always be ordered by inclusion. So any time you take a family of sets and order them by inclusion, you will wind up with a poset structure.
\end{boxexample}

Given that topologies are sets of sets, in particular, topologies are \textit{also} partially ordered by inclusion. Thus, we will find the theory of posets to be quite useful throughout this course.

There is also a ``converse'' relationship between posets and topology: we can define a topology on any poset by taking advantage of the properties of the partial order.

\begin{boxexample}[A $T_0$ Topology on a Poset]\label{Ch1:Eg:Poset_Downward_Cone}
    Let $\parenth{P, \leq}$ be a poset. We introduce a topology $\tau$ on $P$ as follows: we deem a set $U \subseteq P$ to be open iff for all $p \in U$,
    \begin{align*}
        \setst{q \in P}{q \leq P} \subseteq U
    \end{align*}
    That is, we define the open sets to be precisely those $U$ that contain all downward cones of elements in $U$. We can show that this is, indeed, a topology on $P$.
\end{boxexample}

The topology described in \Cref{Ch1:Eg:Poset_Downward_Cone} is the `natural' topology for \textit{forcing posets} in set theory.

\subsection{Neighbourhoods and Neighbourhood Bases}

Next, we introduce the notion of a neighbourhood.

\begin{boxdefinition}[Nbhd]%sounds wise (thanks) % I've introduced $X$ at the start of the subsection
    We say that a point $x\in X$ has nbhd $A\ssq X$ if there exists some open set $O\ssq X$ such that $x\in O \ssq A$.
\end{boxdefinition}

\begin{boxabbrev} 
    Because nobody has the time to write ``neighbourhood'' more than a few times in their life, Professor Cummings will abbreviate it by ``Nbhd'' in the future. We will not, by default, assume that a neighbourhood (or nbhd) of a point is open.
\end{boxabbrev}

We can talk about neighbourhoods in a collective sense, reminiscent of \verb|Filter.nhds| in Lean...

\begin{boxdefinition}[Nbhd Basis]
    Given $x\in X$, a \textbf{nbhd basis} for $x$ is a set $\nbhds$ of nbhds of $x$ such that for every nbhd $A$ of $x$ there exists some $B \in \nbhds$ such that $B\ssq A$ - equivalently, for every open nbhd $U$ of $x$ there exists $B\in \nbhds$ such that $B\ssq U$.
\end{boxdefinition}%wonderful - and interesting, hm % \nhds and \nbhds will now serve as shortcuts for \mathcal{N}
% \nhds because that's what Im used to from Lean for this thing called Filter.nhds, the neighbourhood filter at a point


\begin{boxexample}[The Closed Ball Nbhd Basis]
    Let $\parenth{X, d}$ be a \underline{metric} space. Fix $x \in X$. Let $\nhds$ be a set of nbhds of the form
    \begin{align*}
        \setst{y \in X}{d\of{x, y} \leq \eps}
    \end{align*}
    for all $\eps > 0$. Then, this set all of closed balls at $x$ is a nbhd basis. % Wow - that was a very very quick lecture!!!!!!!!
\end{boxexample}%indeed (and now we move on from separation, closer to more (i.e. an even greater quantity of) interesting things..) ((I love the (nested) parentheses))
\begin{remark}
    It is a ``trivial fact'' that if we have a neighbourhood basis at every point $x\in X$, then we can recover the topology of $X$. (See this as an exercise.)
\end{remark}

\subsection{Connectedness}

Again, fix a topological space $X$. In this subsection, we discuss different notions of connectedness of a toplogical space.

\begin{boxdefinition}[Disconnectedness]
    We say $X$ is \textbf{disconnected} if there exist disjoint open sets $U, V \subsetneq X$ with $U \cup V = X$. If $U$ and $V$ are disjoint ant cover $X$, we say that they \textbf{disconnect $X$}.
\end{boxdefinition}
\begin{remark}
    One must be very careful when disconnecting sets: when you have a disconnected set $X$, it is not at all obvious that the sets you use to disconnect $X$ (namely, $U$ and $V$) are actually disjoint in the rest of $X$. (Remember this for your General Topology basic exam!)
\end{remark}

We can use this definition to define connectedness in the obvious way.

\begin{boxdefinition}[Connectedness]
    We say $X$ is \textbf{connected} if $X$ is not disconnected, ie, if there do not exist disjoint open sets $U$ and $V$ that disconnect $X$.
\end{boxdefinition}

As one would expect, we define connectedness for subsets to be connectedness with respect to the subspace topology.

It turns out we can use the definition of connectedness to say something about the clopen sets of a connected space.

\begin{boxlemma}
    Let $\parenth{A_i}_{i \in \I}$ be some family of subsets of $X$ indexed by some non-empty set $\I$ such that
    \begin{enumerate}
        \item For all $i \in \I$, $A_i$ is a connected subset of $X$
        \item For all $i, j \in \I$, $A_i \cap A_j \neq 0$
    \end{enumerate}
    That is, $\parenth{A_i}$ is a family of connected, \textit{non-disjoint} sets. Then,
    \begin{align*}
        A := \bigcup_{i \in \I} A_i
    \end{align*}
    is a connected subset of $X$.
\end{boxlemma}
\begin{proof}
    Suppose that $A$ is not connected. Then, there exist open subsets $U, V \subsetneq X$ such that
    \begin{align*}
        A = \parenth{A \cap U} \cup \parenth{A \cap V}
    \end{align*}
    with $\parenth{A \cap U}$ and $\parenth{A \cap V}$ being disjoint, non-empty, and proper subsets of $A$. Then, for any $i \in \I$, we have that
    \begin{align*}
        A_i \subseteq A = \parenth{A \cap U} \cup \parenth{A \cap V}
    \end{align*}
    Since the $A_i$ are all connected, we know that for all $i \in \I$, either $A_i \ssq U$ or $A_i \ssq V$. It turns out that we can do better: either every $A_i$ is contained in $U$ or every $A_i$ is contained in $V$. Indeed, if this were not the case, then there would be $i, j \in \I$ with $A_i \in U$ and $A_j \in V$. Then, we would have $A_i \cap A_j \ssq U \cap V$. Moreover, $A_i \cap A_j \ssq A$, since $A_i \ssq A$ and $A_j \ssq A$. Thus,
    \begin{align*}
        A_i \cap V_j \ssq A \cap \cap{U \cap V} = \parenth{A \cap U} \cap \parenth{A \cap V}
    \end{align*}
    Finally, we assumed, in our setup, that $A_i \cap A_j \neq \emptyset$, meaning that there is an element living in the disjoint set $A \cap U$ and $A \cap V$, which is obviously a contradiction. Hence, either every $A_i$ is contained in $U$ or every $A_i$ is contained in $V$.

    Without loss of generality, say that every $A_i$ is contained in $U$. Then, $A \ssq U$ also, meaning that $A \cap U = A$. Hence, $A \cap V = \emptyset$, which contradicts the assumption that $A$ is disconnected, because disconnecting subsets must be proper.
\end{proof}

 We next use the definition of connectedness to establish a ``canonical'' decomposition of $X$.

 Define the binary relation $\sim$ on $X$ such that $a \sim b\iff \exists A\subseteq X$ with $A$ connected and $a,b \in A$. We then claim that $\sim$ is an equivalence relation, and (having shown this) we let the equivalence classes of $\sim$ partition $X$ into sets $X_i$.
 %probably don't want this as the definition, feel free to delete
 % Isn't this what we'd want as the definition? hmm
 %or as maximal connected components, idk - yes, as the maximal connected sets of the sapce X (anything works?)
 % An equivalence class is always maximal in the set of things related to any representative... but if you mean the maximal connected set containing it then idk. 
 \begin{boxdefinition}[Connected Components]
     The \textbf{connected components} of $X$ are the equivalence classes in $\quotient{X}{\sim}$, with $\sim$ being defined as above.
 \end{boxdefinition}

Let us describe these equivalence classes. Fix $x \in X$. Denote its equivalence class in $\quotient{X}{\sim}$ by $\brac{x}$. Then, it is possible to show that
\begin{align*}
    \brac{x} = \setst{y \in X}{y \sim x} = \bigcup\setst{A \ssq X}{x \in A \text{ and } A \text{ is connected}}
\end{align*}
The $\ssq$ inclusion is clear, because $\brac{x}$ is itself connected, meaning that any $y \in \brac{x}$ clearly lies in some connected subset of $X$ containing $x$. Conversely, any element of any connected subset of $X$ containing $x$ must be connected to $x$, putting it in $\brac{x}$.

\begin{boxlemma}\label{Ch1:Lemma:closure_connected_connected}
    If $A \ssq X$ and $A$ is connected, then $\closure{A}$ is also connected.
\end{boxlemma}
\begin{proof} % This is actually not really a true proof by contradiction. In fact, even in the most constructive of type theories, I am quite sure that the definition you'd want to implement for connected is "not disconnected". The definition of "not P" is always "P implies False", which you can prove (constructively) by assuming P and deducing False (ie, "proof by contradiction"). A true proof by contradiction uses LEM in the way that it's set up, which this does not.
    Suppose that $\closure{A}$ is not connected. then, there are open sets $u, V \ssq X$ such that $\closure{A} \ssq U \cup V$, $\closure{A} \cap \closure{B} = \emptyset$, and $\closure{A} \cap U, \closure{A} \cap V \neq \emptyset$.  We can see, since $A \ssq \closure{A}$, that $A = \parenth{A \cap U} \cup \parenth{A \cap V}$, with $A \cap U \cap V = \emptyset$. Since $A$ is connected, we need either $A \cap U = \emptyset$ or $A \cap V = \emptyset$. Assume it is the former. \sorry % Simply show containment in complement
\end{proof}

\begin{boxcorollary}
    The connected components of $X$ are closed subsets of $X$.
\end{boxcorollary}
\begin{proof}
    Fix $x \in X$. Then, by \Cref{Ch1:Lemma:closure_connected_connected}, the closure $\closure{\brac{x}}$ of the connected component $\brac{x}$ containing $X$ is a connected subset of $X$. Since all such connected sets are contained in $\brac{x}$, we have that $\closure{\brac{x}} \ssq \brac{x}$, meaning $\brac{x}$ is equal to its closure, making it closed.
\end{proof}

Surprisingly---or unsurprisingly---connectedness is not preserved by taking subsets.

\begin{boxcexample}[A Disconnected Subset of a Connected Space]
    We know that $\R$ is connected under the Euclidean topology. Consider $\Q \subset \R$. We can clearly see that $\parenth{-\infty, \sqrt{2}} \cap \Q$ and $\parenth{\sqrt{2}, \infty} \cap \Q$ are disjoint, open, proper subsets of $\Q$ that disconnect it.
\end{boxcexample}

However, it turns out the \textit{image} of a connected set in a continuous function \textit{is} connected.

\begin{boxlemma}
    Let $X$ and $Y$ be topological spaces. Let $A \ssq X$ be connected and let $f : X \to Y$ be continuous. Then, $f[A]$ is connected.
\end{boxlemma}
\begin{proof}
    If not, let $U, V \ssq Y$ be open with $f[A]$ \sorry
\end{proof}

% TODO: Split this gigantic section into smaller subsections.