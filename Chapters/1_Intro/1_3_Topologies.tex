\section{A Closer Look at Topologies}

\subsection{Bases and Sub-Bases}

It is interesting to talk about whether a topology can be \textit{generated} by a family of subsets by taking unions. In the case of metric spaces, it is not difficult to show that this is true.

\begin{boxproposition}[Metric Space Topologies are Generated by Open Balls]\label{Ch1:Prop:Met_Top_gen_by_open_balls}
    Let $\parenth{X, d}$ be a metric space. Any $U \subseteq X$ is open if and only if $U$ is an open of open balls.
\end{boxproposition}
\begin{proof}
    \Cref{Ch1:Prop:Properties_of_Open_Sets} tells us that if $U$ is a union of open sets, then it is open. Conversely, for all $x \in U$, there is some $\eps_x$ such that $B_{\eps_x}(x) \subseteq U$. Then, it is easily seen that
    \begin{align*}
        U = \bigcup_{x \in U} B_{\eps_x}(x)
    \end{align*}
    showing that $U$ is a union of open sets.
\end{proof}

We have a special term for families of open sets that `generate' a topology.

\begin{boxdefinition}[Basis of a Topology]
    Let $\parenth{X, \tau}$ be a topological space. A \textbf{basis for $\tau$} is a subset $B \subseteq \tau$ such that every $U \in \tau$ is a union of sets in $B$.
\end{boxdefinition}

\Cref{Ch1:Prop:Met_Top_gen_by_open_balls} tells us precisely that the open balls in a metric space form a \textit{basis} for the topology induced by the metric.

We have a nice criterion to check if a subset of a topology is a basis.

\begin{boxproposition}
    Let $\parenth{X, \tau}$ be a topological space. Let $B \subseteq \tau$. TFAE:
    \begin{enumerate}
        \item $B$ is a basis for $\tau$.
        \item $X$ is a union of elements of $B$. Moreover, for all $U, V \in B$, $U \cap V$ is a union of elements of $B$.
    \end{enumerate}
\end{boxproposition}

It is immediate that the first statement implies the second. The converse is an exercise in set algebra, which we do not do here.

It is also possible to define bases using `sub-bases'. To that end, we have a preliminary fact.

\begin{boxlemma}
    Let $X$ be a set and let $\Pi$ be a non-empty set of topologies on $X$. Then,
    \begin{align*}
        \bigcap \Pi = \setst{A \subseteq X}{\forall \tau \in \Pi, \, A \in \tau}
    \end{align*}
\end{boxlemma}

We do not prove this here. We merely mention that the proof has the same flavour as such statements in algebra as ``an arbitrary intersection of subgroups/ideals/subfields is a subgroup/ideal/subfield''.

A particular consequence of the above is that we can make the following definition.

\begin{boxdefinition}[Topology Generated by a Set]
    Let $\parenth{X}$ be a set and let $C$ be any collection of subsets of $X$. We can define a topology on $X$ by
    \begin{align*}
        \Pi(C) := \bigcap \setst{\tau \text{ a topology on } X}{C \subseteq \tau}
    \end{align*}
    We call this the \textbf{topology on $X$ generated by $C$}.
\end{boxdefinition}

Note that it is \textit{always} possible to define a topology generated by a subset, since $\Pi(C)$ is always non-empty: it always contains the discrete topology.

We can now define a sub-basis of a topology.

\begin{boxdefinition}[Sub-Basis of a Topology]
    Let $X$ be a set and let $C \subseteq X$. If $\tau$ is the least topology of $X$ that contains $C$, then we say that \textbf{$C$ is a sub-basis of $\tau$}.
\end{boxdefinition}

Indeed, any subset is a sub-basis of \textit{some} topology, namely, the topology it generates.

\subsection{New Topologies from Old Ones}

There are numerous techniques to define new topologies from old ones. We begin with the most obvious definition imaginable.

\begin{boxdefinition}[The Subspace Topology]
    Let $X$ be a topological space. Any subset $Y \subseteq X$ inherits a topology from $X$, known as the \textbf{subspace topology}, the open sets of which are precisely those sets of the form $U \cap Y$ for open sets $U \subseteq X$.
\end{boxdefinition}

With that out of the way, we move onto more interesting examples.

We begin with a topology that is defined using continuity of functions.

\begin{boxdefinition}[Initial Topology]\label{Ch1:Def:Init_Top}
    Let $X$ be a set. Given an index set $\I$, topological spaces $X_i$ and functions $f_i : X \to X_i$ for $i \in \I$, we define the \textbf{initial topology on $X$ with respect to $X_i$ and $f_i$} to be the least topology on $X$ (with respect to inclusion) such that for all $i \in \I$, $f_i$ is continuous.
\end{boxdefinition}

Note that it is always possible to do this, because at worst, we the discrete topology on $X$ renders continuous every function from $X$ to any topological space.

Given the definition of continuity (\Cref{Ch1:Def:Continuity_Topological}), it is possible to be explicit about the initial topology.

\begin{boxlemma}
    Let $X, \I, X_i, f_i$ be as in \Cref{Ch1:Def:Init_Top}. The corresponding initial topology $\tau$ is given by
    \begin{align*}
        \tau = \setst{f_i\inv\!\brac{U} \subseteq X}{i \in \I \text{ and } U \subseteq X_i \text{ is open}}
    \end{align*}
\end{boxlemma}

We give an important example illustrating the notion of an initial topology.

\begin{boxexample}[Products]\label{Ch1:Eg:Prod_Top_as_Init_Top}
    Let $\I$ be an index set and let $X_i$ be topologies. Consider the Cartesian product $X$ of these $X_i$, defined as follows:
    \begin{align*}
        X = \prod_{i \in \I} X_i = \setst{\parenth{x_i}_{i \in \I}}{\forall i \in \I\, x_i \in X_i}
    \end{align*}
    We know (either by the category theoretic definition of a product, of which the Cartesian product is the instance in the category of sets, or by sheer common sense) that there are projections $\pi_i : X \to X_i$ that map any $\parenth{x_j}_{j \in \I}$ to $x_i$, indexed by $i \in \I$. We can compute the initial topologies of these projections to obtain a topology on $X$.
\end{boxexample}

We can give a special name to the topology defined above, which is a key way of constructing new topologies from old ones.

\begin{boxdefinition}[The Product Topology]\label{Ch1:Def:Prod_Top}
    Let $\I, X_i, X, \pi_i$ be as in \Cref{Ch1:Eg:Prod_Top_as_Init_Top}. We call the initial topology of the $\pi_i$ the \textbf{product topology} on $X$.
\end{boxdefinition}

We can give a more explicit description of the product topology.

\begin{boxproposition}
    Let $\I, X_i, X, \pi_i$ be as in \Cref{Ch1:Eg:Prod_Top_as_Init_Top} and \Cref{Ch1:Def:Prod_Top}. Then,
    \begin{enumerate}
        \item A sub-basis of the product topology is given by all sets of the form
        \begin{align*}
            \setst{\parenth{x_j}_{j \in \I}}{x_i \in U}
        \end{align*}
        for $i \in \I$ and $U \subseteq X_i$ open.

        \item A basis of the product topology is given by all sets of the form
        \begin{align*}
            \setst{\parenth{x_j}_{j \in \I}}{x_{i_1} \in U_1, \ldots, x_{i_n} \in U_n}
        \end{align*}
        for $n \in \N$, $i_1, \ldots, i_n \in \I$, and $U_k \subseteq X_{i_k}$ open for $1 \leq k \leq n$.
    \end{enumerate}
\end{boxproposition}

We do not prove this here.

Dually to how we defined a topology on the Cartesian product of spaces as the initial topology of the projections, we can define a topology on any subset of a space as the initial topology of the inclusion.\todo{Write this out either as an example followed by a definition or as just a definition.}

Finally\footnote{Pun intended}, we define the \textit{final} topology.

\begin{boxdefinition}[Final Topology]\label{Ch1:Def:Final_Top}
    Let $X$ be a set. Given an index set $\I$, topological spaces $X_i$ and functions $f_i : X_i \to X$ for $i \in \I$, we define the \textbf{final topology on $X$ with respect to $X_i$ and $f_i$} to be the largest topology on $X$ (with respect to inclusion) such that for all $i \in \I$, $f_i$ is continuous.
\end{boxdefinition}
\begin{remark}
    To state it explicitly, we observe that if $X$ has more open sets (i.e. the topology on $X$ is ``larger''), then it is ``harder'' for a function $f_i:X_i\to X$ to be continuous (as $f_i^{-1}(U)$ must be open in $X_i$ for all open $U\subseteq X$, which forces the topology on $X_i$ to be larger as well).
\end{remark} % Go to shortcuts - you'll find a surprise there

Note that it is always possible to construct a final topology, because at worst, we the indiscrete topology on $X$ renders continuous every function into $X$ from any topological space.

Given the definition of continuity (\Cref{Ch1:Def:Continuity_Topological}), it is possible to be explicit about the initial topology.

\begin{boxlemma}
    Let $X, \I, X_i, f_i$ be as in \Cref{Ch1:Def:Final_Top}. The corresponding initial topology $\tau$ is given by
    \begin{align*}
        \tau = \setst{U \subseteq X}{i \in \I \text{ and } f\inv(U) \subseteq X_i \text{ is open}}
    \end{align*}
\end{boxlemma}

There are many good examples of final topologies, some of which are likely to be familiar to the reader.

\begin{boxexample}[Topologies on Quotients]\label{Ch1:Eg:Quot_Top}
    Let $X$ be a topological space and let $\sim$ be an equivalence relation on $X$. Let $q : X \surj \quotient{X}{\sim}$ be the canonical surjection from $X$ to the quotient set. The final topology of $q$ is a topology on $\quotient{X}{\sim}$.
\end{boxexample}

We give a special name to such a topology on a quotient.

\begin{boxdefinition}[Quotient Topology]
    Let $X$, $\sim$, and $q$ be as in \eqref{Ch1:Eg:Quot_Top}. We call the final topology on $\quotient{X}{\sim}$ with respect to $X$ and $q$ the \textbf{quotient topology}.
\end{boxdefinition}

Note that the above definition really applies to \textit{all} situations where $X$ is a topological space and $Y$ is a set such that there is a surjection $q : X \surj Y$: in this case, $Y$ is essentially the quotient of $X$ by the relation $\sim$ where $x \sim y$ iff $f(x) = f(y)$.

We can express the quotient topology in a more familiar way.

\begin{boxproposition}
    \sorry % State (no need to prove) that the usual def of the quotient topology is equivalent to the final topology of the quotient map
\end{boxproposition}

\subsection{Separation Properties}

% No need for \\ at the end of paragraphs or anything - I've set up paragraph spacing so that it happens automatically :)

Throughout this subsection, let $X$ denote a topological space.

We recall the definition of $T_2$ spaces from \Cref{Ch1:Def:Hausdorff}.
% no no no uncomment it you're fine :D
\Cref{Ch1:Def:Hausdorff} already exists btw - just link to it or smth if you want
\begin{boxdefinition}[$T_2$ (i.e. the Hausdorff condition)]
    We say that a topological space is \textbf{$T_2$}, or \textbf{Hausdorff}, if $\forall x\neq y$ there exist open sets $U, V$ such that $x\in U, y \in V$ that we have $U\cap V=\emptyset$
\end{boxdefinition}

There is also a (weaker) notion of separation called the $T_1$ property.

\begin{boxdefinition}[$T_1$ property]
    We say that $X$ is a \textbf{$T_1$ space} if for all \textit{distinct} $x, y \in X$, there is an open set $U \subseteq X$ with $x \in U$ and $y \notin U$.
\end{boxdefinition}

We can give an equivalent characterisation of the $T_1$ property.

\begin{boxproposition}
    $X$ is $T_1$ if and only if every singleton in $X$ is closed.
\end{boxproposition}
\begin{proof}
    First, notice that in the trivial cases where $\abs{X} = 0$ or $\abs{X} = 1$, we are done. So, going forward, assume that $X$ contains at least two distinct elements.
    \begin{description}
        \item[$\parenth{\implies}$] 
        Assume that $X$ is $T_1$. If $\abs{X} = 1$, then we are done. Fix $x \in X$ and consider the singleton $\set{x}$. We know that for any $y \in X \setminus \set{x}$, there exists some open $U_y \ssq X$ such that $y \in U_y$ and $x \notin U_y$.\footnote{Define a pairing $\parenth{y, U_y}$ from the definition of $X$ being $T_1$ using the Axiom of Choice.} Then, it is easy to see that
        \begin{align*}
            X \setminus \set{x} = \bigcup_{y \in X \setminus \set{x}} U_y
        \end{align*}
        Indeed, the $\ssq$ inclusion is obvious, since $y \in U_y$ for all $y \in X \setminus \set{x}$, and the $\supseteq$ inclusion is also clear because $x$ does not lie in any of the $U_y$. Since each $U_y$ is open, so is the union of all of them, making $X \setminus \set{x}$ a union of open sets, hence open. Thus, $\set{x}$ is closed.

        \item[$\parenth{\impliedby}$]
        Assume that every singleton $\set{x} \ssq X$ is closed. Then, fix $x, y \in X$ and assume $x \neq y$. Since $\set{x}$ is closed, $X \setminus \set{x}$ is open, and since $x$ and $y$ are distinct, $y \in X \setminus \set{x}$. Thus, $X \setminus \set{x}$ is an open subset of $X$ containing $y$ but not $x$.
    \end{description}
    Therefore, the $T_1$ condition is equivalent to the condition that every singleton is closed.
\end{proof}

There is a notion that is weaker still.

\begin{boxdefinition}[$T_0$ property]
    We say $X$ is a \textbf{$T_0$ space} if for all distinct $x, y \in X$,
    \begin{align*}
        \setst{U \subseteq X}{U \text{ is open and } x \in U}
        \neq
        \setst{V \subseteq X}{V \text{ is open and } y \in U}
    \end{align*}
    That is, either there is an open set $U$ with $x \in U$ and $y \notin U$ or there is an open set $V$ with $y \in V$ and $x \notin V$.
\end{boxdefinition}
\begin{remark}
    We emphasise that here, we are not able to choose which of $x$ and $y$ is contained in the open set $U$ witnessing the $T_0$ property.
\end{remark}

Before we give a classic example of a $T_0$ topology, we recall the definition of a partially ordered set (also abbreviated `poset').

\begin{boxdefinition}[Partial Order]\label{Ch1:Def:Poset}
    Let $P$ be a set. We say that a binary relation $\leq$ on $P$ is a \textbf{partial order} if it is reflexive, antisymmetric and transitive. We call the pair $\parenth{P, \leq}$ a \textbf{partially ordered set}, often abbreviated \textbf{poset}.
\end{boxdefinition}

There are many familiar examples of posets in mathematics.

\begin{boxexample}[A Collection of Sets as a Poset]
    Given any collection of sets, the collection can always be ordered by inclusion. So any time you take a family of sets and order them by inclusion, you will wind up with a poset structure.
\end{boxexample}

Given that topologies are sets of sets, in particular, topologies are \textit{also} partially ordered by inclusion. Thus, we will find the theory of posets to be quite useful throughout this course.

There is also a ``converse'' relationship between posets and topology: we can define a topology on any poset by taking advantage of the properties of the partial order.

\begin{boxexample}[A $T_0$ Topology on a Poset]\label{Ch1:Eg:Poset_Downward_Cone}
    Let $\parenth{P, \leq}$ be a poset. We introduce a topology $\tau$ on $P$ as follows: we deem a set $U \subseteq P$ to be open iff for all $p \in U$,
    \begin{align*}
        \setst{q \in P}{q \leq P} \subseteq U
    \end{align*}
    That is, we define the open sets to be precisely those $U$ that contain all downward cones of elements in $U$. We can show that this is, indeed, a topology on $P$.
\end{boxexample}

The topology described in \Cref{Ch1:Eg:Poset_Downward_Cone} is the `natural' topology for \textit{forcing posets} in set theory.

\section{A Closer Look at Continuity}

\subsection{Neighbourhoods and Neighbourhood Bases}

Next, we introduce the notion of a neighbourhood.

\begin{boxdefinition}[Nbhd]%sounds wise (thanks) % I've introduced $X$ at the start of the subsection
    We say that a point $x\in X$ has nbhd $A\ssq X$ if there exists some open set $O\ssq X$ such that $x\in O \ssq A$.
\end{boxdefinition}

\begin{boxabbrev} 
    Because nobody has the time to write ``neighbourhood'' more than a few times in their life, Professor Cummings will abbreviate it by ``Nbhd'' in the future. We will not, by default, assume that a neighbourhood (or nbhd) of a point is open.
\end{boxabbrev}

We can talk about neighbourhoods in a collective sense, reminiscent of \verb|Filter.nhds| in Lean...

\begin{boxdefinition}[Nbhd Basis]
    Given $x\in X$, a \textbf{nbhd basis} for $x$ is a set $\nbhds$ of nbhds of $x$ such that for every nbhd $A$ of $x$ there exists some $B \in \nbhds$ such that $B\ssq A$ - equivalently, for every open nbhd $U$ of $x$ there exists $B\in \nbhds$ such that $B\ssq U$.
\end{boxdefinition}%wonderful - and interesting, hm % \nhds and \nbhds will now serve as shortcuts for \mathcal{N}
% \nhds because that's what Im used to from Lean for this thing called Filter.nhds, the neighbourhood filter at a point


\begin{boxexample}[The Closed Ball Nbhd Basis]
    Let $\parenth{X, d}$ be a \underline{metric} space. Fix $x \in X$. Let $\nhds$ be a set of nbhds of the form
    \begin{align*}
        \setst{y \in X}{d\of{x, y} \leq \eps}
    \end{align*}
    for all $\eps > 0$. Then, this set all of closed balls at $x$ is a nbhd basis. % Wow - that was a very very quick lecture!!!!!!!!
\end{boxexample}%indeed (and now we move on from separation, closer to more (i.e. an even greater quantity of) interesting things..) ((I love the (nested) parentheses))
\begin{remark}
    It is a ``trivial fact'' that if we have a neighbourhood basis at every point $x\in X$, then we can recover the topology of $X$. (See this as an exercise.)
\end{remark}

Finally, we note that nbhds give us a definition of continuity that more closely resembles the definition to which we are accustomed in metric spaces.

\begin{boxproposition}
    Given topological spaces $X$ and $Y$ and a function $f : X \to Y$, TFAE:
    \begin{enumerate}
        \item $f$ is continuous
        \item for all $x \in X$ and $N$ a nbhd of $f(x)$, there is a nbhd $M$ of $x$ such that $f[M] \ssq N$.
    \end{enumerate}
\end{boxproposition}

In particular, specialising second characterisation of continuity in the above result to a single point gives us a way of talking about the continuity of a function \textit{at a single point}.

\subsection{Homeomorphisms}

Throughout this subsection, fix topological spaces $X$ and $Y$.

We first define the notion of open and closed functions.

\begin{boxdefinition}[Open and Closed Functions]
    For any $f : X \to Y$, we say:
    \begin{itemize}
        \item $f$ is \textbf{open} iff for all open $U\ssq X$ we have $f(U)$ open in $Y$.
        \item $f$ is \textbf{closed} iff for all closed $C\ssq X$ we have $f(C)$ closed in $Y$.
    \end{itemize}
\end{boxdefinition}

Next, we define the notion of an isomorphism in the category of topological spaces.

\begin{boxdefinition}[Homeomorphism]
    We say $f : X \to Y$ is a \textbf{homeomorphism} if $f$ satisfies the following conditions:
    \begin{enumerate}
        \item $f$ is a bijection
        \item For all $U \ssq X$, $U$ is open in $X$ iff $f[U]$ is open in $Y$
    \end{enumerate}
\end{boxdefinition}

We now mention an easy result (that effectively shows that homeomorphisms \underline{are} isomorphisms in the category of topological spaces).

\begin{boxlemma} % I think this is what you meant? (Unless oyu wanted to characterise the example as easy - in whcih case go for it!)
    For any function $f : X \to Y$, the following are equivalent:
    \begin{enumerate}
        \item $f$ is a homeomorphism
        \item There is $g : Y \to X$ such that
        \begin{enumerate}
            \item $g$ is continuous
            \item $f \circ g = \id_Y$
            \item $g \circ f = \id_X$
        \end{enumerate}
    \end{enumerate}
\end{boxlemma}

We do not prove this result here.

We note that the continuity assumption on the inverse is strictly necessary:
\begin{boxwarning}
    A bijective continuous map need not be a homeomorphism! 
\end{boxwarning}
This is because continuity is in some sense ``one sided" (to take care of the other side, maybe we will need to also require our map be open...).
%wow. (yes, please forgive me later) - I have never noticed you being a goose (maybe in time (looking forwrad to it))
% Oh trust me you will :D
% I am perfectly capable of making typos - I really appreciate that you're checking my work :) (I can be a silly billy goose sometimes)

We illustrate this using a simple counterexample.

\begin{boxcexample}[A Continuous Bijection that is NOT a Homeomorphism]
    % Counterexample environment :D
    Let $X$ be a set of cardinality at least $2$. Consider the spaces $\parenth{X, \tau_1}$ and $\parenth{X, \tau_2}$, with $\tau_1$ being the discrete topology and $\tau_2$ being the indiscrete topology. Then, the identity function $\id_X$ is a bijection from $\parenth{X, \tau_1}$ to $\parenth{X, \tau_2}$; moreover, it is continuous because every function out of $\parenth{X, \tau_1}$ is continuous. However, its inverse (which is also the identity) is \textit{not} continuous.
\end{boxcexample}
% We need to move this to a different section - this technically comes *before* connectedness (due to my arrangement of things here (which might be flawed))
% Yeah - look where we are in the repo
