\section{Introduction to Topological Spaces}

A topological space, broadly speaking, is one in which we wish to be able to discuss the notion of convergence without depending on the notion of distance. The definition of convergence in metric spaces really only relies on the openness of open balls. We can generalise it merely by generalising the definition of open sets.

We begin by defining the notion of a \textbf{topological space}.

\begin{boxdefinition}[Topological Space]
    Let $X$ be a set. A \textbf{topology} on $X$ is a family $\tau$ of subsets of $X$ such that
    \begin{enumerate}
        \item $\emptyset, X \in \tau$
        \item $\tau$ is closed under arbitrary unions
        \item $\tau$ is closed under finite intersections
    \end{enumerate}
    We say a subset of $X$ is \textbf{open} with respect to a topology $\tau$ if it lies in $\tau$ and \textbf{closed} if its complement lies in $\tau$. We call the pair $\parenth{X, \tau}$ a \textbf{topological space}.
\end{boxdefinition}

The terminology for open sets is visibly consistent with the terminology used in metric spaces. In fact, this is exactly what \Cref{Ch1:Prop:Properties_of_Open_Sets} demonstrates: that the open sets of a metric space, defined as in \Cref{Ch1:Def:Open_Sets_Metric_Spaces}, do indeed define a topology on it. Every metric space is thus also a topological space. Note that sets can be both open and closed, such as $\emptyset$ and the entire set, just as with metric spaces. We sometimes refer to such sets as being ``clopen''.

\subsection{Sequences, Convergence and Limits}

The reason why we introduced topological spaces to begin with is because we wanted a more general setting in which to talk about convergence. The question is, what is the best way of talking about convergence in arbitrary topological spaces?

Since every metric space is a topological space, and seeing as we already have a notion of convergence in metric spaces, we would want to define the notion of convergence in a topological space to be a \textit{generalisation}. To that end, we restate the definition of convergence in metric spaces using only the language of open sets.

\begin{boxproposition}\label{Ch1:Prop:Convergence_Metric_Topological}
    Let $\parenth{X, d}$ be a metric space. For every sequence $\parenth{x_n}_{n \in \N} \subseteq X$, the following are equivalent:
    \begin{enumerate}[label = \normalfont \arabic*.]
        \item $x_n$ converges to $x$ as in \Cref{Ch1:Def:Convergence_Metric}
        \item For every open set $U \subseteq X$ such that $x \in U$, there is an $N \in \N$ such that for all $n \geq N$, $x_n \in U$.
    \end{enumerate}
\end{boxproposition}

We do not prove this proposition here, but note that it is not difficult to prove.

Since the second statement in \Cref{Ch1:Prop:Convergence_Metric_Topological} does not actually mention any \textit{metric space} properties of $X$, it is a viable definition for convergence in topological spaces.

\begin{boxdefinition}[Convergence of a Sequence in a Topological Space]\label{Ch1:Def:Convergence_Topological}
    Let $\parenth{X, \tau}$ be a topological space, and let $\parenth{x_n}_{n \in \N} \subseteq X$ be a sequence in $X$. Given some $x \in X$, we say that \textbf{$x_n$ converges to $x$}, denoted $x_n \to x$, if for every open set $U \subseteq X$ such that $x \in U$, there is an $N \in \N$ such that for all $n \geq N$, $x_n \in U$.
\end{boxdefinition}

\Cref{Ch1:Prop:Convergence_Metric_Topological} essentially tells us that if $\parenth{X, d}$ is a metric space, then \Cref{Ch1:Def:Convergence_Metric,Ch1:Def:Convergence_Topological} are equivalent, where we apply \Cref{Ch1:Def:Convergence_Topological} to the topological space structure induced by the metric space structure.

\begin{boxexample}[Convergence in the Indiscrete Topology]\label{Ch1:Eg:Indiscrete_Top_not_Hausdorff}
    Consider a set $X$ along with the indiscrete topology (ie, view it as the topological space $\parenth{X, \set{\emptyset, X}}$). Then, \textit{every sequence converges to every point}.
\end{boxexample}

\Cref{Ch1:Eg:Indiscrete_Top_not_Hausdorff} demonstrates that the uniqueness of limits seen in \Cref{Ch1:Prop:Uniqueness_of_Limits_Metric} does not always hold in topological spaces. See \Cref{Ch1:Def:Hausdorff} for more.

Note that going forward, we often omit the topology when talking about topological spaces, unless it has specific properties that are essential for our purposes.

Recall that limits in a metric space are unique (\Cref{Ch1:Prop:Uniqueness_of_Limits_Metric}). We noted that this is not always true, but we are interested enough in distinguishing spaces where this is true from spaces where this is not true that we have a name for such spaces.

\begin{boxdefinition}[Hausdorff/$T_2$ Spaces]\label{Ch1:Def:Hausdorff}
    Let $\parenth{X, \tau}$ be a topological space. We say $\parenth{X, \tau}$ is \textbf{Hausdorff}, or $T_2$, if for all $a, b \in X$ with $a \neq b$, there exist $U, V \in \tau$ such that $a \in U$, $b \in V$ and $U \cap V = \emptyset$.
\end{boxdefinition}

The Hausdorff/$T_2$ property is an instance of a \textit{separation property}. There are many more such properties, including $T_1$, $T_3$, $T_{3 \frac{1}{2}}$, and $T_4$, and it is highly non-trivial to prove facts like $T_4 \implies T_{3 \frac{1}{2}}$. We will see such things as the course progresses.

Note that \Cref{Ch1:Prop:Uniqueness_of_Limits_Metric} tells us precisely that Metric Spaces are Hausdorff/$T_2$. Indeed, this tells us that Hausdorffitude is a \textit{necessary} condition for a topology on a space to be \textbf{metrisable}---that is, for it to be induced by a metric. In particular, \Cref{Ch1:Eg:Indiscrete_Top_not_Hausdorff} tells us that the indiscrete topology is non-metrisable. There is the more interesting question of what conditions might be \textit{sufficient} for a topology to be metrisable. Metrisability is, indeed, highly non-trivial property, and we will see some metrisation theorems in this course.

\subsection{Continuity}

Just as we generalised the notion of convergence from metric spaces to topological spaces, so too can we generalise the notion of continuity.

\begin{boxdefinition}[Continuity]\label{Ch1:Def:Continuity_Topological}
    Let $X$ and $Y$ be topological spaces, and let $f : X \to Y$ be a function. We say $f$ is \textbf{continuous} if for every $U \subseteq Y$ open in $Y$, the pre-image $f\inv\!\brac{U}$ is open in $X$.
\end{boxdefinition}

It is easy to show that \Cref{Ch1:Def:Continuity_Topological} is equivalent to \Cref{Ch1:Def:Continuity_Metric}. We do not do this here.

There are many examples of continuous functions, some familiar and some slightly pathological.

\begin{boxexample}[Weird Topologies lead to Weird Notions of Continuity]
    Let $X$ and $Y$ be topological spaces.
    \begin{enumerate}
        \item If $X$ and $Y$ both have the \textit{discrete} topology on them, then \textit{any} function $f : X \to Y$ is continuous, because all subsets of $X$ and $Y$ are open.

        \item If $X$ and $Y$ both have the \textit{indiscrete} topology on them, then \sorry
    \end{enumerate}
\end{boxexample}

\subsection{Interiors and Closures}

We briefly discuss how to ``make subsets of a topological space open or closed''. We do this by defining interiors and closures.

\begin{boxdefinition}[The Interior of a Set]
    Let $X$ be a topological space and let $A \subseteq X$. We define the \textbf{interior} of $A$, denoted $\interior{A}$ or $A^{\circ}$, to be the union of all open subsets of $A$.
\end{boxdefinition}

Note that $\interior{A}$ is always open. Moreover, it is the largest (with respect to inclusion) open subset of $A$. Finally, note that the sets that are equal to their interiors are precisely the open sets.

We have a `dual' notion of interiors that capture closure properties.

% Closure is the intersection of all closed subsets.

\begin{boxdefinition}[Closure]
    Let $X$ be a topological space and let $A \subseteq X$. We define the \textbf{closure} of $A$, denoted $\closure{A}$ or $\bar{A}$, to be the intersection of all closed subsets of $X$ containing $A$.
\end{boxdefinition}

Dually to the interior, $\closure{A}$ is always closed. Moreover, it is the smallest (with respect to inclusion) closed subset of $X$ containing $A$. Finally, note that the sets that are equal to their closures are precisely the closed sets.