\section{Topological Spaces}

A topological space, broadly speaking, is one in which we wish to be able to discuss the notion of convergence without depending on the notion of distance. The definition of convergence in metric spaces really only relies on the openness of open balls. We can generalise it merely by generalising the definition of open sets.

We begin by defining the notion of a \textbf{topological space}.

\begin{boxdefinition}[Topological Space]
    Let $X$ be a set. A \textbf{topology} on $X$ is a family $\tau$ of subsets of $X$ such that
    \begin{enumerate}
        \item $\emptyset, X \in \tau$
        \item $\tau$ is closed under arbitrary unions
        \item $\tau$ is closed under finite intersections
    \end{enumerate}
    We say a subset of $X$ is \textbf{open} with respect to a topology $\tau$ if it lies in $\tau$. We call the pair $\parenth{X, \tau}$ a \textbf{topological space}.
\end{boxdefinition}

The terminology for open sets is visibly consistent with the terminology used in metric spaces. In fact, this is exactly what \Cref{Ch1:Prop:Properties_of_Open_Sets} demonstrates: that the open sets of a metric space, defined as in \Cref{Ch1:Def:Open_Sets_Metric_Spaces}, do indeed define a topology on it. Every metric space is thus also a topological space.

\subsection{Sequences, Convergence and Limits}

The reason why we introduced topological spaces to begin with is because we wanted a more general setting in which to talk about convergence. The question is, what is the best way of talking about convergence in arbitrary topological spaces?

Since every metric space is a topological space, and seeing as we already have a notion of convergence in metric spaces, we would want to define the notion of convergence in a topological space to be a \textit{generalisation}. To that end, we restate the definition of convergence in metric spaces using only the language of open sets.

\begin{boxproposition}\label{Ch1:Prop:Convergence_Metric_Topological}
    Let $\parenth{X, d}$ be a metric space. For every sequence $\parenth{x_n}_{n \in \N} \subseteq X$, the following are equivalent:
    \begin{enumerate}[label = \normalfont \arabic*.]
        \item $x_n$ converges to $x$ as in \Cref{Ch1:Def:Convergence_Metric}
        \item For every open set $U \subseteq X$ such that $x \in U$, there is an $N \in \N$ such that for all $n \geq N$, $x_n \in U$.
    \end{enumerate}
\end{boxproposition}

We do not prove this proposition here, but note that it is not difficult to prove.

Since the second statement in \Cref{Ch1:Prop:Convergence_Metric_Topological} does not actually mention any \textit{metric space} properties of $X$, it is a viable definition for convergence in topological spaces.

\begin{boxdefinition}[Convergence of a Sequence in a Topological Space]\label{Ch1:Def:Convergence_Topological}
    Let $\parenth{X, \tau}$ be a topological space, and let $\parenth{x_n}_{n \in \N} \subseteq X$ be a sequence in $X$. Given some $x \in X$, we say that \textbf{$x_n$ converges to $x$}, denoted $x_n \to x$, if for every open set $U \subseteq X$ such that $x \in U$, there is an $N \in \N$ such that for all $n \geq N$, $x_n \in U$.
\end{boxdefinition}

\Cref{Ch1:Prop:Convergence_Metric_Topological} essentially tells us that if $\parenth{X, d}$ is a metric space, then \Cref{Ch1:Def:Convergence_Metric,Ch1:Def:Convergence_Topological} are equivalent, where we apply \Cref{Ch1:Def:Convergence_Topological} to the topological space structure induced by the metric space structure.

\begin{boxexample}[Convergence in the Indiscrete Topology]\label{Ch1:Eg:Indiscrete_Top_not_Hausdorff}
    Consider a set $X$ along with the indiscrete topology (ie, view it as the topological space $\parenth{X, \set{\emptyset, X}}$). Then, \textit{every sequence converges to every point}.
\end{boxexample}

\Cref{Ch1:Eg:Indiscrete_Top_not_Hausdorff} demonstrates that the uniqueness of limits seen in \Cref{Ch1:Prop:Uniqueness_of_Limits_Metric} does not always hold in topological spaces. See \Cref{Ch1:Def:Hausdorff} for more.

Note that going forward, we often omit the topology when talking about topological spaces, unless it has specific properties that are essential for our purposes.

Recall that limits in a metric space are unique (\Cref{Ch1:Prop:Uniqueness_of_Limits_Metric}). We noted that this is not always true, but we are interested enough in distinguishing spaces where this is true from spaces where this is not true that we have a name for such spaces.

\begin{boxdefinition}[Hausdorff/$T_2$ Spaces]\label{Ch1:Def:Hausdorff}
    Let $\parenth{X, \tau}$ be a topological space. We say $\parenth{X, \tau}$ is \textbf{Hausdorff}, or $T_2$, if for all $a, b \in X$ with $a \neq b$, there exist $U, V \in \tau$ such that $a \in U$, $b \in V$ and $U \cap V = \emptyset$.
\end{boxdefinition}

The Hausdorff/$T_2$ property is an instance of a \textit{separation property}. There are many more such properties, including $T_1$, $T_3$, $T_{3 \frac{1}{2}}$, and $T_4$, and it is highly non-trivial to prove facts like $T_4 \implies T_{3 \frac{1}{2}}$. We will see such things as the course progresses.

Note that \Cref{Ch1:Prop:Uniqueness_of_Limits_Metric} tells us precisely that Metric Spaces are Hausdorff/$T_2$. Indeed, this tells us that Hausdorffitude is a \textit{necessary} condition for a topology on a space to be \textbf{metrisable}---that is, for it to be induced by a metric. In particular, \Cref{Ch1:Eg:Indiscrete_Top_not_Hausdorff} tells us that the indiscrete topology is non-metrisable. There is the more interesting question of what conditions might be \textit{sufficient} for a topology to be metrisable. Metrisability is, indeed, highly non-trivial property, and we will see some metrisation theorems in this course.

\subsection{Continuity}

Just as we generalised the notion of convergence from metric spaces to topological spaces, so too can we generalise the notion of continuity.

\begin{boxdefinition}[Continuity]\label{Ch1:Def:Continuity_Topological}
    Let $X$ and $Y$ be topological spaces, and let $f : X \to Y$ be a function. We say $f$ is \textbf{continuous} if for every $U \subseteq Y$ open in $Y$, the pre-image $f\inv\!\brac{U}$ is open in $X$.
\end{boxdefinition}

It is easy to show that \Cref{Ch1:Def:Continuity_Topological} is equivalent to \Cref{Ch1:Def:Continuity_Metric}. We do not do this here.

There are many examples of continuous functions, some familiar and some slightly pathological.

\begin{boxexample}[Weird Topologies lead to Weird Notions of Continuity]
    Let $X$ and $Y$ be topological spaces.
    \begin{enumerate}
        \item If $X$ and $Y$ both have the \textit{discrete} topology on them, then \textit{any} function $f : X \to Y$ is continuous, because all subsets of $X$ and $Y$ are open.

        \item If $X$ and $Y$ both have the \textit{indiscrete} topology on them, then \sorry
    \end{enumerate}
\end{boxexample}

\subsection{Interiors and Closures}

We briefly discuss how to ``make subsets of a topological space open or closed''. We do this by defining interiors and closures.

\begin{boxdefinition}[The Interior of a Set]
    Let $X$ be a topological space and let $A \subseteq X$. We define the \textbf{interior} of $A$, denoted $\interior{A}$ or $A^{\circ}$, to be the union of all open subsets of $A$.
\end{boxdefinition}

Note that $\interior{A}$ is always open. Moreover, it is the largest (with respect to inclusion) open subset of $A$. Finally, note that the sets that are equal to their interiors are precisely the open sets.

We have a `dual' notion of interiors that capture closure properties.

% Closure is the intersection of all closed subsets.

\begin{boxdefinition}[Closure]
    Let $X$ be a topological space and let $A \subseteq X$. We define the \textbf{closure} of $A$, denoted $\closure{A}$ or $\bar{A}$, to be the intersection of all closed subsets of $X$ containing $A$.
\end{boxdefinition}

Dually to the interior, $\closure{A}$ is always closed. Moreover, it is the smallest (with respect to inclusion) closed subset of $X$ containing $A$. Finally, note that the sets that are equal to their closures are precisely the closed sets.

\subsection{Bases and Sub-Bases}

It is interesting to talk about whether a topology can be \textit{generated} by a family of subsets by taking unions. In the case of metric spaces, it is not difficult to show that this is true.

\begin{boxproposition}[Metric Space Topologies are Generated by Open Balls]\label{Ch1:Prop:Met_Top_gen_by_open_balls}
    Let $\parenth{X, d}$ be a metric space. Any $U \subseteq X$ is open if and only if $U$ is an open of open balls.
\end{boxproposition}
\begin{proof}
    \Cref{Ch1:Prop:Properties_of_Open_Sets} tells us that if $U$ is a union of open sets, then it is open. Conversely, for all $x \in U$, there is some $\eps_x$ such that $B_{\eps_x}(x) \subseteq U$. Then, it is easily seen that
    \begin{align*}
        U = \bigcup_{x \in U} B_{\eps_x}(x)
    \end{align*}
    showing that $U$ is a union of open sets.
\end{proof}

We have a special term for families of open sets that `generate' a topology.

\begin{boxdefinition}[Basis of a Topology]
    Let $\parenth{X, \tau}$ be a topological space. A \textbf{basis for $\tau$} is a subset $B \subseteq \tau$ such that every $U \in \tau$ is a union of sets in $B$.
\end{boxdefinition}

\Cref{Ch1:Prop:Met_Top_gen_by_open_balls} tells us precisely that the open balls in a metric space form a \textit{basis} for the topology induced by the metric.

We have a nice criterion to check if a subset of a topology is a basis.

\begin{boxproposition}
    Let $\parenth{X, \tau}$ be a topological space. Let $B \subseteq \tau$. TFAE:
    \begin{enumerate}
        \item $B$ is a basis for $\tau$.
        \item $X$ is a union of elements of $B$. Moreover, for all $U, V \in B$, $U \cap V$ is a union of elements of $B$.
    \end{enumerate}
\end{boxproposition}

It is immediate that the first statement implies the second. The converse is an exercise in set algebra, which we do not do here.

It is also possible to define bases using `sub-bases'. To that end, we have a preliminary fact.

\begin{boxlemma}
    Let $X$ be a set and let $\Pi$ be a non-empty set of topologies on $X$. Then,
    \begin{align*}
        \bigcap \Pi = \setst{A \subseteq X}{\forall \tau \in \Pi, \, A \in \tau}
    \end{align*}
\end{boxlemma}

We do not prove this here. We merely mention that the proof has the same flavour as such statements in algebra as ``an arbitrary intersection of subgroups/ideals/subfields is a subgroup/ideal/subfield''.

A particular consequence of the above is that we can make the following definition.

\begin{boxdefinition}[Topology Generated by a Set]
    Let $\parenth{X}$ be a set and let $C$ be any collection of subsets of $X$. We can define a topology on $X$ by
    \begin{align*}
        \Pi(C) := \bigcap \setst{\tau \text{ a topology on } X}{C \subseteq \tau}
    \end{align*}
    We call this the \textbf{topology on $X$ generated by $C$}.
\end{boxdefinition}

Note that it is \textit{always} possible to define a topology generated by a subset, since $\Pi(C)$ is always non-empty: it always contains the discrete topology.

We can now define a sub-basis of a topology.

\begin{boxdefinition}[Sub-Basis of a Topology]
    Let $X$ be a set and let $C \subseteq X$. If $\tau$ is the least topology of $X$ that contains $C$, then we say that \textbf{$C$ is a sub-basis of $\tau$}.
\end{boxdefinition}

Indeed, any subset is a sub-basis of \textit{some} topology, namely, the topology it generates.

\subsection{New Topologies from Old Ones}

There are numerous techniques to define new topologies from old ones. We begin with a topology that is defined using continuity of functions.

\begin{boxdefinition}[Initial Topology]\label{Ch1:Def:Init_Top}
    Let $X$ be a set. Given an index set $\I$, topological spaces $X_i$ and functions $f_i : X \to X_i$ for $i \in \I$, we define the \textbf{initial topology on $X$ with respect to $X_i$ and $f_i$} to be the least topology on $X$ (with respect to inclusion) such that for all $i \in \I$, $f_i$ is continuous.
\end{boxdefinition}

Note that it is always possible to do this, because at worst, we the discrete topology on $X$ renders continuous every function from $X$ to any topological space.

Given the definition of continuity (\Cref{Ch1:Def:Continuity_Topological}), it is possible to be explicit about the initial topology.

\begin{boxlemma}
    Let $X, \I, X_i, f_i$ be as in \Cref{Ch1:Def:Init_Top}. The corresponding initial topology $\tau$ is given by
    \begin{align*}
        \tau = \setst{f_i\inv\!\brac{U} \subseteq X}{i \in \I \text{ and } U \subseteq X_i \text{ is open}}
    \end{align*}
\end{boxlemma}

We give an important example illustrating the notion of an initial topology.

\begin{boxexample}[Products]\label{Ch1:Eg:Prod_Top_as_Init_Top}
    Let $\I$ be an index set and let $X_i$ be topologies. Consider the Cartesian product $X$ of these $X_i$, defined as follows:
    \begin{align*}
        X = \prod_{i \in \I} X_i = \setst{\parenth{x_i}_{i \in \I}}{\forall i \in \I\, x_i \in X_i}
    \end{align*}
    We know (either by the category theoretic definition of a product, of which the Cartesian product is the instance in the category of sets, or by sheer common sense) that there are projections $\pi_i : X \to X_i$ that map any $\parenth{x_j}_{j \in \I}$ to $x_i$, indexed by $i \in \I$. We can compute the initial topologies of these projections to obtain a topology on $X$.
\end{boxexample}

We can give a special name to the topology defined above, which is a key way of constructing new topologies from old ones.

\begin{boxdefinition}[The Product Topology]\label{Ch1:Def:Prod_Top}
    Let $\I, X_i, X, \pi_i$ be as in \Cref{Ch1:Eg:Prod_Top_as_Init_Top}. We call the initial topology of the $\pi_i$ the \textbf{product topology} on $X$.
\end{boxdefinition}

We can give a more explicit description of the product topology.

\begin{boxproposition}
    Let $\I, X_i, X, \pi_i$ be as in \Cref{Ch1:Eg:Prod_Top_as_Init_Top} and \Cref{Ch1:Def:Prod_Top}. Then,
    \begin{enumerate}
        \item A sub-basis of the product topology is given by all sets of the form
        \begin{align*}
            \setst{\parenth{x_j}_{j \in \I}}{x_i \in U}
        \end{align*}
        for $i \in \I$ and $U \subseteq X_i$ open.

        \item A basis of the product topology is given by all sets of the form
        \begin{align*}
            \setst{\parenth{x_j}_{j \in \I}}{x_{i_1} \in U_1, \ldots, x_{i_n} \in U_n}
        \end{align*}
        for $n \in \N$, $i_1, \ldots, i_n \in \I$, and $U_k \subseteq X_{i_k}$ open for $1 \leq k \leq n$.
    \end{enumerate}
\end{boxproposition}

We do not prove this here.

Dually to how we defined a topology on the Cartesian product of spaces as the initial topology of the projections, we can define a topology on any subset of a space as the initial topology of the inclusion.\todo{Write this out either as an example followed by a definition or as just a definition.}

Finally\footnote{Pun intended}, we define the \textit{final} topology.

% Given spaces X_i and functions X_i \to X, we define the final topology to be the largest topology on X such that all the f_i are continuous. That is, it consists of all U whose pre-images in the f_i are open. Write this out in more detail!!!!