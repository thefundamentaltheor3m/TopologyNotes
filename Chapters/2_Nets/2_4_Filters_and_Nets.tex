\section{Relating Filters to Nets}

We begin with an ``irritating and bafflingly slick proof'' that essentially boils down to a clever choice of index set. Let $X$ be a topological space.

\begin{boxtheorem}
    Let $\parenth{x_a}_{a \in \D}$ be a net in $X$. There exists an ultranet in $X$ that is a subnet of $\parenth{x_a}_{a \in \D}$.
\end{boxtheorem}
\begin{proof}
    Our argument will essentially be an application of \Cref{Ch2:Cor:Filter_extn_ultrafilter}.

    Define the following collection of subsets of $\D$:
    \begin{align*}
        F := \setst{A \ssq \D}{\exists a \in \D \st \setst{b \in \D}{a \leq b} \ssq A}
    \end{align*}
    That is, we define $F$ to be the set of all subsets of $\D$ containing some nonempty cone. We claim that $F$ is a filter on $\D$. According to Professor Cummings, the only ``tinily nontrivial'' thing to show there is showing that $F$ is closed under intersections. The point is that by directedness, the intersection of two cones contains a another cone.

    Apply \Cref{Ch2:Cor:Filter_extn_ultrafilter} to obtain an ultrafilter $\U \supseteq F$ on $\D$. We are now ready to define the index set and maps giving a subnet of $\parenth{x_a}_{a \in \D}$. Define
    \begin{align*}
        E := \setst{\parenth{a, A}}{a \in \D,\, a \in A,\, A \in \U}
    \end{align*}
    Define the following ordering on $\E$:
    \begin{align*}
        \parenth{a_1, A_1} \leq_{\E} \parenth{a_2, A_2}
        \iff
        a_1 \leq_{\D} a_2
        \text{ and }
        A_2 \subseteq A_1
    \end{align*}
    It is not difficult to show that $\E$ is a poset (Professor Cummings would give ``no points'' for that). ``The point really is to show it is directed.'' To that end, fix $\parenth{a_1, A_1}, \parenth{a_2, A_2} \in \E$. Since $\D$ is directed, we can find $b \in \D$ such that $b \geq_{\D} a_1, a_2$. Now define $B = A_1 \cap A_2 \cap \setst{b \in \D}{b \geq_{\D} a}$. It is not difficult to see that $\parenth{a_1, A_1}, \parenth{a_2, A_2} \leq_{\E} \parenth{b, B}$.

    We now find an order-preserving and cofinal map from $\E$ to $\D$. Define $\phi : \E \to \D$ by $\phi\of{\parenth{a, A}} = a$. It is immediate that $\phi$ is order-preserving. Moreover, since $\phi\of{\parenth{a, \D}} = a$, we can see that $\phi$ is cofinal.

    Thus, for all $\parenth{a, A} \in \E$, if we define
    \begin{align*}
        y_{\parenth{a, E}} = x_{\phi\of{\parenth{a, A}}} = x_a
    \end{align*}
    then $\parenth{y_{e}}_{e \in \E}$ is a subnet of $\parenth{x_a}_{a \in \D}$.

    All that remains is to show that $\parenth{y_e}_{e \in \E}$ is an ultranet.
    
    One must remember what is the decisive property of ultranets - we either want to find a point where all the indices above that point hit $Y\ssq X$, or all indices above this point hit $X\setminus Y$.

    Let $A=\{a\in \mathbb{D}:x_a\in Y\}$, as $\mathcal{U}$ is an ultrafilter, either $A\in \mathcal{U}$ or $\mathbb{D}\setminus A\in \mathcal{U}$. Now supposing that $A\in \mathcal{U}$, we can choose $a\in A$ and let $e=(a,A)$. Then for all $f\geq e$, $f=(b, B)$ where $a\leq b, B\ssq A$ we have $b\in B\ssq A\implies x_b\in Y \implies y_d=y_{(b,B)}=x_b$.

    Hopefully you can see that ``rest, similar.''
    \sorry
\end{proof}

And now we've reached a bit of a high-point in the course - we have proved Tychnoff's Theorem!\\

You knew we were not done with the definitions - in fact we have only just begun.