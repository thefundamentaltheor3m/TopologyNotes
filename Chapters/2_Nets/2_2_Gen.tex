\section{Generalising Properties of Metric Spaces}

To set the stage, recall that in a metric space $X$ with $A \ssq X$, we have
\begin{align*}
    \cl(A) = \setst{x \in X}{\exists \parenth{a_n}_{n \in \N} \st a_n \in A \text{ and } a_n \to x \text{ as } n \to \infty}
\end{align*}
Moreover, recall that a metric space $X$ is compact if and only if $X$ is sequentially compact, in the sense that every sequence has a convergent subsequence.

We will show that in an arbitrary topological space, we can replace sequences with nets and subnets and show that they still hold.

\subsection{Compactness and Sequantial Compactness}

While we know that compactness and sequential compactness are not equal in general topological spaces, we show that they \textit{are} if we use nets instead.

\begin{boxtheorem}
    Let $X$ be a topological space. The following are equivalent.
    \begin{enumerate}[label = (\arabic*)]
        \item $X$ is compact.
        \item Ever net has a convergent subnet.
    \end{enumerate}
\end{boxtheorem}
\begin{proof} \hfill
    \begin{description}
        \item[\underline{$(2) \implies (1)$.}]
        Suppose that every net has a convergent subnet. Define an open cover $\setst{U_i}{i \in I}$ and define \sorry % Fill in deets of directed poset and desired sequence

        Let $\parenth{x_a}_{a \in \D}$ be such that
        \begin{align*}
            x_a \in \bigcap_{i \in a} F_i
        \end{align*}
        We know a convergent subnet exists. Let $\parenth{y_b}_{b \in \E}$ be such a subnet, with associated map $\phi : \E \to \D$, and say that $\parenth{y_b}_{b \in \E}$ converges to $y \in X$. We claim that
        \begin{align*}
            y \in \bigcap_{i \in I} F_i
        \end{align*}
        Indeed, if this is not true, then there is some $i \in I$ such that $y \notin F_i$. That is, $y \in X \setminus F_i$, which is open because $F_i$ is closed. Moreover, we know that $y_b \to y$. By the definition of convergence, there is some $c \in \E$ such that for all $c' \in \E$, if $c' \geq c$ then $y_{c'} \in X \setminus F_i$.

        Our setup essentially tells us that for all $i \in I$ and $a \in \D$,
        \begin{align}
            a \geq \set{i} \iff i \in a \iff x_a \in F_i
            \label{Ch2:Eq:Subnets_seq_compact_setup_mem_order}
        \end{align}
        Since $\phi$ is cofinal, there is some index $\bar{c} \in \E$ such that $\phi\of{\bar{c}} \geq \set{i}$ (that is, $i \in \phi\of{\bar{c}}$). Let $e \in \E$ be such that $e \geq c, \bar{c}$ (such an $e$ exists by directedness). Then,
        \begin{align*}
            x_{\phi(e)} = y_e \in X \setminus F_i
        \end{align*}
        Indeed, $\phi(e) \geq \phi\of{\bar{c}} \geq \set{i}$, with the ordering being inclusion. Therefore, $i \in \phi(e)$. But \eqref{Ch2:Eq:Subnets_seq_compact_setup_mem_order} tells us that this is equivalent to $x_{\phi(e)} \in F_i$, which is a contradiction.

        \underline{$(1) \implies (2)$.}
        Let $X$ be compact. Let $\parenth{x_a}_{a \in \D}$ be a net. For each $a \in \D$, define
        \begin{align*}
            F_a := \cl\of{\setst{x_b}{b \geq a}}
        \end{align*}
        We claim that this family $\setst{F_a}{a \in \D}$ of closed sets has the finite intersection property.

        Indeed, fix $a_1, \ldots, a_n \in \D$. Find, by directedness, some $a \in A$ such that
        \begin{align*}
            a \geq a_1, \ldots, a_n
        \end{align*}
        Then, $x_a \in F_{a_i}$ for all $1 \leq i \leq n$, just by unfolding definitions. Thus,
        \begin{align*}
            x_a \in \bigcap_{i=1}^{n} F_{a_i}
        \end{align*}
        Since $X$ is compact, we can produce, by ``dubious magic'', some
        \begin{align*}
            x \in \bigcap_{a \in \D} F_a
        \end{align*}
        We construct a subnet of $\parenth{x_a}_{a \in \D}$ which converges to $x$.

        Define the following set:
        \begin{align*}
            \E = \setst{\parenth{a, U}}{a \in \D\text{, } U \text{ is an open nbhd of $x$, and } x_a \in U}
        \end{align*}
        We order $\E$ by the ordering $\parenth{a, U} \leq \parenth{b, V}$ iff $a \leq_{\D} b$ and $V \ssq U$. We show that under this ordering, $\E$ is directed. That is, we show that for all $\parenth{a_1, U_1}, \parenth{a_2, U_2} \in \E$, there exists some $\parenth{c, V}$ such that $\parenth{a_1, U_1}, \parenth{a_2, U_2} \leq_{\E} \parenth{c, V}$. Indeed, for all $\parenth{a_1, U_1}, \parenth{a_2, U_2} \in \E$, we can find some $b \in \D$ such that $b \geq a_1, a_2$, because $\D$ is directed. Moreover, $U_1 \cap U_2$ is an open neighbourhood of $x$. Indeed, if we take
        \begin{align*}
            F_b := \cl\of{\setst{x_c \in X}{c \geq b}}
        \end{align*}
        then we have $x \in F_b$, so the triple intersection
        \begin{align*}
            \setst{x_c \in c}{c \geq b} \cap U_1 \cap U_2
        \end{align*}
        is nonempty. Find $c \geq b$ such that $x_c \in U_1 \cap U)2$. Then, we have $\parenth{c, U_1 \cap U_2} \in \E$. Moreover, $c \geq b \geq a_1, a_2$ and $U_1 \cap U_2 \subseteq U_1, U_2$. Thus, $\parenth{a_1, U_1}, \parenth{a_2, U_2} \leq_{\E} \parenth{c, U_1 \cap U_2}$.

        To define a subnet, we need first a map $\phi : \E \to \D$ satisfying the desired conditions. Define
        \begin{align*}
            \phi : \E \to \D : \parenth{a, U} \mapsto a
        \end{align*}
        Define the net $\parenth{y_{\parenth{a, U}}}_{\parenth{a, U} \in \E}$ by $y_{\parenth{a, U}} = x_a$. Then,
        \begin{enumerate}
            \item $\phi$ is order-preserving because for all $\parenth{a, U}, \parenth{b, V} \in \D$, if $\parenth{a, U} \leq \parenth{b, V}$ then $a \leq b$.
            \item $\phi$ is cofinal because for all $a \in \D$, we can see that $\parenth{a, X} \in \E$ has the property that $a \leq \phi\of{\parenth{a, X}} = a$.
            \item For all $\parenth{a, U} \in \E$, clearly $y_{\parenth{a, U}} = x_a$ by definition.
        \end{enumerate}
        Lastly, we show that $\parenth{y_{\parenth{a, U}}}_{\parenth{a, U} \in \E}$ converges to $x$.
        \sorry
    \end{description}
\end{proof}
%``going back to my habitual heavy-handed hints''?
%``seeing as this is the afternoon of the heavy-handed hint''
%when defining a subnet, make sure the index set is directed. then need to define values in subnet, and specify the values of the function \phi, and then make sure that \phi is a vlid map
\begin{boxwarning}
    When defining a subnet, always first make sure the index set is in fact directed. Then one needs to  specify the values of the function $\phi$, and then make sure that $\phi$ is a valid map (``as advertised'').
\end{boxwarning}

% Willard book general topology has things about nets

\subsection{Cluster Points}

Before we describe how the concept of closures being defined by sequences is generalised to arbitrary topological spaces using nets, we will find it useful to ask (and answer) the following question:
\begin{quote}
    \centering
    A subset of a subset is a subset. Is a subnet of a subnet also a subnet? 
\end{quote}
\begin{remark}
    The answer is ``yes it is'', and you get no prizes for showing so (let $\psi:\mathbb{F}\rightarrow \mathbb{E}$ and $\phi:\mathbb{E}\rightarrow \mathbb{D}$ define a subnet and a subnet of that subnet respectively - then consider the map $\phi\circ \psi$, and show that this defines a subnet of $\D$).
    %thanks
    % You can use \E, \F and \D for \mathbb{D} etc. Also, you can use \to instead of \rightarrow. 
    % sure
\end{remark}

For the remainder of this subsection,
\begin{itemize}
    \item A topological space $X$
    \item Directed posets $\D, \E, \F$
    \item Nets $\parenth{x_d}_{d \in \D}$, $\parenth{y_e}_{e \in \E}$, $\parenth{z_f}_{f \in \F}$ in $X$
    \item Cofinal and order-preserving maps $\phi : \E \to \D$ and $\psi : \F \to \E$
\end{itemize}

We define the notion of a cluster point.

\begin{boxdefinition}[Cluster Points]\label{Ch2:Def:ClusterPt}
    Fix $x \in X$. We say that $x$ is a \textbf{cluster point of $\parenth{x_d}_{d \in \D}$} if and only if for all open sets $U \ni x$ and $a \in \D$, there is some $b \geq a$ such that $x_b \in U$.
\end{boxdefinition}

There is an equivalent characterisation of cluster points in terms of subnets. The proof is ``one of those proofs where you follow your nose, and just do some stenography with the definitions.''  %%  Ahh yes indeed, thank you!!
%``This is completely in the wrong typeface'' (while writing on the board)

\begin{boxproposition}[An Equivalent Characterisation of Cluster Points]\label{Ch2:Prop:ClusterPt_iff}
    The following are equivalent.
    \begin{enumerate}[label = (\arabic*)]
        \item $x \in X$ is a cluster point of $\parenth{x_d}_{d \in \D}$
        \item $\parenth{x_a}_{a \in \D}$ has a subnet which converges to $x$
    \end{enumerate}
\end{boxproposition}
\begin{proof}
    We begin the process of nose-following and definitional stenography. Yay - fun!
    \begin{description}
        \item[\underline{$(2) \implies (1)$.}]
        Let $\parenth{y_e}_{e \in \E}$ and $\phi : \E \to \D$ together form a subnet of $\parenth{x_d}_{d \in \D}$. We show that $y_e \to x$.

        Let $U \ni x$ be open. Fix $d \in \D$. Let $e \in \E$ be such that for all $e' \geq e$, $y_{e'} \in U$. Let $e_1 \in \E$ be such that $\phi\of{e_1} \geq_{\D} d$. Since $\E$ is directed, we can find some $e' \in \E$ that simultaneously satisfies $e' \geq_{\D} e, e_1$.

        Since $\phi$ is order-preserving, we have that
        \begin{align*}
            \phi\of{e'} \geq \phi\of{e_1} \geq \phi(d)
        \end{align*}
        Since $e' \geq e$, we know that
        \begin{align*}
            x_{\phi\of{e'}} = y_{e'} \in U
        \end{align*}
        Thus, $x$ is a cluster point. As Professor Cummings said, ``At this point, e have ``won the game.'' Yippee!!

        \item[\underline{$(2) \implies (1)$.}]
        ``As you may have guessed, that was the less painful part of the proof.''
        
        Let $x$ be a cluster point of $\parenth{x_d}_{d \in \D}$. We are going to cook [up a convergent subnet of $x$]. Define
        \begin{align*}
            \E := \setst{\parenth{a, U}}{a \in \D, \, U \ni x, \, U \text{ is open}, \, x_a \in U}
        \end{align*}
        Define the ordering
        \begin{align*}
            \parenth{a, U} \leq_{\E} \parenth{b, V} \iff a \leq_{\D} b \text{ and } V \ssq U
        \end{align*}
        Define the map
        \begin{align*}
            \phi : \E \to \D : \phi\of{\parenth{a, U}} = a
        \end{align*}
        We now need to verify that $\phi$ is order-preserving and cofinal. The former is ``just a joke'' (i.e. follows by definition). The latter is true for the ``dumbest possible reason'': for all $d \in \D$, $\parenth{d, X} \in \E$ satisfies the property that $\phi\of{d, X} = d$.
        
        Next, we need to show that $\E$ is directed - which is pretty much immediate from the observation that $x$ is a cluster point. In particular, given $(a_1, U_1), (a_2, U_2)\in \E$ with $b\geq a_1\cap a_2, U_1\cap U_2$ open, as $x$ is a cluster point of the net we can find $c\geq b$ such that $x_c\in U_1\cup U_2$, and then $(c, U_1\cap U_2)\geq (a_1, U_1), (a_2, U_2)$.

        We can now define a subnet $y_{\parenth{a, U}} := x_a$. All that remains is to show that $\parenth{y_e}_{e \in \E}$ actually \textit{converges} to $x$. Fix $V \ni x$ an open neighbourhood of $x$. Find any old $b \in \D$ such that $x_b \in V$. Take $e = \parenth{b, V}$. Then, for all $e' = \parenth{b', V'} \geq_{\E} \parenth{b, V}$, we have
        \begin{align*}
            y_{e'} = x_{b'} \in V' \ssq V
        \end{align*}

        % I'm doing this earlier - follow me
        %sorry (of course feel free to delete anything)
        %no apologies ever needed
        % I just like reordering things. Sorry about that, it is a bit disruptive (even to my own live-TeXing, so I can't imagine how it is for someone else...
    \end{description}``I've told this joke before in a slightly different context.'' Also, verifying the subnet criteria is
    ``as fun as watching paint dry - but it's good to be thorough.''
\end{proof}
``We are kind of compelled to turn things upside down.''
%:)) (thank you once again) 
% sure!!  wouldn't miss pure gold like that :DDD

We now state an important fact that relates cluster points to ultranets

\begin{boxproposition}\label{Ch2:Prop:ofUltraNet_clusterPt_iff_tendsTo}
    Let $X$ be a topological space, $x$ a point in $X$, and $\parenth{x_d}_{d \in \D}$ an ultranet in $X$ (cf. \Cref{Ch2:Def:Ultranet}). The following are equivalent.
    \begin{enumerate}[label = (\arabic*)]
        \item $x$ is a cluster point of $\parenth{x_d}_{d \in \D}$
        \item $x_d \to x$
    \end{enumerate}
\end{boxproposition}
\begin{proof}
    \sorry
\end{proof}

\begin{boxcorollary}\label{Ch2:Cor:Ultranet_tendsTo_in_compact}
    If $X$ is compact and $\parenth{x_a}_{a \in \D}$ is an ultranet, then $\parenth{x_a}_{a \in \D}$ converges to some point.
\end{boxcorollary}
\begin{proof}
    Since $X$ is compact and $\parenth{x_a}_{a \in \D}$ is a net, we know that $\parenth{x_a}_{a \in \D}$ has a convergent subnet converging to some $x$. This makes $x$ a cluster point of $\parenth{x_a}_{a \in \D}$. Thus, by $(2) \implies (1)$ in \Cref{Ch2:Prop:ofUltraNet_clusterPt_iff_tendsTo}, $x_a \to x$.
\end{proof}

Next, we state ``Fact 4 in [Professor Cummings's] litany of facts'' (the first two being the points in \Cref{Ch2:Prop:ofUltraNet_clusterPt_iff_tendsTo} and the third being \Cref{Ch2:Cor:Ultranet_tendsTo_in_compact}).

\begin{boxcorollary}\label{Ch2:Cor:Func_comp_Ultranet_is_Ultranet}
    If $X$ and $Y$ are topological spaces and $\parenth{x_a}_{a \in \D}$ is an ultranet in $X$ and $f : X \to Y$ is any function, then $\parenth{f\of{x_a}}_{a \in \D}$ is an ultranet in $Y$.
\end{boxcorollary}
\begin{proof}[Proof sketch]
    Let $B \ssq Y$ be arbitrary. Define $A := f\inv\!\brac{B} \ssq X$. Use ultranet properties.
\end{proof}

Finally, we state a powerful theorem about ultranet (ultrasubnet? subultranet?) existence.

\begin{boxproposition}\label{Ch2:Prop:Subnet_Ultranet_Existence}
    Every net has a subnet which is an ultranet.
\end{boxproposition}
\begin{proof}
    \sorry
\end{proof}

Professor Cummings shall now cheat us in a somewhat innocuous way by using ultranets to prove Tychonoff's Theorem.

\subsection{Tychonoff's Theorem} % incredible theorem

% Hey Dang! Thanks for joining us!!
% Hellooooo I'm here!
% for emotional support obviously

We are now ready to prove the main result of this chapter (indeed, the result that motivated our development of the theory of nets, subnets and ultranets).

\begin{boxtheorem}[Tychonoff's Theorem]
    An \textit{arbitrary} product of compact spaces is compact.
\end{boxtheorem}
\begin{proof}
    Let $\parenth{X_i}_{i \in \I}$ be a family of compact spaces. Let $X := \prod_{i \in \I} X_i$. We will show that $X$ is compact by showing that every net in $X$ has a convergent subnet. Then, \sorry\todo{Add reference} will tell us that $X$ is compact.

    Let $\parenth{x_a}_{a \in \D}$ be a net in $X$. We doubly index $\parenth{x_a}_{a \in \D}$ in the following manner: for each $a \in \D$, we write
    \begin{align*}
        x_a = \parenth{x_{a, i}}_{i \in \I}
    \end{align*}
    That is, we denote by $x_{a, i}$ the $i$th component of $x_a$.
    
    Let $\parenth{y_b}_{b \in \E}$ be a subnet of $\parenth{x_a}_{a \in \D}$ such that $\parenth{y_b}_{b \in \E}$ is an ultranet in $X$ (as given by \Cref{Ch2:Prop:Subnet_Ultranet_Existence}). We again doubly index $\parenth{y_b}_{b \in \E}$ by writing the $i$th component of every $y_b$ as $y_{b, i}$. We will show that $\parenth{y_b}_{b \in \E}$ converges.

    For each $i \in \I$, let $\pi_i : X \to X_i$ be the canonical projection map sending $z \in X$ to its $i$th component, denoted $z_i$. By Fact 4 in the ``litany of facts'' (\Cref{Ch2:Cor:Func_comp_Ultranet_is_Ultranet}), since $\parenth{y_b}_{b \in \E}$ is an ultranet, so is $\parenth{\pi_i\of{y_b}}_{b \in \E}$. Indeed, observe that for all $i \in \I$ and $b \in \E$, $\pi_i\of{y_b} = y_{b, i}$. Thus, for all $i \in \I$, the ultranet $\parenth{\pi_i\of{y_b}}_{b \in \E}$ is exactly equal to $\parenth{y_{b, i}}_{b \in \E}$.

    Fix $i \in \I$. Since $X_i$ is compact, Fact 3 in the ``litany'' (\Cref{Ch2:Cor:Ultranet_tendsTo_in_compact}), there is some $y_i \in X_i$ such that $\parenth{y_{b, i}}_{b \in \E}$ converges to $y_i$. This gives us an element $y \in X$ defined to be the tuple $\parenth{y_i}_{i \in \I}$.

    We now claim that $\parenth{y_b}_{b \in \E}$ converges to $y$. If we can prove this claim, we will be done with the proof, as it will establish that $\parenth{y_b}_{b \in \E}$ is indeed a convergent subnet of $\parenth{x_a}_{a \in \D}$.

    Recall that there is an inherent \textit{finiteness} built into the product topology. In particular, the product topology is \textbf{strictly contained} in the box topology, because we require the box topology to be a product object in the category of topological spaces (see \Cref{APP:Subsec:Products} for more). This will be absolutely crucial in establishing the claim that $\parenth{y_b}_{b \in \E}$ converges to $y$.

    Fix a neighbourhood $U \ni y$ that is open with respect to the product topology on $X$. Fix a basic open set $\prod_{i \in \I} U_i$ containing $y$ such that $\prod_{i \in \I} U_i \ssq U$. We know there are finitely many indices $\set{i_1, \ldots, i_n} \ssq \I$ such that for $1 \leq s \leq n$, $U_{i_s} \subsetneq X_{i_s}$.

    For each $1 \leq s \leq n$, the net $\parenth{y_{b, i_s}}_{b \in \E}$ converges to $y_{i_s}$ in $X_{i_s}$. Indeed, $y_{i_s} \in U_{i_s}$, and $U_{i_s}$ is open in $X_{i_s}$, so there is some index $b_s \in \E$ such that for all $c \in \E$, if $c \geq_{\E} b_s$, then ${y_{c, i_s}} \in U_{i_s}$ Thus, we get a finite set of indices $b_1, \ldots, b_n \in \E$ such that each $b_s$ gives the threshold in \textit{that $X_{i_s}$} beyond which elements of $\parenth{y_{b, i_s}}_{b \in \E}$ all lie in $U_{i_s}$.

    Since $\E$ is directed, we know there exists some $b \in \E$ such that $b \geq b_1, \ldots, b_n$. We claim that for all $c \geq b$, $y_c \in U$.

    Fix some $c \geq b$. We show that for all $i \in \I$, $y_{c, i} \in U_i$ for all $i \in \I$. This would then imply that $y_c \in \prod_{i \in \I} U_i$, and we know that $ \prod_{i \in \I} U_i \ssq U$, so we would be done.

    So, fix $i \in \I$. Either $i = i_s$ for some $1 \leq s \leq n$ or not.
    \begin{description}
        \item[\underline{Case 1: $i = i_s$ for some $1 \leq s \leq n$.}]
        In this case, $y_{c, i_s} \in U_{i_s}$ because $c \geq b \geq b_s$.

        \item[\underline{Case 2: $i \neq i_s$ for any $1 \leq s \leq n$.}]
        In this case, $U_i = X_i$. Thus, $y_{c, i} \in X_i = U_i$.
    \end{description}
    Either way, each component of $y_c$ is contained in each $U_i$. Therefore, $y_c \in \prod_{i \in \I} U_i$, and since $\prod_{i \in \I} U_i \ssq U$, we can conclude that $y_c \in U$, and we are done.
\end{proof}

\begin{boxwarning}
    The distinction between the product and box topologies is \textbf{absolutely critical} for Tychonoff's Theorem! In fact, the conclusion typically \textbf{fails} when we take a box product.
\end{boxwarning}
%``interesting and fun to look at'' in topology is (perhaps often) equivalent to ``badly behaved'' - the box topology is ``interesting and fun to look at''

A detailed discussion on the difference between 
