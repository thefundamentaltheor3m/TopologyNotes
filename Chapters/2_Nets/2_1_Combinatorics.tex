\section{Some ``Completely Trivial Combinatorics''}

Fix a topological space $X$.

\subsection{Directed Sets and Nets}

We recall the definition of a poset from earlier in the course (\Cref{Ch1:Def:Poset}). We begin with a generalisation of this concept.

\begin{boxdefinition}[Directed Set]
    A \textbf{directed set} is a poset $\parenth{\D, \leq}$ such that for all $x, y \in \D$, there is some $z \in \D$ such that $x, y \leq z$.
\end{boxdefinition}

It is easy to extend the definition, by induction, to show that for any directed set $\parenth{\D, \leq}$ and $n \in \N$, for all $x_1, \ldots, x_n \in \D$, there is some $z \in \D$ such that $x_1, \ldots, x_n \leq z$.

\begin{boxexample}[Familiar Examples of Directed Sets]\label{Ch2:Eg:Dir_Sets}
    \hfill
    \begin{enumerate}
        \item $\parenth{\N, \leq}$ is a directed set.

        \item If $X$ is any infinite set, then the set
        \begin{align*}
            \D = \setst{a \ssq X}{\abs{a} < \omega}
        \end{align*}
        ordered by inclusion is a directed set.

        \item Let $X$ be a topological space and $x \in X$ a point. Define
        \begin{align*}
            \D \setst{U \ssq X}{U \text{ is open and } x \in U}
        \end{align*}
        $\D$ is ordered by the relation $U \leq V \iff V \ssq U$, and with this ordering, forms a directed set.
    \end{enumerate}
\end{boxexample}

Recall that at the start of the section, we fixed a topological space $X$. We define a \textbf{net in $X$} to be a \textit{sequence in $X$ indexed by a directed set}.

\begin{boxdefinition}[Net]\label{Ch2:Def:Net}
    A \textbf{net in $X$} is a function $\D \to X$, where $\D$ is a nonempty directed set. We denote a net by $\parenth{x_a}_{a \in \D}$, just as we would any sequence.
\end{boxdefinition}

It is clear that every sequence $\N \to X$ is a net $\parenth{\N, \leq} \to X$.

\subsection{Convergence of Nets}

What do we mean when we use words like ``eventually'' and ``frequently''? Let's find out.

\begin{boxdefinition}[Eventuality of Occurrence]
    Let $P(x)$ be a property of points $x \in X$. Let $\parenth{x_a}_{a \in \D}$ be a net. We say \textbf{$P$ occurs eventually} if there is some $a \in \D$ such that for all $b \in \D$, if $b \geq a$ then $P\of{x_b}$ holds in $X$.
\end{boxdefinition}

\begin{boxdefinition}[Frequency of Occurrence]
    \sorry
\end{boxdefinition}

Next, we state what it means for a net to converge.

\begin{boxdefinition}[Convergence]
    \sorry
\end{boxdefinition}

We can say something about \sorry.

\begin{boxproposition}
    Consider a subspace $A \ssq X$ and a point $x \in X$. TFAE:
    \begin{enumerate}[label = (\arabic*)]
        \item There is some directed set $\D$ and some net $\parenth{x_a}_{a \in \D}$ such that the following both hold:
        \begin{itemize}
            \item $x_a \in A$ for all $a \in D$
            \item $\parenth{x_a}_{a \in \D}$ converges to $x$
        \end{itemize}

        \item $x \in \closure{A}$
    \end{enumerate}
\end{boxproposition}
\begin{proof}\hfill
    \begin{description}
        \item[\underline{$(1) \implies (2)$.}]
        \sorry % "Essentially just definitions"

        \item[\underline{$(2) \implies (1)$.}]
        Consider the directed set
        \begin{align*}
            \D \setst{U \ssq X}{U \text{ is open and } x \in U}
        \end{align*}
        ordered by the relation $U \leq V \iff V \subseteq U$ (the third example in \Cref{Ch2:Eg:Dir_Sets}). We claim we can find a net $\parenth{x_U}_{U \in \D}$ such that $x_U \in U \cap A$. \sorry
    \end{description}
\end{proof}
%``there is fly in the ointment''
%``I'll tell you this for free''
%``it's good to know the actual definition of a subnet if you want to take the topology basic exam''
It is an ``amusing (if not very useful) fact'' that a space is Hausdorff if and only if nets have limits.

\subsection{Subnets}
%thanks, sorry about that - (haha) if you say so
% sure! I had \interior because \int means something else, so I made \closure, but that's horrible so this was an excuse to fix it (or rather add a more convenient macro)
%``being me, how am I going to prove ___ contradiction!''
% [...] :DD
Before we define what a subnet is, we define a property of maps between directed posets.

\begin{boxdefinition}[Cofinality]
    Let $\E, \D$ be directed posets. We say a function $\phi : \E \to \D$ is \textbf{cofinal} if for all $d \in \D$, there is some $e \in \E$ such that $d \leq \phi(e)$.
\end{boxdefinition}

For the Topology Basic Exam, it is ``good to know what the actual definition of a subnet is.'' Sounds like sage advice!(!!)

\begin{boxdefinition}[Subnet]\label{Ch2:Def:Subnet}
    Let $\parenth{x_d}_{d \in \D}$ be a net in $X$. A \textbf{subnet} of this is some net $\parenth{y_e}_{e \in \E}$ together with a map $\phi : \E \to \D$ such that
    \begin{enumerate}
        \item $\phi$ is order-preserving.
        \item $\phi$ is cofinal.
        \item $y_e = x_{\phi(e)}$ for all $e \in \E$.
    \end{enumerate}
\end{boxdefinition}

Now that we have the right definition of a subnet, we define a very tempting mistake to make. It is tempting to define the property of being a subnet to be a specialisation of the definintion given in \Cref{Ch2:Def:Subnet} to the case where $\E \ssq \D$ and $\phi$ is the inclusion. While there do exist subnets of this form, not every subnet takes this form.

\subsection{Ultranets}

This closely resembles the theory of filters and ultrafilters from set theory (or from Bourbaki, or from the topological corners of \mathlib).

\begin{boxdefinition}[Ultranet]\label{Ch2:Def:Ultranet}
    Let $X$ be a topological space and let $\parenth{x_d}_{d \in \D}$ be a net. We say that $\parenth{x_d}_{d \in \D}$ is an \textbf{ultranet} if for all $Y \ssq X$, \textit{at least one} of the following holds:
    \begin{enumerate}
        \item There is some $a \in \D$ such that for all $b \geq a$, we have $x_b \in Y$
        \item There is some $a \in \D$ such that for all $b \geq a$, we have $x_b \notin Y$
    \end{enumerate}
\end{boxdefinition}

The reason we said \textit{at least one} of the conditions should hold is that the two are not mutually exclusive. \sorry\todo{Give example where both hold}

This is related to the concept of locales in Grothendieck topology.

Professor Cummings leaves us with the following ``utterly trivial remark''.

\begin{boxexercise}\label{Ch2:Exo:Ultranet_comp}
    Let $X$ and $Y$ be topological spaces. If $\parenth{x_a}_{a \in \D}$ is an ultranet in $X$ and $f : X \to Y$ is any function, then $\parenth{f\of{x_d}}_{d \in \D}$ is an ultranet in $Y$.
\end{boxexercise}
\begin{remark}
    ``Inverse images play very nicely with complements in set theory...and I'm done, with a little bit of fast talking.''
\end{remark} 