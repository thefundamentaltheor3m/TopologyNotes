\section{Countability Properties}

\begin{boxdefinition}[1st, 2nd Countability]
    Let $X$ be a space, then
    \begin{enumerate}
        \item we say that $X$ is 1st countable to say that every point in $X$ has a countable neighbourhood basis (and...)
        \item we say that $X$ is second countable to say that $X$ has a countable basis.
    \end{enumerate}
\end{boxdefinition}
``one whose relationship with the first two might remain a little questionable for now...''

``those of you who took field theory might hate the name of this next one...''
\begin{boxdefinition}[Separable (``ohhhhhhhhhh...'' insert disappointment)]
We say that $X$ is ``separable'' to say that $X$ has a countable dense subset.
    
\end{boxdefinition}
\begin{boxdefinition}[Lindelof (``Poor Man's Subcover'')]
    We say that $X$ is Lindelof to say that every open covering of $X$ has a countable subcover.
\end{boxdefinition}
Now another definition, to get some feel for the ``geometry'' of a metric space
\begin{boxdefinition}[distance from a point to a set in a metric space]
    Let $(X,d)$ be a metric space, $A\ssq X, A\neq \emptyset$ (to pre-emptively take care of the empty set police) and $x\in X$. Then $d(x, A)=\inf\{d(x,a):a\in A\}$.
\end{boxdefinition}
\begin{boxtheorem}
    $d(x,A)$ is a continuous function of $x$.
\end{boxtheorem}
\begin{proof}
    Let $x,y\in X$ and $a\in A$, then $d(x, A)\leq d(x,a)\leq d(x,y)+d(y,a)\implies d(y,a)\geq d(x, A)-d(x,y)$ for all $a\in A\implies d(y,A)\geq d(x,A)-d(x,y)$. Now one can fill in the details at their leisure.
\end{proof}

With this, we can show that $X$ a metric space implies $X$ is $T_4$:
\begin{proof}
    Let $C,D\ssq X$ with $(X,d)$ is a metric space with $C,D$ closed and $C\cap D=\emptyset$. If we then let $g(x)=d(x,C)-d(x,D)$ [we NEED that $C,D$ are closed and disjoint!], then $g$ is continuous as a difference of two continuous functions. Let $X=g^{-1}[(-\infty, 0)]$ and $V=g^{-1}[(0, \infty)]$ - then maybe we're basically done \sorry.
\end{proof}

\begin{boxdefinition}[Embedding]
    We say $f:X\rightarrow Y$ is an ``embedding'' of $X$ into $Y$ if $f$ is an injective map with the additional constraint that $f$ is a homeomorphism from $X$ to $f[X]$ (where $f[X]\ssq Y$ is endowed with the subspace topology).
\end{boxdefinition}
\begin{boxwarning}
    Crucially, note $f[U]$ need not be open in $Y$ for $U$ open in $X$ - we only require $f[U]$ open in subspace topology of $f[X]\ssq Y$.
\end{boxwarning}
Recall that for spaces $X, Y_i$ ($i\in I$) that $F:X\rightarrow \prod_{i\in I}Y_i$ is continuous iff for all $i$, $\pi_i\circ F$ is continuous. In what follows, we let $f_i:=\pi_i\circ F$ given some $F(x)=(f_i(x))_{i\in I}$.

Our goal now is to find conditions on $f_i:X\rightarrow Y_i$ to ensure that $F$ is an embedding of $X$ in $\prod_{i\in I}Y_i$. To ensure that $F$ is continuous, each $f_i$ needs to be continuous. To be sure $F$ is injective, we need that for all $x,x'\in X$ that $x\neq x'\implies \exists i \in I$ such that $f_i(x)\neq f_i(x')$.

``and you're going to groan when I give you the next definition...''
\begin{boxdefinition}[Separating Points from Closed Sets]
    Given $X, (Y_i)_{i\in I}$ and continuous functions $f_i:X\rightarrow Y_i$ we say that $(f_i)_{i\in I}$ ``separates points from closed sets'' iff for all closed $C\ssq X$ with $x\in X, x\notin C$ here is some $i\in I$ such that $f_i(x)\notin \text{cl}(f_i[C])$.
\end{boxdefinition}
\begin{boxtheorem}
    If $f_i:X\rightarrow Y_i$ with $i\in I$ are such that each $f_i$ separates points, separates points on closed sets with $i\in I$ and is continuous, then $F$ is an embedding.
\end{boxtheorem}
Here is the proof, but ``you're not going to like it'' - is there set-theoretic nonsense??
\begin{proof}
    Given $U\ssq X$ open, we show $F[U]$ is relatively open in $F[X]$. In fact, we will show that for all $x\in U$ that there is $V\ssq \prod_i Y_i$, $F(x)\in V$, with $V$ open and $V\cap F[X]\ssq F[U]$. However observing that $x\notin X\setminus U$ a closed set, as $(f_i)_{i\in I}$ separates points from closed sets, there is $i$ such that $f_i(x)\notin \text{cl}(f_i[X\setminus U])$. 
    
    We now let $E=\text{cl}(f_i[X\setminus U])\ssq Y_i$ with $E$ closed. But then $f_i(x)\in Y_i\setminus E$ (an open set in $Y_i$). If we now let $V=\pi^{-1}[Y_i\setminus E]$, we can find that $V$ is open in $\prod_i Y_i$, and $F(x)\in V$ - we are left only to substantiate this claim.

    Finally then, we let $F(z)=V\cap F[X]$. As $F(z)\in V$, we have that $f_i(z)=\pi_i\circ F(z)\in Y_i\setminus E$. Now ``chasing through definitions'' we claim $z\in U$ and then proceed to observe that $z\notin X\setminus U \implies f_i(z)\in f_i[X\setminus U]$ and $f_i(z)\in \text{cl}(f_i[X\setminus U])=E$.
\end{proof}
So to ensure you get a topological embedding, ensure you separate points from points, then separate points from closed sets.

\section{Regularity}

Throughout this subsection, fix a topological space $X$.

We recall the definition of a regular space.

\begin{boxdefinition}[Regular Space]
    We say that $X$ is \textbf{regular} if for all closed $F \ssq X$ and $x \in X \setminus F$, there are $U, V \ssq X$ that are open such that $x \in U$, $F \ssq V$ and $U \cap V = \emptyset$.
\end{boxdefinition}

There is a strictly stronger notion as well.

\begin{boxdefinition}[Complete Regularity]
    We say $X$ is \textbf{completely regular} if for all $F \ssq X$ closed and $x \in X \setminus F$, there is a continuous function $f : X \to [0, 1]$ such that $f(x) = 0$ and $f[F] = \ssq \set{1}$.
\end{boxdefinition}

It makes complete sense for complete regularity to strictly generalise regularity. (See what I did there?) We do not show this here, but we reassure the reader that it is true.

In this section, we will explore how regularity and complete regularity interact with topological constructions.

\subsection{Regularity and Subspaces}

Fix a subset $A$ of $X$, and consider it as a topological space, endowed with the subspace topology inherited from $X$. Recall that $B \ssq A$ is closed with respect to the subspace topology if and only if $B = A \cap F$ for some $F \ssq X$ closed.

It turns out that subspaces of regular spaces are themselves regular.

\begin{boxlemma}
    If $X$ is regular, then so is $A$.
\end{boxlemma}
\begin{proof}
    Fix $B \ssq A$ closed and fix $a \in A \setminus B$. We know that $B = A \cap F$ for some closed $F \ssq X$. Since $a \notin B$, we must also have $a \notin F$. Then, since $X$ is regular, we can find $U, V \ssq X$ that are open and contain $a$ such that $F \ssq V$ and $U \cap V = \emptyset$. Then, $U \cap A$ and $V \cap A$ are relatively open subsets of $A$ that also contain $a$ and satisfy the properties that $B \ssq V \cap A$ and $\parenth{U \cap A} \cap \parenth{V \cap A} = \emptyset$.
\end{proof}
% Check out Appendix B
%haha, was goinf to put it there (many thanks) - most indeed :DDD
% I think we NEED an appendix dedicated to these kinds of quotes

We can say something analogous about complete regularity.

\begin{boxlemma}
    If $X$ is completely regular then so is $A$.
\end{boxlemma}
\begin{proof}
    Fix $B \ssq A$ relatively closed and $a \in A \setminus B$. We know $B = A \cap F$ for some $F \ssq X$ closed. We know there is some continuous $f : X \to [0, 1]$ such that $f(a) = 0$ and $f[F] = \set{1}$. Consider its restriction $g = f \restriction A$. Obviously $g(a) = 0$ because $a \in A$. Moreover, since $g$ agrees with $f$ on $A$, it is also clear that $g[A] = f[A] \ssq f[F] = \set{1}$.\footnote{I suppose the `empty set brigade' might have some objections but honestly who even cares about such people... (I kinda do but I am choosing to ignore my feelings in the interest of convenience!)}
\end{proof}

\begin{boxproposition}\label{Ch3:Prop:Regular_TFAE}
    The following are equivalent.
    \begin{enumerate}[label = (\arabic*)]
        \item $X$ is regular.
        \item for all $x \in X$ and all open $U \ni x$, there exists an open $V \ni x$ such that $\cl{V} \ssq U$.
    \end{enumerate}
\end{boxproposition}
\begin{proof}
    The picture we want to have in mind is given in \Cref{Ch3:Fig:Regular_TFAE}. \sorry\todo{Finish!}

    \begin{figure}[H]
        \centering
        \begin{tikzpicture}
            \draw (-2,-2) rectangle (4, 4); % TODO: Finish
        \end{tikzpicture}
        \caption{The picture we want to have in mind for \Cref{Ch3:Prop:Regular_TFAE}.}
        \label{Ch3:Fig:Regular_TFAE}
    \end{figure}

    \begin{description}
        \item[\underline{$(1) \implies (2)$.}]
        Fix $x \in X$ and some open $U \ni x$. Evidently, $x \notin X \setminus U$, and since $X \setminus U$ is closed and $X$ is regular, we can find some open $V, W \ssq X$ such that $x \in W$, $X \setminus U \ssq W$, and $W \cap V = \emptyset$. Then, the closure $\cl{V} \ssq X \setminus W$, and this is contained in $U$.

        \item[\underline{$(2) \implies (1)$.}]
        Exercise. \sorry \todo{Finish!}
    \end{description}
\end{proof}

\subsection{Regularity and Products}

Our eventual goal in this subsection will be to show that products of regular spaces are regular.

Throughout this subsection, fix topological spaces $\parenth{X_i}_{i \in I}$. Denote by $X$ the product $\prod_{i \in I} X_i$.

\begin{boxlemma}
    If $\parenth{B_i}_{i \in I}$ are basic open sets of the $X_i$, then
    \begin{align*}
        \prod_{i \in I} \cl{B_i} = \cl{\prod_{i \in I} B_i}
    \end{align*}
\end{boxlemma}
\begin{proof}
    \sorry % I'm usually quick with these proofs - but unfortunately I didn't quite get this one... briefly zoned out and looked at some other things (sorry)
\end{proof}

We can now show that a product of regular spaces is regular.

\begin{boxtheorem}
    If each $X_i$ is regular, then so is $X$.
\end{boxtheorem}
\begin{proof}
    Fix $x = \parenth{x_i}_{i \in I}$ lie in some open set $U \ssq X$. We can write $U_i$ as a product of open sets $U_i$, with only finitely many $U_i$ being \textit{properly} contained in $X_i$. \sorry
\end{proof}

We can say something analogous for completely regular spaces.

\begin{boxtheorem}
    If each $X_i$ is completely regular, then so is $X$.
\end{boxtheorem}
\begin{proof}
    Fix $x \in X$ and $F \ssq X$ closed, and assume $x \notin F$. Let $U$ be a basic open set containing $x$ and disjoint from $F$. We know that there are open subsets $U_i \ssq X_i$ for every $i \in I$ such that
    \begin{align*}
        U = \prod_{i \in I} U_i
        \qquad \text{and} \qquad
        A := \setst{i \in I}{U_i \subsetneq X_i} \text{ is finite}
    \end{align*}
    We know that $F \ssq X \setminus U$. Moreover, since $x \in U$, $x \notin X \setminus U$. Finally, note that
    \begin{align*}
        y \in U &\iff \text{For all } i \in I,\ y_i \in U_i \\
        &\iff \text{For all } i \in A,\ y_i \in U_i
    \end{align*}
    Since $X_i$ is completely regular, for each $i \in A$, we can find continuous functions $f_i : X_i \to [0, 1]$ such that $f_i\of{x_i} = 0$ and $f_i(z) = 1$ for all $z \in X \setminus U_i$. We can then define
    \begin{align*}
        f : X \to [0, 1] : y \mapsto \max_{i \in A} f_i\of{y_i}
    \end{align*}
    Observe that $f(x) = 0$. For all $y \in F$, $y \in X \setminus U$, so there is some $i \in A$ such that $y_i \in X_i \setminus U_i$ for all $i$. We then see that $f_i\of{y_i} = 1$, so $f(y) = 1$.
\end{proof}

\begin{boxdefinition}[Hilbert Cube]
    Let $I$ be a nonempty set. The ``Hilbert cube'' associated with $I$ (i.e. ``from $I$'') is the space $[0,1]^I$, the product of copies of $[0,1]$ indexed by elements of $I$ endowed with the product topology.
\end{boxdefinition}
%thank you for that - was about to get it down :)
% Sure :DD
This is `a potentially very large cube-like thingy'.

What properties of the unit interval would we expect the Hilbert Cube to have? Compactness? Hausdorffness? Thus, $T_4$? Thus $T_{3.5}$? All of the above, actually - and purely by results we have seen so far! Cool, eh?
%wow!

We can ask ourselves what spaces can be embedded in Hilbert cubes. Indeed, we will be asking a lot of questions of this type over the course of this course.

% I have added the definitions of $T_4$ and $T_{3.5}$ to the appropriate previous section.

Certainly we would want such a space to be $T_{3.5}$. Indeed, we know that subspaces of $T_{3.5}$ spaces are $T_{3.5}$. So being $T_{3.5}$ is necessary for being embeddable in the Hilbert Cube.

But astonishingly, it turns out that the converse is also true!

\begin{boxtheorem}
    If $X$ is a $T_{3.5}$ space, then $X$ is embeddable into some Hilbert cube.
\end{boxtheorem}
\begin{proof}
    Define $I := \setst{f : X \to [0, 1]}{f \text{ is continuous}}$. Since $X$ is $T_{3.5}$, $I$ separates points and $I$ separates points from closed sets---that is, for $x \in X$ and $C \ssq X$ with $x\notin C$, there is some $f \in I$ such that $f(x) \notin \cl{f[C]}$.

    An `old theorem'\todo{Add ref from earlier in course} says that if we define $H : X \to [0, 1]^I : x \mapsto \parenth{f(x)}_{f \in I}$, then $H$ is the desired embedding.
\end{proof}

\section{Urysohn's Lemma}

In this section, we develop the tools needed to prove Urysohn's Lemma, which (according to Professor Cummings) should really be a \textit{theorem}.

It goes as follows.

\begin{boxtheorem}[Urysohn's Lemma]\label{SP:Urysohn}
    If $X$ is a normal topological space, for all closed $C, D \ssq X$, if $C \cap D = \emptyset$ then there is a continuous $f : X \to [0, 1]$ such that $f[C] \ssq \set{0}$ and $f[D] \ssq \set{1}$.
\end{boxtheorem}

In some sense, normality tells us we can `separate closed sets using open sets', and Urysohn's Lemma says that if we can do that, then we can also `separate closed sets using continuous functions'.

For the remainder of this section, fix a normal topological space $X$. Our proof will be \textit{constructive}: we will give an explicit function.

%as you wish of course - as you saw, when I was note-taking (for that brief brief period) I made no new subsections, and also last I checked there is more missing on the separation properties front

% I think this should be a subsection inside the more on separation properties section. I also think there should be more subsections of that section.

\begin{boxlemma}\label{Ch3:Lemma:Urysohn_helper}
    Let $F$ be a closed subset of $X$ and $O$ an open subset of $X$. If $F \ssq O$, then there exist $O_1, F_1 \ssq X$ such that $O_1$ is open, $F_1$ is closed, and
    \begin{align*}
        F \ssq O_1 \ssq F_1 \ssq O
    \end{align*}
\end{boxlemma}
% The picture we want to have in mind is the following:
% \begin{figure}
%     \centering
%     \begin{tikzpicture}
        
%     \end{tikzpicture}
%     \caption{Caption}
%     \label{fig:placeholder}
% \end{figure}
\begin{proof}
    The proof is just an ``irritating game involving naïve set theory''.

    If $F \ssq O$, then $F \cap \parenth{X \setminus O} = \emptyset$. Normality then tells us that we have disjoint open sets $U, V \ssq X$ such that $F \ssq U$ and $X \setminus O \ssq V$. Simply take $O_1 = U$ and $F_1 = X \setminus V$.
\end{proof}

We will also briefly explain what the dyadic rationals are.

\begin{boxdefinition}[Dyadic Rational]
    We say $q \in \Q$ is \textbf{dyadic} if there is some $c \in \Z$ and $i \in \N$ such that $q = \frac{c}{2^i}$.
\end{boxdefinition}

The dyadic rationals are dense in $\R$: indeed, this is why real numbers admit binary expansions.

\begin{boxlnotation}
    Denote by $D$ the set of all dyadic rationals contained in $(0, 1)$.
\end{boxlnotation} % Local notation :DD

We are now ready to prove Urysohn's Lemma.

\begin{proof}[Proof of Urysohn's Lemma (\Cref{SP:Urysohn}).]
    We will proceed by constructing sets $\setst{U_q}{q \in D}$ such that
    \begin{enumerate}[noitemsep]
        \item $C \ssq U_q \ssq X \setminus D$ for all $q \in D$.
        \item If $p, q \in D$ and $p < q$ then $\cl{U_p} \ssq U_q$.
    \end{enumerate}
    To do this, we will actually build $\setst{U_q, V_q}{q \in D}$ such that for all $q \in D$,
    \begin{enumerate}[noitemsep]
        \item $U_q$ is open
        \item $V_q$ is closed
        \item $V_q \ssq U_r$ for $r > q$
    \end{enumerate}
    At this stage, it will be useful to draw a picture.
    % Can I live-TeX a picture? Let's find out...
    % as per usual, incredibly impressive
    % Well it would've been if it had worked...
    % quite honestly, nonetheless...
    % \begin{figure}[H]
    %     \centering
    %     % TODO: Translate this left
    %     \begin{tikzpicture}
    %         \draw (0, 0) circle (5pt); % C
    %         \draw (1, 0) circle (5pt); % D

    %         % U 1/4, V 1/4
    %         \draw [blue,thin,domain=-30:30] plot (0.2 * {cos(\x)}, 0.2 * {sin(\x)});
    %         \draw [blue,thin,domain=-30:30] plot (0.3 * {cos(\x)}, 0.3 * {sin(\x)});
            
    %         % U 1/2, V 1/2
    %         \draw [blue,thin,domain=-30:30] plot ({0.4 * cos(\x)}, {0.4 * sin(\x)});
    %         \draw [blue,thin,domain=-30:30] plot ({0.6 * cos(\x)}, {0.6 * sin(\x)});
    %     \end{tikzpicture}
    %     \caption{Caption}
    %     \label{fig:placeholder}
    % \end{figure}
    Here's a more formal way of describing the construction.
    \begin{enumerate}
        \item By \Cref{Ch3:Lemma:Urysohn_helper}, we can choose an open set $U_{\frac{1}{2}}$ and a closed set $V_{\frac{1}{2}}$ with $$C\ssq U_{\frac{1}{2}}\ssq V_{\frac{1}{2}}\ssq X\setminus D$$
        \item Then, given $C \ssq U_{\frac{1}{2}}$, we can apply \Cref{Ch3:Lemma:Urysohn_helper} to get $U_{\frac{1}{4}}$ open and $V_{\frac{1}{4}}$ closed such that
        \begin{align*}
            C \ssq U_{\frac{1}{4}} \ssq V_{\frac{1}{4}} \ssq U_{\frac{1}{2}}
        \end{align*}
        Similarly, since $V_{\frac{1}{2}} \ssq X \setminus D$, we can again apply \Cref{Ch3:Lemma:Urysohn_helper} to get $U_{\frac{3}{4}}$ open and $V_{\frac{3}{4}}$ with
        \begin{align*}
            V_{\frac{1}{2}} \ssq U_{\frac{3}{4}} \ssq V_{\frac{3}{4}} \ssq X \setminus D
        \end{align*}
        \item At this point it should be clear what all the steps (which we will not enumerate) are going to look like...in any case, we now have all our $U_q, V_q$. 
        %In case it is not, say we have constructed $U_q, V_q$ for $q_1, q_2, \ldots, q_{2^{n+1}-2}\in D\cap \{\frac{a}{2^k}: k<n\}$ (where $i<j\implies q_i<q_j$). If $q_i=\frac{a}{2^n}$ (i.e. for some $a\in (0, 2^n)$) then between $V_{q_i}$ and $U_{q_{i+1}}$ we can find an open $U_{q_i+\frac{1}{2^{n+1}}}$ and a closed $V_{q_i+\frac{1}{2^{n+1}}}$ with 
    \end{enumerate}
    It will be incredibly useful to bear in mind that for all dyadic rationals $p < q$, we get
    \begin{align*}
        C \ssq U_p \ssq X \setminus D
        \qquad \text{ and } \qquad U_p\ssq V_p\ssq U_q \implies
        \cl{U_p} \ssq U_{q}
    \end{align*}
    % Muchas gracias senora
    %por supuesto
    We will now get on to the construction of our continuous function. To do so, we define $f:X \to [0,1]$ such that, if there us no $r\in D$ with $x\in U_r$, then we let $f(x) := 1$ - otherwise, we define $f(x) := \inf\!\setst{r\in D}{x\in U_r}$.

    It is ``easy'' to see that if $x\in D$, then $f(x)=1$, as $x\in C\implies f(x)=\inf(D)=0$. However we must now get on to the trickier part...we claim that for all $x \in X$ and $p \in D$,
    \begin{enumerate}[label = (\alph*), noitemsep]
        \item If $x \in \cl{U_p}$, $f(x) \leq p$
        \item If $x \notin \cl{U_p}$, $f(x) \geq p$
    \end{enumerate}
    Indeed, if $x \in \cl{U_p}$, then $x \in U_q$ for all $q \in D$ with $q > p$, so $f(x) \leq p$. On the other hand, if $x \notin \cl{U_p}$, then assume, for contradiction, that $f(x) < p$. Then we can easily see that there must be some $r < p$ and $x \in U_r$ such that $f(x) \leq r < p$. But we know that $r<p$ and $x\in U_r\implies x\in U_p\ssq \cl{U_p}$ - a contradiction! So indeed, things are just as we wanted.%\sorry

    Now that we know our claims are true, taking the contrapositive of each claim we have that for all $x \in X$ and $p \in D$,
    \begin{enumerate}[label = (\alph*), noitemsep]
        \item If $f(x) > p$ then $x \notin \cl{U_p}$ (and thus $x \notin U_p$, so $f(x) \geq p$).
        \item If $f(x) < p$ then $x \in U_p$ (and thus $f(x) \leq p$).
    \end{enumerate}

    Consider the following (very natural) basis for the Euclidean (subspace) topology on $[0,1]$:
    \begin{align*}
        \setst{[0, a)]}{0 < a} \cup
        \setst{(a, b)}{0 \leq a < b \leq 1} \cup
        \setst{(b, 1]}{b < 1}
    \end{align*}
    We can now complete our proof by showing that $f$ is continuous - to do so, it suffices to show that the pre-image in $f$ of every basic open subset of $[0, 1]$ is open.
    
    We begin by showing that $f\inv\!\brac{(a, b)}$ is open. We do this by showing every point has an open neighbourhood. Fix $x \in f\inv\!\brac{(a, b)}$. We know that $a < f(x) < b$. Find $p, q \in D$ such that $a < p < f(x) < q < b$. Then $f(x) > p$, so $x \notin \cl{U_p}$. Similarly, $f(x) < q$, so $x \in U_q$. Now, let $V = U_q \setminus \cl{U_p}$. Observe that $x \in V$. Moreover, for all $y \in V$, $y \in U_q$, so $f(y) \leq q$, and $y \notin U_p$, so $f(y) \geq p$. Then, we can see that $f[V] \ssq [p, q] \ssq (a, b)$.
    % I'm still on the left board, please feel free to type away

    %sure - this was bottom of the left board - once I can see again, haha
    
    So for $f\inv [[0,a)]$ we let $f(x)<a$ and find $p\in D$ with $f(x)<p<a$, $x\in U_p$ and $f[U_p]\ssq [0,p]\ssq [0,a)$. Meanwhile, for $f\inv [(b,1]]$ we let $f(x)>b$ we find $q\in D$ such that $b<q<f(x)$ and the $x\notin \cl{U_q}$ and $y\in X\setminus \cl{U_q}$ then $y\notin U_q\implies f(y)\geq q>b$. 
\end{proof}

The good news is that we have a very rich supply of normal spaces. Indeed, we know that Metric spaces are Hausdorff and normal ($T_4$). 

\section{The Stone-Čech Compactification}

\begin{boxdefinition}[Compactification]
    Given a space $X$, we say that $X$ has the \textbf{compactification} $Y$ if there is a space $Y$ such that the following hold:
    \begin{enumerate}
        \item $X$ is a subspace of $Y$.
        \item $X$ is dense in $Y$.
        \item $Y$ is compact.
    \end{enumerate}
\end{boxdefinition}

\subsection{One-Point Compactification}

\begin{boxdefinition}[One-Point Compactification]
    Let $X$ be a locally compact, non-compact Hausdorff space. Let $Y = X \cup \set{\infty}$, where $\infty$ is a formal symbol that does not lie in $X$, endowed with the topology $\tau$ consisting of
    \begin{enumerate}
        \item Open subsets of $X$.
        \item Sets of the form $\infty \cup \parenth{X \setminus K}$, where $K \ssq X$ is compact.
    \end{enumerate}
    We call $Y$ the \textbf{one-point compactification of $X$}. We sometimes denote $Y$ by $\alpha X$.
\end{boxdefinition}

Think of projective space.

Suppose, as before, that $X$ is a $T_{3.5}$ topological space. Define
\begin{align*}
    I = \setst{f : X \to [0, 1]}{f \text{ is continuous}}
\end{align*}

\subsection{Constructing the Stone-Čech Compactification}

Throughout this subsection, let $X$ be a $T_{3.5}$/Tychonoff space. That is, $X$ is Hausdorff and completely regular (cf. \Cref{Ch3:Def:T3.5}). Write
\begin{align*}
    \F := \setst{f : X \to [0,1]}{f \text{ is continuous}}
\end{align*}
Observe that $\F$ separates points (because $X$ is Hausdorff) and also separates points from closed sets (because $X$ is completely regular).

Let $Y = [0, 1]^{\F}$, endowed with the product topology. We know, by Tychonoff's Theorem (\sorry), that $Y$ is compact. We also know $Y$ is Hausdorff. So $Y$ is $T_{3.5}$ as well.

Consider $H : X \to Y : x \mapsto \parenth{f(x)}_{f \in \F}$. We can show that $H$ is an embedding of topological spaces, ie, that $H$ is continuous and injective.

\begin{boxdefinition}[Stone-Čech Compactification]
    The \textbf{Stone-Čech Compactification of $X$}, denoted $\beta X$, is the closure in $Y$ of the image of $X$ under $H$, where $H$ is the embedding described above. Ie, %I wasn't sure if I was misreading not at all, yes I see (thanks)
    \begin{align*} % Better? Haha nw - I just meant the image is in the function $H$. Sure! The equation below is probably clearer
        \beta X = \cl{H[X]}
    \end{align*}
    From a categorical standpoint, the data of the Stone-Čech Compactification consists of both the space $\beta X$ and the embedding $H : X \to \beta X$.
\end{boxdefinition}

As one would hope from the name, $\beta X$ is indeed compact. Moreover, $\beta X$ contains $H[X]$, which is homeomorphic to $X$. So it does indeed make sense to call $\beta X$ a compactification of $X$. As an added bonus, $\beta X$ is Hausdorff.

It will also be sensible to show that the above compactification process behaves exactly as one would hope on spaces that are already compact: it ``does nothing''.

Let $K$ be compact and Hausdorff. $K$ is $T_4$, so $K$ is $T_{3.5}$, and so one can define its Stone-Čech Compactification $\beta K$. Define
\begin{align*}
    G := \setst{f : K \to [0, 1]}{g \text{ is continuous}}
\end{align*}
Consider the embedding $\bar{H} : K \to [0, 1]^G : k \mapsto \parenth{g(k)}_{g \in G}$. Then $\beta G = \cl{\bar{H}[K]}$.

Since $K$ is compact and $\bar{H}$ is continuous, $\bar{H}[K]$ is compact. Moreover, $\bar{H}$ is a bijection from $K$ to $\bar{H}[K]$, and $\bar{H}[K]$ is Hausdorff. Thus, the corestriction of $\bar{H}$ to its image is a continuous bijection from a compact space to a Hausdorff space, and thus a homeomorphism. So indeed, $\beta K$ is homeomorphic to $K$.

\subsection{The Universal Property of the Stone-Čech Compactification}

It turns out that we can characterise the Stone-Čech Compactification using a universal property. Indeed, as one might guess, $\beta$ is functorial, so it makes sense that we can do something categorical here...

The desired universal property turns out to be an \textbf{extension property}. What it says is that the Stone-Čech Compactification is universal in the sense of every compactification of $X$ into a compact, Hausdorff space factors uniquely through $\beta X$.

\begin{boxtheorem}[Universal Property of the Stone-Čech Compactification]\label{Ch3:Thm:Stone-Cech-UP}
    Let $X$ be $T_{3.5}$ and let $H : X \to \beta X$ be the embedding described above. Let $K$ be any compact, Hausdorff space and let $f : X \to K$ be any continuous function. Then there is a unique continuous $g : \beta X \to K$ such that $f = g \circ H$. That is, the following diagram commutes (in the category of topological spaces):
    \begin{cd*}
        \beta X \arrow[dr, "\exists! g", dashed] & \\[1em]
        X \arrow[u, "H", hook] \arrow[r, "f"'] & K
    \end{cd*}
\end{boxtheorem}
\begin{proof}
    ``I'm not doing this in the \textit{most} bafflingly slick way, but I'm still doing it in a somewhat bafflingly slick way.'' ``I have made everything maximally confusing---it's a gift.''
    
    As above, denote
    \begin{align*}
        \F := \setst{f : X \to [0, 1]}{f \text{ is continuous}}
    \end{align*}
    Then, $\beta X = \cl{H[X]}$ and $H(x) = \parenth{f(x)}_{f \in \F}$.

    \begin{enumerate}
        \item \underline{Special Case: $K = [0, 1]$.}
        Fix a continuous map $f : X \to [0, 1]$. We need to show there is a unique $g : \beta X \to [0, 1]$ such that $g \circ H = f$. Since $f \in \F$ and $\beta X \ssq [0, 1]^{\F}$, we can define $g := \pi_f \restriction \beta X$, where $\pi_f$ is the projection from $[0, 1]^{\F}$ to the $f$ coordinate. Obviously $\pi_f$ is continuous and makes the triangle commute.

        It remains to show uniqueness. Indeed, the key is that $H[X]$ is dense in $\cl{H[X]} = \beta X$. So if we had $g' : \beta X \to K$ such that $g' \circ H = g \circ H$, then we would require $g'$ and $g$ to agree on the dense subset $H[X]$ of $\beta X$, so they must agree on all of $\beta X$.

        \item \underline{General case: $K$ is an arbitrary (Hausdorff) compactification of $X$.}
        Fix a continuous $f : X \to K$. Denote by $\mathcal{G}$ the set of all continuous functions from $K$ to $[0, 1]$. For all $t \in \mathcal{G}$, we know $t \circ f \in \F$.

        Define $G : \beta X \to [0, 1]^{\mathcal{G}} : \parenth{u_f}_{f \in \F} \mapsto \parenth{u_{t \circ f}}_{t \in \mathcal{G}}$.
        
        First, we show that $G$ is continuous. Denote by $\pi_t$ the projection from $[0,1]^{\mathcal{G}}$ to the $t$th coordinate. One can show that $\pi_t \circ G = \pi_{t \circ f}$, and it follows readily from this that $G$ is continuous.

        Next, we show that $G[\beta X] \ssq \cl{\bar{H}[K]}$. Fix $u = \parenth{u_s}_{s \in \F} \in \beta X$. We know that $G(u)$ is contained in some basic open subset of $[0, 1]^{\mathcal{G}}$. Such sets are finite intersections of images of projections: that is, there exist $t_1, \ldots, t_n \in \mathcal{G}$ and open sets $V_1, \ldots, V_n \ssq [0, 1]$ such that
        \begin{align*}
            G(u) \in \bigcap_{i = 1}^{n} \pi_{t_i\inv}\!\brac{V_i}
        \end{align*}
        Then, $u_{t_i \circ f} = \pi_{t_i}\of{G(u)} \in V_i$ for all $1 \leq i \leq n$. Thus,
        \begin{align*}
            u \in \bigcap_{i=1}^{n} \pi_{t_i \circ f}\inv\!\brac{V_i}
        \end{align*}
        Since $u \in \beta X = \cl{H[X]}$, we can find a point $x \in X$ such that $H(x) \in \bigcap_{i=1}^{n} \pi_{t_i \circ f}\inv\!\brac{V_i}$. So for $1 \leq i \leq n$, $H(x)_{t_i \circ f} = \parenth{t_i \circ f}\of{x} \in V_i$.
        
        ``Our suffering is almost over.'' Let $k = f(x) \in K$. Then, $t_i(k) \in V_i$ for $1 \leq i \leq n$, so $\bar{H}(k) \in \bigcap_{i=1}^{n} \pi_{t_i\inv}\!\brac{V_i}$. So $G(u) \in \cl{\bar{H}[K]}$, so $G[\beta X] \ssq \cl{\bar{H}[X]} = \bar{H}[X] = \beta K$. We then find ourselves in the following situation:
        \begin{cd*}
            & \beta X \arrow[d, dashed, "g"] \arrow[dr, "G"] & \\
            X \arrow[ur, "H"] \arrow[r, "f"] & K \arrow[r, "\sim", "\bar{H}"'] & \bar{H}[K]
        \end{cd*}
        where for all $u \in \beta X$, we define $g(u)$ to be the unique $k \in H$ such that $\bar{H}(k) = G(k)$.
        
        ``Ok, I'm tired of this proof, you're probably also tired of this proof, so let's just check a few more details and move on.''
        
        We finish by verifying that $g$ is continuous. \sorry
    \end{enumerate} %I presume you've gotten that down  B)% Of course, no way I'd miss pure gold like that :DDDD
\end{proof}

\subsection{Applications of Stone-Čech Compactification}

We already saw how the Stone-Čech Compactification behaves with compact spaces. It turns out to be interesting to explore how it behaves with locally compact spaces.

We begin with a fact (the proof of which is left as an exercise).

\begin{boxexercise}\label{Ch3:Exercise:Stone-Cech-TFAE-aux}
    If $Y$ is compact and Hausdorff and $X$ is an open subspace of $Y$, then $X$ is locally compact.
\end{boxexercise}
\begin{remark}
    Note that when $X$ is an open subspace of $Y$, then the subspace topology on $X$ consists exactly of those open sets (in the topology on $Y$) which are contained in $X$ (trivially so, but still this may be a helpful fact)...
\end{remark}

\begin{boxtheorem}
    Let $X$ be a $T_{3.5}$ space with $H : X \to \beta X$ the Stone-Čech Compactification, then TFAE:
    \begin{enumerate}[label = (\arabic*)]
        \item $X$ is locally compact.
        \item $H[X]$ is open in $\beta X$.
    \end{enumerate}
\end{boxtheorem}
\begin{proof} \hfill
    \begin{description}
        \item[\underline{$(2) \implies (1)$}] Follows from \Cref{Ch3:Exercise:Stone-Cech-TFAE-aux}.
        
        \item[\underline{$(1) \implies (2)$}] If $X$ is compact, then $H[X] = \beta X$. If $X$ is not compact, let $\alpha X$ denote its one-point compactification, with inclusion $\iota : X \inj \alpha X$. Recall that $\alpha X$ is compact and Hausdorff. The Universal Property (\Cref{Ch3:Thm:Stone-Cech-UP}) then gives us some (uniquely defined) $g : \beta X \to X$ such that $g \circ H = \iota$.
        \begin{cd*}
            \beta X \arrow[dr, "\exists! g", dashed] & \\[1em]
            X \arrow[u, "H", hook] \arrow[r, "\iota"'] & K
        \end{cd*}
        One can show that $\beta X \setminus H[X] = g\inv\!\brac{\set{\infty}}$\footnote{``I'd rather put that on the homework than show it in class'' - Professor Cummings}. Since $\alpha X$ is Hausdorff, $\set{\infty}$ is closed, so $\beta X \setminus H[X]$ is closed, thus $H[X]$ is open.
    \end{description}
\end{proof}

\section{Urysohn's Metrisation Theorem}
% Consider merging with Urysohn's Lemma and calling it "Urysohnian Madness: The Amazing Contributions of Pavel Urysohn"

Urysohn's Metrisation Theorem gives conditions under which we can construct a metric on a topological space that induces the same topology on it.

\begin{boxtheorem}[Urysohn's Metrisation Theorem]\label{SP:Urysohn_Metrisation}
    Let $X$ be a topological space. If $X$ is both $T_3$ and second-countable, then $X$ is metrisable.
\end{boxtheorem}

The way we will prove this theorem is by embedding $X$ into a metrisable space, namely, $[0, 1]^{\N}$. This is the same sort of object that comes up in the construction of the Stone-Čech Compactification.

It's not obvious that $[0, 1]^{\N}$ really is metrisable - we will 'brush it under the rug' because it is 'a consequence of some nonsense on this week's homework'. The metric we define is
\begin{align*}
    d\of{x, y} = \sum_{n \in \N} 2^{-n} \abs{x_n - y_n}
\end{align*}
and it isn't hard to show that this sum converges absolutely for all $x, y \in [0, 1]^{\N}$.

We begin by proving some more general facts, which we will later relate to the proof of Urysohn's Metrisation Theorem (\Cref{SP:Urysohn_Metrisation}).

% \subsection{Countability Properties Revisited}
% Consider moving to countability properties section

Recall the definition of a Lindelöf space.
\begin{boxdefinition}[Lindelöf Space]
    A space $X$ is \textbf{Lindelöf} if every open cover has a countable subcover.
\end{boxdefinition}
It is not hard to show that any closed subset of a Lindelöf space is Lindelöf (when endowed with the subspace topology).

\begin{boxlemma}\label{Ch3:Lemma:2nd_countable_is_Lindelof}
    Any 2nd countable space is Lindelöf.
\end{boxlemma}
\begin{proof}
    If you remember what a 2nd countable space is, you can probably prove this yourself.
    
    If not, recall that a 2nd countable space is defined as having a countable basis. We can express every open set $O_i$ in an arbitrary open cover $X=\bigcup_{i\in I}O_i$ as a union of basis elements. For each basis element $B$ we can choose some $O_i$ in our open cover with $B\ssq O_i$ - evidently the union of all such $O_i$ will cover $X$, and just as evidently there will be only countably many of them (so here we have our countable subcover).
\end{proof}

\begin{boxlemma}
    Any space that is both Lindelöf and $T_3$ is $T_4$.
\end{boxlemma}
\begin{proof}
    Let $A, B \ssq X$ be closed and disjoint. Notice that both $A$ and $B$ (as subspaces) are Lindelöf themselves. We will use the hypothesis that $X$ is $T_3$ to cook up open covers of $A$ and $B$.

    Since $X$ is $T_3$ and $A,B$ are closed, for each $a\in A$ we can find disjoint open sets $U_a, O_a$ with $a\in U_a, B\ssq O_a$. Note that $U_a\cap O_a=\emptyset \implies U_a\ssq X\setminus O_a\ssq X\setminus B$ - in particular, since $X\setminus O_a$ is closed, we must have $\cl\of{U_a}\cap B=\emptyset$. We now find some such open $U_a$ for all $a\in A$, and we similarly find an open $V_b$ for all $b\in B$ satisfying $b\in V_b$ and $\cl\of{V_b}\cap A=\emptyset$ for all $b\in B$.

    Now since $A,B$ are both Lindelöf, we can find countably many $a_n\in A, b_n\in B$ to obtain (countable) subcovers with the form $\{U_{a_n}:n\in \N\}, \{V_{b_n}:n\in \N\}$ covering $A,B$.
    
   Remember that we need to find disjoint open sets containing $A,B$. We do this by constructing countably many disjoint open sets $Y_n, Z_n$, where each $Y_n$ covers part of $A$ and each $Z_n$ covers part of $B$ - the sets $Y_n, Z_n$ will be constructed using the sets in our countable open covers of $A,B$ (respectively). We define the sets $Y_n, Z_n$ as follows:
    \begin{align*}
        Y_n &:= U_{a_n} \cap \parenth{\bigcap_{k=1}^{n} \parenth{X \setminus \cl\of{V_{b_n}}}} & Y &:= \bigcup_{n \in \N} Y_n \\
        Z_n &:= V_{a_n} \cap \parenth{\bigcap_{k=1}^{n} \parenth{X \setminus \cl\of{U_{a_n}}}} & Z &:= \bigcup_{n \in \N} Z_n
    \end{align*}

     Recall that we have $A\ssq X\setminus\cl\of{V_{b_n}}$ and $B\ssq X\setminus \cl\of{U_{a_n}}$ holding for all $n\in \N$ - this is because $V_{b_n}$ was chosen such that $\cl\of{V_b}\cap A=\emptyset=\cl\of{U_a}\cap B$, which was done using the normality of $X$. Now since the $U_{a_n}$ cover $A$, for each $a\in A$ there is some $n\in \N$ with $a\in U_{a_n}$, and since $A \ssq \bigcap_{n \in \N} \parenth{X \setminus \cl\of{V_{b_n}}}$, we know then that $a\in Y_n$. This holds for every $a\in A$, so we know that indeed $A\ssq Y$, and essentially the same argument shows that $B\ssq Z$.

    We now show that for all $m, n \in \Z$, $Y_m \cap Z_n = \emptyset$, which will imply that $Y\cap Z=\emptyset$. To see this, assume for contradiction that there exist some $m, n \in \N$ such that $Y_m\cap Z_n\neq \emptyset \implies \exists x \in Y_m \cap Z_n$ - WLOG we let $m\leq n$. Since $x \in Y_m$ we must have $x \in U_m$, but if $x \in Z_n$ and $m \leq n$, then we must have $x \in X \setminus \cl\of{U_m}\ssq X\setminus \cl\of{U_m}\implies x\notin U_m$. Having reached contradiction, we conclude that for all $m, n \in \Z$, $Y_m \cap Z_n = \emptyset$.

    And now that we have our $Y,Z$ disjoint open sets covering $A,B$ respectively, we can conclude that if $X$ is Lindelöf and $T_3$ then (indeed) $X$ must be $T_4$.
\end{proof}

We are now ready to prove Urysohn's Metrisation Theorem (i.e. that if $X$ is 2nd countable and $T_3$, then $X$ must be metrisable). Remember that (to do so) our plan is to find a homeomorphic copy of $X$ in $[0,1]^{\N}$, which is sufficient because $[0,1]^{\N}$ is metrisable.

\begin{proof}[Proof of \Cref{SP:Urysohn_Metrisation}]
    Let $X$ be a 2nd countable $T_3$ space. By \Cref{Ch3:Lemma:2nd_countable_is_Lindelof}, since $X$ is 2nd countable, we have that $X$ is Lindelöf. To embed $X$ into $[0, 1]^{\N}$, it will suffice to find a countable set $\F$ of continuous functions from $X$ to $[0, 1]$ that separate points from closed sets.

    \textbf{Why?} Well, first enumerate such a set $\F$ as functions $\set{f_1, f_2, \ldots}$ and define $F : X \to [0, 1]^{\N} : x \mapsto \parenth{f_n(x)}_{n \in \N}$. If $\F$ separates points from closed sets, 
    
    \textcolor{red}{teresa pick up here!}
    
    \sorry. Note that it is crucial that $\F$ is countable for this\todo{What?} to work.

    So now let's get to work and actually find such a family!

    We will make use of the fact that $X$ is second countable. This means precisely that the topology of $X$ has a countable basis $B$. For all pairs $\parenth{U, V}$ with $U, V \in B$ and $\cl\of{U} \ssq V$, we observe that $\cl\of{U} \cap \parenth{X \setminus V} = \emptyset$\todo{Is this right??} So define $f_{U, V} : X \to [0, 1]$ by \sorry.

    It just remains to show that $\F$ separates points from closed sets. Let $F \ssq X$ be closed. Fix some $x \in X \setminus F$. Clearly $X \setminus F$ is open, so there is some $V \in B$ such that $x \in V$ and $V \ssq X \setminus F$ (meaning $V \cap F = \emptyset$). By regularity\todo{which follows from...?}, we know that there is an open set $U'$ that is a neighbourhood of $x$ whose closure is contained in $V$. Ie, we have $x \in U'$ and $\cl\of{U'} \ssq V$. Indeed, we can find a \textit{basic} $U \in B$ such that $x \in U \ssq U'$ and $\cl\of{U} \ssq \cl\of{U'} \ssq V$.

    \sorry
\end{proof}

This was a bit of a slog, so let's break it down into parts and understand the weird hypotheses.

First, you have a modest separation property, $T_3$, and a strong countability property, 2nd countability. And really the weaker consequence of being Lindelöf is what makes the argument work, because that gives us $T_4$, which is comforting because metric spaces are $T_4$. We then need to use all our ingredients to construct a clever family of functions that separate points from closed sets, and this allows us to embed our space into $[0, 1]^{\N}$, showing metrisability.

There are even sharper metrisability theorems. This theorem works great, as far as it goes, but the hypotheses are overkill. In particular, they're not \textit{equivalent} to the property of metrisability. A much better theorem is the Nagata-Smirnov theorem, which gives sufficient \textit{and} necessary conditions for metrisability. (Of course, an obvious equivalent condition to metrisability \textit{is just metrisability itself}. We're talking about equivalent conditions that are actually \textit{helpful}/that occur not uncommonly.)