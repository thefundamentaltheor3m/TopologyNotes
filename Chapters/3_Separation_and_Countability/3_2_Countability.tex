\section{Countability Properties}

\begin{boxdefinition}[1st, 2nd Countability]
    Let $X$ be a space, then
    \begin{enumerate}
        \item we say that $X$ is 1st countable to say that every point in $X$ has a countable neighbourhood basis (and...)
        \item we say that $X$ is second countable to say that $X$ has a countable basis.
    \end{enumerate}
\end{boxdefinition}
``one whose relationship with the first two might remain a little questionable for now...''

``those of you who took field theory might hate the name of this next one...''
\begin{boxdefinition}[Separable (``ohhhhhhhhhh...'' insert disappointment)]
We say that $X$ is ``separable'' to say that $X$ has a countable dense subset.
    
\end{boxdefinition}
\begin{boxdefinition}[Lindelof (``Poor Man's Subcover'')]
    We say that $X$ is Lindelof to say that every open covering of $X$ has a countable subcover.
\end{boxdefinition}
Now another definition, to get some feel for the ``geometry'' of a metric space
\begin{boxdefinition}[distance from a point to a set in a metric space]
    Let $(X,d)$ be a metric space, $A\ssq X, A\neq \emptyset$ (to pre-emptively take care of the empty set police) and $x\in X$. Then $d(x, A)=\inf\{d(x,a):a\in A\}$.
\end{boxdefinition}
\begin{boxtheorem}
    $d(x,A)$ is a continuous function of $x$.
\end{boxtheorem}
\begin{proof}
    Let $x,y\in X$ and $a\in A$, then $d(x, A)\leq d(x,a)\leq d(x,y)+d(y,a)\implies d(y,a)\geq d(x, A)-d(x,y)$ for all $a\in A\implies d(y,A)\geq d(x,A)-d(x,y)$. Now one can fill in the details at their leisure.
\end{proof}

With this, we can show that $X$ a metric space implies $X$ is $T_4$:
\begin{proof}
    Let $C,D\ssq X$ with $(X,d)$ is a metric space with $C,D$ closed and $C\cap D=\emptyset$. If we then let $g(x)=d(x,C)-d(x,D)$ [we NEED that $C,D$ are closed and disjoint!], then $g$ is continuous as a difference of two continuous functions. Let $X=g^{-1}[(-\infty, 0)]$ and $V=g^{-1}[(0, \infty)]$ - then maybe we're basically done \sorry.
\end{proof}

\begin{boxdefinition}[Embedding]
    We say $f:X\rightarrow Y$ is an ``embedding'' of $X$ into $Y$ if $f$ is an injective map with the additional constraint that $f$ is a homeomorphism from $X$ to $f[X]$ (where $f[X]\ssq Y$ is endowed with the subspace topology).
\end{boxdefinition}
\begin{boxwarning}
    Crucially, note $f[U]$ need not be open in $Y$ for $U$ open in $X$ - we only require $f[U]$ open in subspace topology of $f[X]\ssq Y$.
\end{boxwarning}
Recall that for spaces $X, Y_i$ ($i\in I$) that $F:X\rightarrow \prod_{i\in I}Y_i$ is continuous iff for all $i$, $\pi_i\circ F$ is continuous. In what follows, we let $f_i:=\pi_i\circ F$ given some $F(x)=(f_i(x))_{i\in I}$.

Our goal now is to find conditions on $f_i:X\rightarrow Y_i$ to ensure that $F$ is an embedding of $X$ in $\prod_{i\in I}Y_i$. To ensure that $F$ is continuous, each $f_i$ needs to be continuous. To be sure $F$ is injective, we need that for all $x,x'\in X$ that $x\neq x'\implies \exists i \in I$ such that $f_i(x)\neq f_i(x')$.

``and you're going to groan when I give you the next definition...''
\begin{boxdefinition}[Separating Points from Closed Sets]
    Given $X, (Y_i)_{i\in I}$ and continuous functions $f_i:X\rightarrow Y_i$ we say that $(f_i)_{i\in I}$ ``separates points from closed sets'' iff for all closed $C\ssq X$ with $x\in X, x\notin C$ here is some $i\in I$ such that $f_i(x)\notin \text{cl}(f_i[C])$.
\end{boxdefinition}
\begin{boxtheorem}
    If $f_i:X\rightarrow Y_i$ with $i\in I$ are such that each $f_i$ separates points, separates points on closed sets with $i\in I$ and is continuous, then $F$ is an embedding.
\end{boxtheorem}
Here is the proof, but ``you're not going to like it'' - is there set-theoretic nonsense??
\begin{proof}
    Given $U\ssq X$ open, we show $F[U]$ is relatively open in $F[X]$. In fact, we will show that for all $x\in U$ that there is $V\ssq \prod_i Y_i$, $F(x)\in V$, with $V$ open and $V\cap F[X]\ssq F[U]$. However observing that $x\notin X\setminus U$ a closed set, as $(f_i)_{i\in I}$ separates points from closed sets, there is $i$ such that $f_i(x)\notin \text{cl}(f_i[X\setminus U])$. 
    
    We now let $E=\text{cl}(f_i[X\setminus U])\ssq Y_i$ with $E$ closed. But then $f_i(x)\in Y_i\setminus E$ (an open set in $Y_i$). If we now let $V=\pi^{-1}[Y_i\setminus E]$, we can find that $V$ is open in $\prod_i Y_i$, and $F(x)\in V$ - we are left only to substantiate this claim.

    Finally then, we let $F(z)=V\cap F[X]$. As $F(z)\in V$, we have that $f_i(z)=\pi_i\circ F(z)\in Y_i\setminus E$. Now ``chasing through definitions'' we claim $z\in U$ and then proceed to observe that $z\notin X\setminus U \implies f_i(z)\in f_i[X\setminus U]$ and $f_i(z)\in \text{cl}(f_i[X\setminus U])=E$.
\end{proof}
So to ensure you get a topological embedding, ensure you separate points from points, then separate points from closed sets.

\section{Regularity}

Throughout this subsection, fix a topological space $X$.

We recall the definition of a regular space.

\begin{boxdefinition}[Regular Space]
    We say that $X$ is \textbf{regular} if for all closed $F \ssq X$ and $x \in X \setminus F$, there are $U, V \ssq X$ that are open and contain $x$ such that $F \ssq V$ and $U \cap V = \emptyset$.
\end{boxdefinition}

There is a strictly stronger notion as well.

\begin{boxdefinition}[Complete Regularity]
    We say $X$ is \textbf{completely regular} if for all $F \ssq X$ closed and $x \in X \setminus F$, there is a continuous function $f : X \to [0, 1]$ such that $f(x) = 0$ and $f[F] = \ssq \set{1}$.
\end{boxdefinition}

It makes complete sense for complete regularity to strictly generalise regularity. (See what I did there?) We do not show this here, but we reassure the reader that it is true.

In this section, we will explore how regularity and complete regularity interact with topological constructions.

\subsection{Regularity and Subspaces}

Fix a subset $A$ of $X$, and consider it as a topological space, endowed with the subspace topology inherited from $X$. Recall that $B \ssq A$ is closed with respect to the subspace topology if and only if $B = A \cap F$ for some $F \ssq X$ closed.

It turns out that subspaces of regular spaces are themselves regular.

\begin{boxlemma}
    If $X$ is regular, then so is $A$.
\end{boxlemma}
\begin{proof}
    Fix $B \ssq A$ closed and fix $a \in A \setminus B$. We know that $B = A \cap F$ for some closed $F \ssq X$. Since $a \notin B$, we must also have $a \notin F$. Then, since $X$ is regular, we can find $U, V \ssq X$ that are open and contain $a$ such that $F \ssq V$ and $U \cap V = \emptyset$. Then, $U \cap A$ and $V \cap A$ are relatively open subsets of $A$ that also contain $a$ and satisfy the properties that $B \ssq V \cap A$ and $\parenth{U \cap A} \cap \parenth{V \cap A} = \emptyset$.
\end{proof}
% Check out Appendix B
%haha, was goinf to put it there (many thanks) - most indeed :DDD
% I think we NEED an appendix dedicated to these kinds of quotes

We can say something analogous about complete regularity.

\begin{boxlemma}
    If $X$ is completely regular then so is $A$.
\end{boxlemma}
\begin{proof}
    Fix $B \ssq A$ relatively closed and $a \in A \setminus B$. We know $B = A \cap F$ for some $F \ssq X$ closed. We know there is some continuous $f : X \to [0, 1]$ such that $f(a) = 0$ and $f[F] = \set{1}$. Consider its restriction $g = f \restriction A$. Obviously $g(a) = 0$ because $a \in A$. Moreover, since $g$ agrees with $f$ on $A$, it is also clear that $g[A] = f[A] \ssq f[F] = \set{1}$.\footnote{I suppose the `empty set brigade' might have some objections but honestly who even cares about such people... (I kinda do but I am choosing to ignore my feelings in the interest of convenience!)}
\end{proof}

\begin{boxproposition}\label{Ch3:Prop:Regular_TFAE}
    The following are equivalent.
    \begin{enumerate}[label = (\arabic*)]
        \item $X$ is regular.
        \item for all $x \in X$ and all open $U \ni x$, there exists an open $V \ni x$ such that $\cl{V} \ssq U$.
    \end{enumerate}
\end{boxproposition}
\begin{proof}
    The picture we want to have in mind is given in \Cref{Ch3:Fig:Regular_TFAE}. \sorry\todo{Finish!}

    \begin{figure}[H]
        \centering
        \begin{tikzpicture}
            \draw (-2,-2) rectangle (4, 4); % TODO: Finish
        \end{tikzpicture}
        \caption{The picture we want to have in mind for \Cref{Ch3:Prop:Regular_TFAE}.}
        \label{Ch3:Fig:Regular_TFAE}
    \end{figure}

    \begin{description}
        \item[\underline{$(1) \implies (2)$.}]
        Fix $x \in X$ and some open $U \ni x$. Evidently, $x \notin X \setminus U$, and since $X \setminus U$ is closed and $X$ is regular, we can find some open $V, W \ssq X$ such that $x \in W$, $X \setminus U \ssq W$, and $W \cap V = \emptyset$. Then, the closure $\cl{V} \ssq X \setminus W$, and this is contained in $U$.

        \item[\underline{$(2) \implies (1)$.}]
        Exercise. \sorry \todo{Finish!}
    \end{description}
\end{proof}

\subsection{Regularity and Products}

Our eventual goal in this subsection will be to show that products of regular spaces are regular.

Throughout this subsection, fix topological spaces $\parenth{X_i}_{i \in I}$. Denote by $X$ the product $\prod_{i \in I} X_i$.

\begin{boxlemma}
    If $\parenth{B_i}_{i \in I}$ are basic open sets of the $X_i$, then
    \begin{align*}
        \prod_{i \in I} \cl{B_i} = \cl{\prod_{i \in I} B_i}
    \end{align*}
\end{boxlemma}
\begin{proof}
    \sorry % I'm usually quick with these proofs - but unfortunately I didn't quite get this one... briefly zoned out and looked at some other things (sorry)
\end{proof}

We can now show that a product of regular spaces is regular.

\begin{boxtheorem}
    If each $X_i$ is regular, then so is $X$.
\end{boxtheorem}
\begin{proof}
    Fix $x = \parenth{x_i}_{i \in I}$ lie in some open set $U \ssq X$. We can write $U_i$ as a product of open sets $U_i$, with only finitely many $U_i$ being \textit{properly} contained in $X_i$. \sorry
\end{proof}

We will continue next time with similar results on \textit{complete} regularity.

\section{Urysohn's Lemma}

In this section, we develop the tools needed to prove Urysohn's Lemma, which (according to Professor Cummings) should really be a \textit{theorem}.

It goes as follows.

\begin{boxtheorem}[Urysohn's Lemma]\label{SP:Urysohn}
    If $X$ is a normal topological space, for all closed $C, D \ssq X$, if $C \cap D = \emptyset$ then there is a continuous $f : X \to [0, 1]$ such that $f[C] \ssq \set{0}$ and $f[D] \ssq \set{1}$.
\end{boxtheorem}

In some sense, normality tells us we can `separate closed sets using open sets', and Urysohn's Lemma says that if we can do that, then we can also `separate closed sets using continuous functions'.

For the remainder of this section, fix a normal topological space $X$. Our proof will be \textit{constructive}: we will give an explicit function.

%as you wish of course - as you saw, when I was note-taking (for that brief brief period) I made no new subsections, and also last I checked there is more missing on the separation properties front

% I think this should be a subsection inside the more on separation properties section. I also think there should be more subsections of that section.

\begin{boxlemma}\label{Ch3:Lemma:Urysohn_helper}
    Let $F$ be a closed subset of $X$ and $O$ an open subset of $X$. If $F \ssq O$, then there exist $O_1, F_1 \ssq X$ such that $O_1$ is open, $F_1$ is closed, and
    \begin{align*}
        F \ssq O_1 \ssq F_1 \ssq O
    \end{align*}
\end{boxlemma}
% The picture we want to have in mind is the following:
% \begin{figure}
%     \centering
%     \begin{tikzpicture}
        
%     \end{tikzpicture}
%     \caption{Caption}
%     \label{fig:placeholder}
% \end{figure}
\begin{proof}
    The proof is just an ``irritating game involving naïve set theory''.

    If $F \ssq O$, then $F \cap \parenth{X \setminus O} = \emptyset$. Normality then tells us that we have disjoint open sets $U, V \ssq X$ such that $F \ssq U$ and $X \setminus O \ssq V$. Simply take $O_1 = U$ and $F_1 = X \setminus V$.
\end{proof}

We will also briefly explain what the dyadic rationals are.

\begin{boxdefinition}[Dyadic Rational]
    We say $q \in \Q$ is \textbf{dyadic} if there is some $c \in \Z$ and $i \in \N$ such that $q = \frac{c}{2^i}$.
\end{boxdefinition}

The dyadic rationals are dense in $\R$: indeed, this is why real numbers admit binary expansions.

\begin{boxlnotation}
    Denote by $D$ the set of all dyadic rationals contained in $(0, 1)$.
\end{boxlnotation} % Local notation :DD

We are now ready to prove Urysohn's Lemma.

\begin{proof}[Proof of Urysohn's Lemma (\Cref{SP:Urysohn}).]
    We will proceed by constructing sets $\setst{U_q}{q \in D}$ such that
    \begin{enumerate}[noitemsep]
        \item $C \ssq U_q \ssq X \setminus D$ for all $q \in D$.
        \item If $p, q \in D$ and $p < q$ then $\cl{U_p} \ssq U_q$.
    \end{enumerate}
    To do this, we will actually build $\setst{U_q, V_q}{q \in D}$ such that for all $q \in D$,
    \begin{enumerate}[noitemsep]
        \item $U_q$ is open
        \item $V_q$ is closed
        \item $V_q \ssq U_r$ for some $r > q$
    \end{enumerate}
    At this stage, it will be useful to draw a picture.
    % Can I live-TeX a picture? Let's find out...
    % as per usual, incredibly impressive
    % Well it would've been if it had worked...
    % quite honestly, nonetheless...
    % \begin{figure}[H]
    %     \centering
    %     % TODO: Translate this left
    %     \begin{tikzpicture}
    %         \draw (0, 0) circle (5pt); % C
    %         \draw (1, 0) circle (5pt); % D

    %         % U 1/4, V 1/4
    %         \draw [blue,thin,domain=-30:30] plot (0.2 * {cos(\x)}, 0.2 * {sin(\x)});
    %         \draw [blue,thin,domain=-30:30] plot (0.3 * {cos(\x)}, 0.3 * {sin(\x)});
            
    %         % U 1/2, V 1/2
    %         \draw [blue,thin,domain=-30:30] plot ({0.4 * cos(\x)}, {0.4 * sin(\x)});
    %         \draw [blue,thin,domain=-30:30] plot ({0.6 * cos(\x)}, {0.6 * sin(\x)});
    %     \end{tikzpicture}
    %     \caption{Caption}
    %     \label{fig:placeholder}
    % \end{figure}
    Here's a more formal way of describing the construction.
    \begin{enumerate}
        \item By \Cref{Ch3:Lemma:Urysohn_helper}, we can \sorry
        \item Then, given $C \ssq U_{\frac{1}{2}}$, we can apply \Cref{Ch3:Lemma:Urysohn_helper} to get $U_{\frac{1}{4}}$ open and $V_{\frac{1}{4}}$ closed such that
        \begin{align*}
            C \ssq U_{\frac{1}{4}} \ssq V_{\frac{1}{4}} \ssq U_{\frac{1}{2}}
        \end{align*}
        Similarly, since $V_{\frac{1}{2}} \ssq X \setminus D$, we can again apply \Cref{Ch3:Lemma:Urysohn_helper} to get $U_{\frac{3}{4}}$ open and $V_{\frac{3}{4}}$ such that
        \begin{align*}
            V_{\frac{1}{2}} \ssq U_{\frac{3}{4}} \ssq V_{\frac{3}{4}} \ssq X \setminus D
        \end{align*}
    \end{enumerate}
    It will be incredibly useful to bear in mind that for all dyadic rationals $p < q$, we get
    \begin{align*}
        C \ssq U_p \ssq X \setminus D
        \qquad \text{ and } \qquad
        \cl{U_p} \ssq U_{q^2}
    \end{align*}
    % Muchas gracias senora
    %por supuesto
    We will now get on to the construction of our continuous function. To do so, we define $f:X \to [0,1]$ such that, if there us no $r\in D$ with $x\in U_r$, then we let $f(x) := 1$ - otherwise, we define $f(x) := \inf\!\setst{r\in D}{x\in U_r}$.

    It is ``easy'' to see that if $x\in D$, then $f(x)=1$, as $x\in C\implies f(x)=\inf(D)=0$. However we must now get on to the trickier part...we claim that for all $x \in X$ and $p \in D$,
    \begin{enumerate}[label = (\alph*), noitemsep]
        \item If $x \in \cl{U_p}$, $f(x) \leq p$
        \item If $x \notin \cl{U_p}$, $f(x) \geq p$
    \end{enumerate}
    Indeed, if $x \in \cl{U_p}$, then $x \in U_q$ for all $q \in D$ with $q > p$, so $f(x) \leq p$. On the other hand, if $x \notin \cl{U_p}$, then assume, for contradiction, that $f(x) < p$. Then we can easily see that there must be some $r < p$ and $x \in U_r$ such that $f(x) \leq r < p$. \sorry

    Taking the contrapositives of the above claims, we see that for all $x \in X$ and $p \in D$,
    \begin{enumerate}[label = (\alph*), noitemsep]
        \item If $f(x) > p$ then $x \notin \cl{U_p}$ (and thus $x \notin U_p$, so $f(x) \geq p$).
        \item If $f(x) < p$ then $x \in U_p$ (and thus $f(x) \leq p$).
    \end{enumerate}

    Consider the following basis for the Euclidean (subspace) topology on $[0,1]$:
    \begin{align*}
        \setst{[0, a)]}{0 < a} \cup
        \setst{(a, b)}{0 \leq a < b \leq 1} \cup
        \setst{(b, 1]}{b < 1}
    \end{align*}
    We show $f$ is continuous by showing that the pre-image in $f$ of every basic open subset of $[0, 1]$ is open.
    
    We begin by showing that $f\inv\!\brac{(a, b)}$ is open. We do this by showing every point has an open neighbourhood. Fix $x \in f\inv\!\brac{(a, b)}$. We know that $a < f(x) < b$. Find $p, q \in D$ such that $a < p < f(x) < q < b$. Then $f(x) > p$, so $x \notin \cl{U_p}$. Similarly, $f(x) < q$, so $x \in U_q$. Now, let $V = U_q \setminus \cl{U_p}$. Observe that $x \in V$. Moreover, for all $y \in V$, $y \in U_q$, so $f(y) \leq q$, and $y \notin U_p$, so $f(y) \geq p$. Then, we can see that $f[V] \ssq [p, q] \ssq (a, b)$.
    % I'm still on the left board, please feel free to type away

    %sure - this was bottom of the left board - once I can see again, haha
    
    So for $f\inv [[0,a)]$ we let $f(x)<a$ and find $p\in D$ with $f(x)<p<a$, $x\in U_p$ and $f[U_p]\ssq [0,p]\ssq [0,a)$. Meanwhile, for $f\inv [(b,1]]$ we let $f(x)>b$ we find $q\in D$ such that $b<q<f(x)$ and the $x\notin \cl{U_q}$ and $y\in X\setminus \cl{U_q}$ then $y\notin U_q\implies f(y)\geq q>b$. 
\end{proof}

The good news is that we have a very rich supply of normal spaces. Indeed, we know that Metric spaces are Hausdorff and normal ($T_4$). 